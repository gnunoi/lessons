\documentclass[landscape, 320mm, 180mm]{beamer}
\usepackage[utf8]{inputenc}
\usepackage{graphicx}
\usepackage{tikz}
\usepackage{multicol}
\usepackage{ctex}
\usetikzlibrary{shapes.geometric, arrows, arrows.meta}

\title{中学生学习 C++ 的好处}
\author{}
\date{}

\begin{document}

\frame{\titlepage}

% 第一页:时间轴
\begin{frame}{信息学竞赛时间轴}
    \begin{tikzpicture}[node distance=2cm]
        % 定义节点样式
        \tikzstyle{event} = [rectangle, draw, fill=blue!20, text centered, text width=3cm]
        \tikzstyle{line} = [draw, thick, -{Circle[open]}, line width=0.5mm]
        
        % 时间轴
        \node (start) [event] {1984年:NOI 创立};
        \node (apio) [event, below of=start] {2007年:APIO 创立};
        \node (ioi) [event, below of=apio] {1989年:IOI 创立};
        \node (noip) [event, below of=ioi] {1995年:NOIP 创立};
        \node (csp) [event, below of=noip] {2019年:CSP-J/S 创立};
        
        % 连接线
        \draw[line] (start) -- (apio);
        \draw[line] (apio) -- (ioi);
        \draw[line] (ioi) -- (noip);
        \draw[line] (noip) -- (csp);
    \end{tikzpicture}
\end{frame}

% 第二页:CSP 介绍
\begin{frame}{CSP(Certified Software Professional)}
    \begin{itemize}
        \item \textbf{创立年份:} 2019年
        \item \textbf{面向对象:} 所有年龄段
        \item \textbf{分级:} 入门级(CSP-J)和提高级(CSP-S)
        \item \textbf{赛制:} 第一轮笔试,第二轮上机考试
        \item \textbf{内容:} 计算机基础知识、算法与数学知识
    \end{itemize}
\end{frame}

% 第三页:NOIP 介绍
\begin{frame}{NOIP(全国青少年信息学奥林匹克联赛)}
    \begin{itemize}
        \item \textbf{创立年份:} 1995年
        \item \textbf{面向对象:} 高中生
        \item \textbf{赛制:} 初赛和复赛
        \item \textbf{内容:} 计算机基础知识、算法与数学知识
        \item \textbf{备注:} 2019年起,NOIP 仅有一个组别,面向提高组水平选手
    \end{itemize}
\end{frame}

% 第四页:NOI 介绍
\begin{frame}{NOI(全国信息学奥林匹克竞赛)}
    \begin{itemize}
        \item \textbf{创立年份:} 1984年
        \item \textbf{面向对象:} 各省选拔出的优秀选手
        \item \textbf{赛制:} 现场赛和网络赛
        \item \textbf{内容:} 高级算法、复杂数据结构和创新编程解决方案
        \item \textbf{备注:} 前50名金牌选手组成国家集训队,获得保送资格
    \end{itemize}
\end{frame}

% 第五页:APIO 介绍
\begin{frame}{APIO(亚太地区信息学奥林匹克竞赛)}
    \begin{itemize}
        \item \textbf{创立年份:} 2007年
        \item \textbf{面向对象:} 亚太地区在校中学生
        \item \textbf{赛制:} 线上竞赛
        \item \textbf{内容:} 算法和编程
        \item \textbf{备注:} 每年五月初举办中国赛区镜像赛
    \end{itemize}
\end{frame}

% 第六页:IOI 介绍
\begin{frame}{IOI(国际信息学奥林匹克竞赛)}
    \begin{itemize}
        \item \textbf{创立年份:} 1989年
        \item \textbf{面向对象:} 全球中学生
        \item \textbf{赛制:} 现场赛
        \item \textbf{内容:} 高级算法、复杂数据结构和创新编程解决方案
        \item \textbf{备注:} 每年举办一次,每个国家有四人参赛
    \end{itemize}
\end{frame}

% 第七页:五大学科竞赛简介
\begin{frame}{五大学科竞赛简介}
    \begin{itemize}
        \item \textbf{数学奥林匹克:} 1959年
        \item \textbf{物理奥林匹克:} 1967年
        \item \textbf{化学奥林匹克:} 1968年
        \item \textbf{生物奥林匹克:} 1990年
        \item \textbf{信息学奥林匹克:} 1989年
    \end{itemize}
\end{frame}

% 第八页:NOI 保送政策
\begin{frame}{NOI 保送政策}
    \begin{itemize}
        \item \textbf{金牌:} 前50名,组成国家集训队,获得保送资格
        \item \textbf{银牌:} 部分高校可破格入围强基计划
        \item \textbf{铜牌:} 部分高校可破格入围综合评价招生
    \end{itemize}
\end{frame}

% 第九页:学习 C++ 的好处
\begin{frame}{学习 C++ 的好处}
    \begin{itemize}
        \item \textbf{高效性能:} 适用于需要高性能的应用
        \item \textbf{广泛应用:} 游戏开发、系统编程、嵌入式系统等
        \item \textbf{深入理解计算机原理:} 了解内存管理、指针等底层概念
        \item \textbf{良好的
::contentReference[oaicite:0]{index=0}
 
