\documentclass{ctexbeamer}

\makeatletter
\def\input@path{{../../../styles}}  % 
\makeatother
\usepackage{ubeamer}
\uBigPaper
\uSetBigFont

\title{一元一次方程基础概念}
\author{丁保华}
\date{\today}

\begin{document}

\begin{frame}
  \titlepage
\end{frame}

\section{方程相关概念}

\begin{frame}{什么是方程}
  \begin{itemize}
    \item 方程是指含有未知数的等式。\\
    如 \(2x + 3 = 7\) \\
    它表达了未知数与已知数之间的关系。
    \item 方程的核心在于通过等式来寻找未知数的值
    \item 是数学中解决实际问题的重要工具。
  \end{itemize}
\end{frame}

\begin{frame}{什么是方程的元}
  \begin{itemize}
    \item 方程的元是指方程中所含有的未知数的个数
    \item 如 \(3x + 2y = 6\) 是二元方程。
    \item 一元方程只有一个未知数。
    \item 这是方程分类的重要依据之一。
  \end{itemize}
\end{frame}

\begin{frame}{什么是方程的次}
  \begin{itemize}
    \item 方程的次是指方程中未知数的最高次数
    \item 如 \(x^2 + 3x + 2 = 0\) 是一元二次方程。
    \item 一元一次方程的未知数最高次数为1,这是其显著特点。
  \end{itemize}
\end{frame}

\begin{frame}{什么是一元一次方程}
  \begin{itemize}
    \item 一元一次方程是指只含有一个未知数,且未知数的最高次数为1的方程。
    \item 如 \(4x - 5 = 3\)。
    \item 它是最简单的方程类型,是进一步学习其他方程的基础。
  \end{itemize}
\end{frame}

\begin{frame}{什么是方程的解}
  \begin{itemize}
    \item 方程的解是指能使方程左右两边相等的未知数的值
    \item 如 \(2x = 4\) 的解是 \(x = 2\)。
    \item 求解方程的过程就是找到这个能使等式成立的未知数的值。
  \end{itemize}
\end{frame}

\section{等式的基本性质及变形}

\begin{frame}{等式性质1}
  \begin{itemize}
    \item 等式两边同时加上或减去同一个数,等式仍然成立。
    \item 如 \(3x + 2 = 8\),两边同时减去2得 \(3x = 6\)。
    \item 这个性质是方程变形的基础,可用于消去等式中的常数项。
  \end{itemize}
\end{frame}

\begin{frame}{等式性质2}
  \begin{itemize}
    \item 等式两边同时乘以或除以同一个不为零的数,等式仍然成立。
    \item 如 \(4x = 12\),两边同时除以4得 \(x = 3\)。
    \item 这个性质可用于将未知数的系数化为1,从而求出未知数的值。
  \end{itemize}
\end{frame}

\begin{frame}{等式性质的应用}
  \begin{itemize}
    \item 通过等式性质,可将复杂的方程逐步化简为简单形式。
    \item 如 \(2x - 3 = 5\),先加3得 \(2x = 8\),再除以2得 \(x = 4\)。
    \item 灵活运用等式性质是解一元一次方程的关键。
  \end{itemize}
\end{frame}

\section{例题与练习}

\begin{frame}{例题1}
  \begin{itemize}
    \item 解方程 $3x - 5 = 10$。
    \item 先加5得 $3x = 15$,
    \item 再除以3得 $x = 5$。
    \item 本题考查了等式性质1和2的简单应用,通过逐步变形求解。
  \end{itemize}
\end{frame}

\begin{frame}{例题2}
  \begin{itemize}
    \item 解方程 \(2x + \dfrac{1}{2}x = 6\)。
    \item 先合并同类项得 \(\dfrac{5}{2}x = 6\),
    \item 再乘以 \(\dfrac{2}{5}\) 得 \(x = \dfrac{12}{5}\)。
    \item 本题涉及分数系数,需先化简方程再求解。
  \end{itemize}
\end{frame}

\begin{frame}{例题3}
  \begin{itemize}
    \item 解方程 \(4x - 3 = 2x + 5\)。
    \item 先移项得 \(2x = 8\),
    \item 再除以2得 \(x = 4\)。
    \item 本题考查了移项的概念,通过将未知数项移到一边,常数项移到另一边求解。
  \end{itemize}
\end{frame}

\section{练习题}

\begin{frame}{练习题1}
  \begin{itemize}
    \item 解方程 \(5x + 2 = 17\)。
    \item 先减2得 \(5x = 15\),
    \item 再除以5得 \(x = 3\)。
    \item 练习等式性质1和2的运用。
  \end{itemize}
\end{frame}

\begin{frame}{练习题2}
  \begin{itemize}
    \item 解方程 \(3x - \dfrac{1}{3}x = 4\)。
    \item 先合并同类项得 \(\dfrac{8}{3}x = 4\),
    \item 再乘以 \(\dfrac{3}{8}\) 得 \(x = \dfrac{3}{2}\)。
    \item 练习分数系数的处理。
  \end{itemize}
\end{frame}

\begin{frame}{练习题3}
  \begin{itemize}
    \item 解方程 \(2x - 4 = 6x + 2\)。
    \item 先移项得 \(-4x = 6\),
    \item 再除以-4得 \(x = -\dfrac{3}{2}\)。
    \item 练习移项及负系数的处理。
  \end{itemize}
\end{frame}

\begin{frame}{练习题4}
  \begin{itemize}
    \item 解方程 \(\dfrac{1}{4}x + 3 = 5\)。
    \item 先减3得 \(\dfrac{1}{4}x = 2\),
    \item 再乘以4得 \(x = 8\)。
    \item 练习分数系数及等式变形。
  \end{itemize}
\end{frame}

\begin{frame}{练习题5}
  \begin{itemize}
    \item 解方程 \(6x - 2 = 4x + 6\)。
    \item 先移项得 \(2x = 8\),
    \item 再除以2得 \(x = 4\)。
    \item 综合练习等式性质及移项。
  \end{itemize}
\end{frame}

\section{习题}

\begin{frame}{习题1}
  \begin{itemize}
    \item 解方程 \(7x - 3 = 2x + 12\)。
    \item 先移项得 \(5x = 15\),
    \item 再除以5得 \(x = 3\)。
    \item 练习移项及求解。
  \end{itemize}
\end{frame}

\begin{frame}{习题2}
  \begin{itemize}
    \item 解方程 \(3x + \dfrac{2}{3}x = 10\)。
    \item 先合并同类项得 \(\dfrac{11}{3}x = 10\),
    \item 再乘以 \(\dfrac{3}{11}\) 得 \(x = \dfrac{30}{11}\)。
    \item 练习分数系数的处理。
  \end{itemize}
\end{frame}

\begin{frame}{习题3}
  \begin{itemize}
    \item 解方程 \(4x - 5 = 3x + 7\)。
    \item 先移项得 \(x = 12\)。
    \item 练习简单移项求解。
  \end{itemize}
\end{frame}

\begin{frame}{习题4}
  \begin{itemize}
    \item 解方程 \(5x - \dfrac{1}{5}x = 9\)。
    \item 先合并同类项得 \(\dfrac{24}{5}x = 9\),
    \item 再乘以 \(\dfrac{5}{24}\) 得 \(x = \dfrac{15}{8}\)。
    \item 练习分数系数的化简与求解。
  \end{itemize}
\end{frame}

\begin{frame}{习题5}
  \begin{itemize}
    \item 解方程 \(2x + 3 = 6x - 9\)。
    \item 先移项得 \(-4x = -12\),
    \item 再除以-4得 \(x = 3\)。
    \item 练习移项及负系数的处理。
  \end{itemize}
\end{frame}

\begin{frame}{习题6}
  \begin{itemize}
    \item 解方程 \(\dfrac{3}{4}x + 2 = 5\)。
    \item 先减2得 \(\dfrac{3}{4}x = 3\),
    \item 再乘以 \(\dfrac{4}{3}\) 得 \(x = 4\)。
    \item 练习分数系数及等式变形。
  \end{itemize}
\end{frame}

\begin{frame}{习题7}
  \begin{itemize}
    \item 解方程 \(8x - 4 = 6x + 10\)。
    \item 先移项得 \(2x = 14\),
    \item 再除以2得 \(x = 7\)。
    \item 练习移项及求解。
  \end{itemize}
\end{frame}

\begin{frame}{习题8}
  \begin{itemize}
    \item 某班学生分组做实验,若每组4人,则多2人;若每组5人,则少3人。求该班学生人数。
    \item 解:设该班有 \(x\) 人,
    \item 根据题意得 \(4 \times \left(\dfrac{x - 2}{4}\right) = 5 \times \left(\dfrac{x + 3}{5}\right)\),
    \item 解得 \(x = 27\)。
    \item 练习应用题的列方程与求解。
  \end{itemize}
\end{frame}

\begin{frame}{习题9}
  \begin{itemize}
    \item 甲、乙两人从相距100千米的两地同时出发,相向而行。甲每小时行15千米,乙每小时行10千米。问几小时后两人相遇?
    \item 解:设 \(x\) 小时后相遇,
    \item 根据题意得 \(15x + 10x = 100\),
    \item 解得 \(x = 4\)。
    \item 练习应用题的列方程与求解。
  \end{itemize}
\end{frame}

\begin{frame}{习题10}
  \begin{itemize}
    \item 某工厂计划生产一批零件,若每天生产50个,则比计划晚2天完成;若每天生产60个,则比计划提前1天完成。问计划生产多少天?
    \item 解:设计划生产 \(x\) 天,
    \item 根据题意得 \(50(x + 2) = 60(x - 1)\),
    \item 解得 \(x = 16\)。
    \item 练习应用题的列方程与求解。
  \end{itemize}
\end{frame}

\section{教学反思}

\begin{frame}{反思问题}
  \begin{itemize}
    \item 方程的变形规则是什么?
      \begin{itemize}
        \item 方程的变形规则是基于等式的基本性质,通过加减乘除等运算,将方程逐步化简为未知数的系数为1的形式,从而求出未知数的值。
      \end{itemize}
    \item 什么叫做移项?
      \begin{itemize}
        \item 移项是指将方程中的某一项从等式的一边移到另一边,同时改变该项的符号。
        \item 其目的是将未知数项集中到一边,常数项集中到另一边,便于求解。
      \end{itemize}
  \end{itemize}
\end{frame}

\begin{frame}{如何将未知数的系数化为1}
  \begin{itemize}
    \item 将未知数的系数化为1,
    \item 可通过等式性质2,
    \item 即等式两边同时除以未知数的系数来实现。
    \item 如 \(3x = 9\),两边同时除以3得 \(x = 3\)。
  \end{itemize}
\end{frame}

\begin{frame}{将方程进行适当的变形,最终得到什么形式}
  \begin{itemize}
    \item 将方程进行适当的变形,
    \item 最终得到 \(x = a\) 的形式,其中 \(a\) 是方程的解。
  \end{itemize}
\end{frame}

\begin{frame}{通过本节课的学习,有什么收获?需要注意什么}
  \begin{itemize}
    \item 收获:掌握了等式的基本性质及其应用,
    \item 学会了如何解一元一次方程,
    \item 包括移项、合并同类项、系数化为1等步骤。
    \item 需要注意:在变形过程中要严格遵循等式的基本性质,确保每一步的变形都是正确的;
    \item 在解应用题时,要准确地列出方程,理解题意是关键。
  \end{itemize}
\end{frame}

\section{一元一次方程的发展历史}

\begin{frame}{古代起源}
  \begin{itemize}
    \item 一元一次方程的起源可追溯到古代文明,
    \item 如古埃及的《莱因德纸草书》中就出现了类似一元一次方程的问题。
  \end{itemize}
\end{frame}

\begin{frame}{古代中国}
  \begin{itemize}
    \item 在中国古代数学著作《九章算术》中,
    \item 已有关于一元一次方程的解法,
    \item 采用算筹进行计算。
  \end{itemize}
\end{frame}

\begin{frame}{近代发展}
  \begin{itemize}
    \item 随着数学的发展,
    \item 一元一次方程的求解方法逐渐系统化,
    \item 成为代数学的基础内容之一。
  \end{itemize}
\end{frame}

\begin{frame}{现代意义}
  \begin{itemize}
    \item 一元一次方程在现代数学中具有重要地位,
    \item 是解决实际问题的重要工具,
    \item 广泛应用于物理、化学、经济等领域。
  \end{itemize}
\end{frame}

\end{document}