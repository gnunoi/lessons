\documentclass{article}
\usepackage{ctex}
\usepackage{xcolor}
\usepackage{listings}
\usepackage{fontspec}
\usepackage[a4paper, margin=1in]{geometry}
\usepackage{multirow}
\usepackage{tabularx}

\lstset{
    basicstyle=\ttfamily\normalsize,          % 设置基本字体
    keywordstyle=\color{orange}\bfseries,     % 关键字用橙色和加粗显示
    commentstyle=\color{red},                 % 注释用红色显示
    stringstyle=\color{blue},                 % 字符串用蓝色显示
    numbers=left,                             % 在左侧显示行号
    numberstyle=\small\color[RGB]{96,96,96},  % 行号为灰色且字号为小
    stepnumber=1,                             % 每1行显示一个行号
    frame=single,                             % 给代码加框
    backgroundcolor=\color[RGB]{224,224,224}, % 背景色为浅灰色
    showstringspaces=false,                   % 不显示字符串中的空格
    tabsize=2,                                % 设置tab键为2个空格宽    
}

\title{Python编程高级讲义}
\author{你的名字}
\date{\today}

\begin{document}
\maketitle

\section{数据类型与数据结构}
\subsection{基础数据结构}
\begin{lstlisting}[language=Python]
# List(可变序列)
matrix = [
    [1, 2, 3],
    [4, 5, 6],
    [7, 8, 9]
]

# Tuple(不可变序列)
color_rgb = (255, 128, 0)

# Dict(键值对映射)
employee = {
    "name": "John",
    "age": 30,
    "skills": ["Python", "SQL", "Linux"],
    "contact": {"email": "john@example.com", "phone": "123-4567"}
}

# 数据结构组合示例
company = {
    "departments": [
        ("IT", {"head": "Alice", "members": 15}),
        ("HR", {"head": "Bob", "members": 8})
    ]
}
\end{lstlisting}

\section{控制结构}
\subsection{条件控制}
\begin{lstlisting}[language=Python]
def evaluate_temp(temp):
    if temp > 100:
        status = "BOILING"
    elif 50 <= temp <= 100:
        status = "HOT"
    elif 20 <= temp < 50:
        status = "WARM"
    else:
        status = "COLD"
    return status

print(evaluate_temp(75))  # 输出: HOT
\end{lstlisting}

\subsection{循环结构}
\begin{lstlisting}[language=Python]
# 列表推导式示例
squares = [x**2 for x in range(10) if x % 2 == 0]
# 结果: [0, 4, 16, 36, 64]

# 嵌套循环矩阵转置
matrix = [[1,2,3], [4,5,6], [7,8,9]]
transposed = [[row[i] for row in matrix] for i in range(3)]
# 结果: [[1,4,7], [2,5,8], [3,6,9]]

# While循环实现斐波那契数列
a, b = 0, 1
while a < 100:
    print(a, end=' ')
    a, b = b, a+b
# 输出: 0 1 1 2 3 5 8 13 21 34 55 89 
\end{lstlisting}

\section{八大排序算法实现}
\subsection{算法概览}
\begin{tabularx}{\textwidth}{|l|l|l|}
\hline
算法名称 & 时间复杂度 & 核心思想 \\
\hline
冒泡排序 & O(n²) & 相邻元素交换 \\
选择排序 & O(n²) & 选择最小元素 \\
插入排序 & O(n²) & 构建有序序列 \\
希尔排序 & O(n log n) & 分组插入排序 \\
归并排序 & O(n log n) & 分治与合并 \\
快速排序 & O(n log n) & 基准值分区 \\
堆排序 & O(n log n) & 构建二叉堆 \\
基数排序 & O(nk) & 按位分配收集 \\
\hline
\end{tabularx}

\subsection{算法实现(部分示例)}
\begin{lstlisting}[language=Python]
def merge_sort(arr):
    """归并排序实现"""
    if len(arr) <= 1:
        return arr
    
    mid = len(arr) // 2
    left = merge_sort(arr[:mid])
    right = merge_sort(arr[mid:])
    
    return merge(left, right)

def merge(left, right):
    result = []
    i = j = 0
    while i < len(left) and j < len(right):
        if left[i] < right[j]:
            result.append(left[i])
            i += 1
        else:
            result.append(right[j])
            j += 1
    result += left[i:]
    result += right[j:]
    return result

def heap_sort(arr):
    """堆排序实现"""
    def sift_down(start, end):
        root = start
        while True:
            child = 2 * root + 1
            if child > end:
                break
            if child + 1 <= end and arr[child] < arr[child + 1]:
                child += 1
            if arr[root] < arr[child]:
                arr[root], arr[child] = arr[child], arr[root]
                root = child
            else:
                break
    
    # 构建最大堆
    for start in range((len(arr)-2)//2, -1, -1):
        sift_down(start, len(arr)-1)
    
    # 堆排序
    for end in range(len(arr)-1, 0, -1):
        arr[0], arr[end] = arr[end], arr[0]
        sift_down(0, end-1)
    return arr
\end{lstlisting}

\section{类与对象:八皇后问题}
\begin{lstlisting}[language=Python]
class EightQueensSolver:
    def __init__(self, size=8):
        self.size = size
        self.solutions = []
    
    def solve(self):
        self._place_queen([], [], [])
        return self.solutions
    
    def _place_queen(self, positions, diffs, sums):
        row = len(positions)
        if row == self.size:
            self.solutions.append(positions)
            return
        
        for col in range(self.size):
            if col not in positions and \
               row-col not in diffs and \
               row+col not in sums:
                self._place_queen(
                    positions + [col],
                    diffs + [row - col],
                    sums + [row + col]
                )

# 使用示例
solver = EightQueensSolver()
solutions = solver.solve()
print(f"找到 {len(solutions)} 种解法")
print("第一种解法:", solutions[0])

# 输出:
# 找到 92 种解法
# 第一种解法: [0, 4, 7, 5, 2, 6, 1, 3]

# 可视化输出示例
def print_solution(solution):
    for col in solution:
        line = ['□'] * len(solution)
        line[col] = '♛'
        print(' '.join(line))

print_solution(solutions[0])
\end{lstlisting}

\subsection{排序算法测试}
\begin{lstlisting}[language=Python]
if __name__ == "__main__":
    test_data = [64, 34, 25, 12, 22, 11, 90, 54]
    
    algorithms = {
        "冒泡排序": bubble_sort,
        "快速排序": quick_sort,
        "归并排序": merge_sort,
        "堆排序": heap_sort,
        # 其他算法...
    }
    
    for name, func in algorithms.items():
        print(f"{name}: {func(test_data.copy())}")
\end{lstlisting}
\end{document}