\documentclass{ctexart}
\usepackage{xcolor}
\usepackage{listings}
\usepackage{fontspec}
\usepackage[a4paper, margin=1in]{geometry}
\usepackage{amsmath}

\lstset{
    basicstyle=\ttfamily\normalsize,
    backgroundcolor=\color[rgb]{0.9,0.95,0.95}, % 背景色为浅灰色
    keywordstyle=\color{blue},
    commentstyle=\color{green!60!black},
    stringstyle=\color{orange},
    showstringspaces=false,
    breaklines=true,
    frame=single, % t:顶部,b:底部
    numbers=left,                             % 在左侧显示行号
    numberstyle=\small\color[RGB]{96,96,96},  % 行号为灰色且字号为小
    stepnumber=1,                             % 每1行显示一个行号
    numberstyle=\color{gray},
    tabsize=4,
    escapeinside=||
}

\begin{document}

\section{数据类型与数据结构}
\subsection{列表(List)}
列表是Python中最常用的可变序列结构,具有以下特点:
\begin{itemize}
\item 元素可修改,支持动态增删
\item 用方括号[]定义,元素用逗号分隔
\item 支持嵌套形成多维数据结构
\end{itemize}

\begin{lstlisting}[language=Python]
# 二维列表示例
matrix = [
    [1, 2, 3],
    [4, 5, 6],
    [7, 8, 9]
]
print(matrix[1][2])  # 输出: 6
\end{lstlisting}

\subsection{字典(Dict)}
字典是键值对的无序集合,具有以下特征:
\begin{itemize}
\item 通过键(key)实现快速查找,键必须不可变
\item 用花括号\{\}定义,键值对用冒号:分隔
\item 支持任意数据类型的嵌套组合
\end{itemize}

\begin{lstlisting}[language=Python]
# 字典嵌套示例
university = {
    "name": "清华大学",
    "departments": {
        "CS": {"students": 500, "courses": 32},
        "EE": {"students": 300, "courses": 28}
    }
}
print(university["departments"]["CS"]["courses"])  # 输出: 32
\end{lstlisting}

\section{控制结构}
\subsection{条件语句执行逻辑}
条件语句通过布尔表达式控制程序流,注意:
\begin{itemize}
\item 使用缩进定义代码块范围
\item elif可进行多重条件判断
\item 条件表达式可进行链式比较(如 1 <= x < 5)
\end{itemize}

\begin{lstlisting}[language=Python]
# 成绩评级系统
score = 88
if score >= 90:
    grade = "A"
elif 80 <= score < 90:
    grade = "B"  # 执行此分支
else:
    grade = "C"
print(f"成绩等级:{grade}")  # 输出: 成绩等级:B
\end{lstlisting}

\section{排序算法专题}
\subsection{快速排序原理}
快速排序采用分治策略,包含三个步骤:
\begin{enumerate}
\item 选择基准元素(pivot)
\item 分区操作:将元素分为小于和大于基准的两个子序列
\item 递归地对子序列排序
\end{enumerate}
时间复杂度:平均情况$O(n \log n)$,最坏情况$O(n^2)$

\begin{lstlisting}[language=Python]
def quick_sort(arr):
    if len(arr) <= 1:
        return arr
    pivot = arr[len(arr)//2]
    left = [x for x in arr if x < pivot]
    middle = [x for x in arr if x == pivot]
    right = [x for x in arr if x > pivot]
    return quick_sort(left) + middle + quick_sort(right)
\end{lstlisting}

\subsection{堆排序实现}
堆排序基于二叉堆数据结构,主要步骤:
\begin{itemize}
\item 建立最大堆(父节点值总大于子节点)
\item 将堆顶元素(最大值)与末尾元素交换
\item 调整剩余元素重建最大堆
\end{itemize}
时间复杂度:始终为$O(n \log n)$

\begin{lstlisting}[language=Python]
def heap_sort(arr):
    def sift_down(start, end):
        # 堆调整函数实现...
    
    # 完整实现代码(保持之前的实现)
\end{lstlisting}

\section{面向对象编程}
\subsection{八皇后问题解析}
八皇后问题要求在国际象棋棋盘放置8个皇后,使其互不攻击。解决方案要点:
\begin{itemize}
\item 使用回溯算法逐行放置皇后
\item 通过三个集合记录已被占据的列、左对角线和右对角线
\item 递归搜索所有可能的解空间
\end{itemize}

\begin{lstlisting}[language=Python]
class NQueensSolver:
    def __init__(self, size=8):
        self.size = size
        self.solutions = []
    
    def solve(self):
        def backtrack(row, cols, diag1, diag2):
            if row == self.size:
                self.solutions.append(cols)
                return
            for col in range(self.size):
                if col not in cols and \
                   row+col not in diag1 and \
                   row-col not in diag2:
                    backtrack(row+1, cols+[col], 
                            diag1|{row+col}, diag2|{row-col})
        
        backtrack(0, [], set(), set())
        return self.solutions

# 使用示例
solver = NQueensSolver(8)
print(f"找到 {len(solver.solve())} 种解法")  # 输出: 找到 92 种解法
\end{lstlisting}

\end{document}