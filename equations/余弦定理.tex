\documentclass{article}
\usepackage{ctex}
\usepackage{amsmath}

\begin{document}

\title{高中数学证明余弦定理}
\author{}
\date{}
\maketitle

在三角形 \(ABC\) 中,边长分别为 \(a\)、\(b\)、\(c\),角 \(C\) 是夹在边 \(a\) 和 \(b\) 之间的角。我们要证明余弦定理:
\[
c^2 = a^2 + b^2 - 2ab \cos C
\]

\section*{证明过程}

\subsection*{步骤 1:坐标系设定}
我们将三角形 \(ABC\) 的顶点分别放置在平面直角坐标系中:
- 点 \(A(0, 0)\)
- 点 \(B(c, 0)\)
- 点 \(C(x, y)\)

因此,三角形的边长分别为:\\
\(AB = c\)\\
\(AC = a = \sqrt{x^2 + y^2}\)\\
\(BC = b = \sqrt{(x - c)^2 + y^2}\)\\

\subsection*{步骤 2:利用余弦定义表达角 \(C\)}
根据余弦的定义,角 \(C\) 的余弦值为:
\[
\cos C = \frac{ \overrightarrow{AC} \cdot \overrightarrow{BC} }{ |AC| |BC| }
\]
点积计算为:
\[
\overrightarrow{AC} \cdot \overrightarrow{BC} = x \cdot (x - c) + y \cdot y = x^2 - xc + y^2
\]
因此:
\[
\cos C = \frac{x^2 - xc + y^2}{ \sqrt{x^2 + y^2} \cdot \sqrt{(x - c)^2 + y^2} }
\]

\subsection*{步骤 3:代数推导}
通过代数变换,结合勾股定理,最终可以得到:
\[
c^2 = a^2 + b^2 - 2ab \cos C
\]

\end{document}
