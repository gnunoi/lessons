\coordinate (A) at (0,0); % 定义三角形顶点A
\coordinate (B) at (4,0); % 定义三角形顶点B
\coordinate (C) at (1,3); % 定义三角形顶点C
\draw[thick] (A) node[below left]{$A$} -- 
(B) node[below right]{$B$} -- 
(C) node[above]{$C$} -- cycle; % 绘制三角形ABC
\draw[red, dotted, name path=arcA] 
(A) ++(90:4) arc (90:-270:4);  % % 以A为圆心画弧,++操作保持圆心在A
\draw[red, dotted, name path=arcB] 
(B) ++(90:4) arc (90:-270:4);  % % 以B为圆心画弧,++操作保持圆心在B
\draw[blue, dotted, name path=arcC] 
(C) ++(0:4) arc (0:360:4); %   % 以C为圆心画弧,++操作保持圆心在C
\path[name intersections={of=arcA and arcB, by={D,E}}]; % 求交点
\draw[red, densely dashed, name path=DE] (D) -- (E); % 绘制垂直平分线DE
\path[name intersections={of=arcA and arcC, by={F,G}}]; % 求交点
\draw[blue, densely dashed, name path=FG] (F) -- (G); % 绘制垂直平分线FG
\path[name intersections={of=arcB and arcC, by={M,N}}]; % 求交点
\draw[green!50!black, densely dashed, name path=FG] (M) -- (N); % 绘制垂直平分线MN
\path[name intersections={of=DE and FG, by={O}}]; % 确定外心O
\node[above left] at (O) {$O$};
\node[draw=orange, thick, circle through=(A)] at (O) {};  % 绘制外接圆
\draw[gray] (O) -- (A) (O) -- (B) (O) -- (C);
\foreach \p in {A,B,C,O} 
\node[circle, inner sep=1pt, fill=orange] at (\p) {};
