% 定义三角形顶点
\coordinate (A) at (0,0);
\coordinate (B) at (4,0);
\coordinate (C) at (1,3);
% 绘制三角形ABC
\draw[thick] (A) node[below]{A} -- 
(B) node[below]{B} -- 
(C) node[above]{C} -- cycle;
% ====== AB边垂直平分线 ======
% 以A为圆心画弧
\draw[red, dotted, name path=arcAB1] 
(A) ++(90:3) arc (90:-270:3);  % ++操作保持圆心在A
% 以B为圆心画弧
\draw[red, dotted, name path=arcAB2] 
(B) ++(90:3) arc (90:-270:3);  % ++操作保持圆心在B
% 求交点并绘制垂直平分线DE
\path[name intersections={of=arcAB1 and arcAB2, by={D,E}}];
\draw[red, densely dashed, name path=DE] (D) -- (E);
% ====== AC边垂直平分线 ======
% 计算AC边角度
\pgfmathsetmacro{\angleAC}{atan2(3,1)}  % atan2(y,x)=71.565°
% 以A为圆心画弧
\draw[blue, dotted, name path=arcAC1] 
(A) ++(\angleAC+90:2.5) arc (\angleAC+90:\angleAC-270:2.5);
% 以C为圆心画弧
\draw[blue, dotted, name path=arcAC2] 
(C) ++(\angleAC-90:2.5) arc (\angleAC-90:\angleAC+270:2.5);
% 求交点并绘制垂直平分线FG
\path[name intersections={of=arcAC1 and arcAC2, by={F,G}}];
\draw[blue, densely dashed, name path=FG] (F) -- (G);
% ====== 确定外心O ======
\path[name intersections={of=DE and FG, by={O}}];
\fill[purple] (O) circle (2pt) node[above right] {O};
% ====== 绘制外接圆 ======
\node[draw=green!50!black, thick, 
circle through=(A)] at (O) {};  % 自动通过三点验证
% ====== 几何验证 ======
\draw[gray, opacity=1] (O) -- (A) (O) -- (B) (O) -- (C);
\foreach \p in {A,B,C,O} 
\node[circle, inner sep=2pt, fill=orange] at (\p) {};
