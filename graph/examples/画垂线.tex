\documentclass[tikz,border=5mm]{standalone}
\usetikzlibrary{calc}
\begin{document}
\begin{tikzpicture}
    % 定义线段的两个端点
    \coordinate (A) at (0,0);
    \coordinate (B) at (5,0);
    
    % 定义线外的点
    \coordinate (C) at (3,2);
    
    % 绘制原始线段AB
    \draw (A) node[below]{$A$} -- (B) node[below]{$B$};
    
    % 计算从点C到线段AB的垂足点D
    \coordinate (D) at ($(A)!(C)!(B)$);
    
    % 绘制从点C到垂足点D的垂线
    \draw[dashed,blue] (C) node[right]{$C$} -- (D) node[below]{$D$};
    
    % 标注垂线
    \draw (C) -- (D) node[midway,right] {\textcolor{blue}{垂线}};
    
    % 绘制点
    \foreach \point in {A,B,C,D}
        \fill (\point) circle (1pt);
\end{tikzpicture}
\end{document}