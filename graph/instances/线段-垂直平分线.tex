\documentclass[tikz, border=12pt]{standalone}
\usepackage{tikz}
\usetikzlibrary{intersections}

\begin{document}
\begin{tikzpicture}
\coordinate (A) at (0:0); % 定义点A
\coordinate (B) at (0:5); % 定义点B
% A为圆心,半径5,65~55度角
\draw [name path=arcA1] [thick] (A)+(65:5) arc(65:55:5); 
% B为圆心,半径5,125~115度角
\draw[thick, name path=arcB1] (B)+(125:5) arc(125:115:5); 
% A为圆心,半径5,-65~-55度角
\draw[thick, name path=arcA2] (A)+(-65:5) arc(-65:-55:5);
% B为圆心,半径5,-125~-115度角
\draw [name path=arcB2] [thick] (B)+(-125:5) arc(-125:-115:5); 

\path [name intersections={of=arcA1 and arcB1, by=P}]; % 求交点P
\path [name intersections={of=arcA2 and arcB2, by=Q}]; % 求交点Q
\draw (A) -- (B) (P) -- (Q); % 连接PQ两点
\draw [dotted] (A) -- (P) -- (B) (A) -- (Q) -- (B); % 画点线的辅助线
\node at (A) [left] {$A$}; % 点A左边标注A
\node at (B) [right] {$B$}; % 点B右边标注B
\node at (P) [above] {$P$}; % 点P上方标注P
\node at (Q) [below] {$Q$}; % 点Q下方标注Q
\end{tikzpicture}
\end{document}
