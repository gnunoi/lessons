\documentclass[border=12pt]{standalone}
\usepackage{tikz}
\usetikzlibrary{calc,intersections}
\begin{document}
\begin{tikzpicture}

\coordinate (A) at (0,0); % 设定A点坐标
\coordinate (B) at (5,0); % 设定A点坐标
\coordinate (C) at (3,3); % 设定A点坐标
\coordinate (D) at ($(A)!(C)!(B)$); % $(起点)!(投影点)!(终点)$
\fill (A) circle(1pt); % A点画1pt的实心圆
\fill (B) circle(1pt); % B点画1pt的实心圆
\fill (C) circle(1pt); % C点画1pt的实心圆
\fill (D) circle(1pt); % C点画1pt的实心圆
\node[left] at (A) {$A$}; % 标记A
\node[right] at (B) {$B$}; % 标记B
\node[above] at (C) {$C$}; % 标记C
\node[below] at (D) {$D$}; % 标记C
\draw (A) -- (B); % 画线AB
\draw[red,dashed] (C) -- (D); % 画线CD

\end{tikzpicture}
\end{document}
