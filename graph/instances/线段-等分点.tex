\documentclass[border=12pt]{standalone}
\usepackage{tikz}
\usetikzlibrary{calc,intersections}
\begin{document}
\begin{tikzpicture}

\coordinate (A) at (0,0); % 设定A点坐标
\coordinate (B) at (8,0); % 设定B点坐标
\coordinate (C) at (4,5); % 设定C点坐标
\draw (A) -- (B) -- (C) -- cycle; % 画三角形ABC
\fill (C) circle(1pt) node [above] {\textit{C}}; % 标注C点
% $(起点)!放大系数!(终点)$,放大系数正负数均可
\coordinate (M) at ($(A)!0.5!(B)$); 
\coordinate (M1) at ($(A)!1/3!(B)$); 
\coordinate (M2) at ($(A)!2/3!(B)$); 
\foreach \p in {A,B,M,M1,M2} {
	\fill (\p) circle(1pt) node [below] {\textit{\p}};
	\draw (C) -- (\p);
};



\end{tikzpicture}
\end{document}
