\documentclass[aspectratio=169]{ctexbeamer} %[t]:顶端对齐

\makeatletter
\def\input@path{{../../../styles}}  % 
\makeatother
\usepackage{ubeamer}
\uBigPaper

\date{\today}
\begin{document}

\begin{frame}[t]{等比数列}
题目2:【多选】已知数列$\{a_n\}$是等比数列,公比为$q$,前$n$项和为$S_n$,则下列说法正确的是 \\
A. $\{ \dfrac{1}{a_n} \}$是等比数列    B. $\{  log_{2}a_n \}$是等差数列    C. $\{ a_n+a_{n+1} \}$是等比数列  D. 若$S_n = 3^{n-1} + r$,则$r = -\dfrac{1}{3}$  \\
\vspace{0.5cm}
\pause
解:A. 令$b_n = \dfrac{1}{a_n}$ ,则$\dfrac{b_{n+1}}{b_n} = \dfrac{a_n}{a_{n+1}} = \dfrac{1}{q} \text{(非零常数)} $,所以$\{ \dfrac{1}{a_n} \}$是等比数列,正确\\
\pause
B. 若$a_n < 0$,则$log_2a_n$无意义,错误\\
\pause
C. 当$q = -1$时,$a_n + a_{n+1} = 0$,此时$\{  a_n + a_{n+1} \}$不是等比数列,错误\\
\pause
D. 当$q = 1$时,$S_n = 3^{n-1} + r$的形式不存在,故$q \neq 1$;\\
当$q \neq 1$时,$S_n = A \cdot q^n - A(A = \dfrac{a_1}{q-1})$,\\
由$S_n = 3^{n-1} + r = \dfrac{1}{3} \times 3^n + r = \dfrac{1}{3} \times (3^n - 3r)$得$r = -\dfrac{1}{3}$,正确\\
\pause
正确的选项是AD.
\end{frame}

\end{document}
