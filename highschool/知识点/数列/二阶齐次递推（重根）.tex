\documentclass[aspectratio=169]{ctexbeamer} %[t]:顶端对齐

\makeatletter
\def\input@path{{../../../styles}} 
\makeatother
\usepackage{ubeamer}

\begin{document}

\begin{frame}
\title{二阶齐次递推公式的通项公式推导}
\titlepage
\end{frame}

\begin{frame}{二阶齐次递推公式的通项公式推导}
\frametitle{问题说明}
给定二阶齐次递推公式:
\[
a_{n+1} = 6a_n - 9a_{n-1}
\]
初始条件为:
\[
a_1 = 3, \quad a_2 = 15
\]
求该数列的通项公式。

\end{frame}

\begin{frame}{推导过程}
\frametitle{特征方程}
假设解为等比数列形式$a_n = r^n$,代入递推公式得到:
\[
r^{n+1} = 6r^n - 9r^{n-1}
\]
整理后得到特征方程:
\[
r^2 - 6r + 9 = 0 \quad \Rightarrow (r - 3)^2 = 0
\]
因此,特征方程有一个重根$r = 3$。
\end{frame}

\begin{frame}{通项公式的一般形式}
\frametitle{重根情况下的通项公式}
由于特征方程有重根,通项公式为:
\[
a_n = (C_1 + C_2 n) \cdot 3^n
\]
代入初始条件$a_1 = 3$和$a_2 = 15$:
\[
\begin{cases}
3(C_1 + C_2) = 3 \\
9(C_1 + 2C_2) = 15
\end{cases}
\]
解得:
\[
C_1 = \frac{1}{3}, \quad C_2 = \frac{2}{3}
\]

\end{frame}

\begin{frame}{通项公式的最终形式}
\frametitle{写出通项公式}
代入常数$C_1$和$C_2$得到:
\[
a_n = \left( \frac{1}{3} + \frac{2}{3}n \right) \cdot 3^n
\]
化简后得到:
\[
a_n = (2n + 1) \cdot 3^{n-1}
\]
通过代入初始条件和递推关系验证结果正确。
\end{frame}


\end{document}