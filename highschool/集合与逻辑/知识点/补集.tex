\documentclass[10pt, aspectratio=169]{beamer}
\usetheme{Madrid} % 兼容性最佳的主题
\usecolortheme{seahorse} % 清新配色
\usepackage{ctex} % 中文字体支持
\usepackage{amsmath, amssymb, amsthm} % 数学符号
\usepackage{graphicx} % 图片插入
\usepackage{booktabs} % 表格美化
\usepackage{array} % 表格列对齐

% 标题信息
\title{补集及其表示方法}
\subtitle{高中数学集合论专题}
\author{教师姓名}
\institute{XX学校}
\date{\today}

% 定理环境定制
\theoremstyle{definition}
\newtheorem{definition}{定义}[section]
\theoremstyle{example}
\newtheorem{example}{示例}[section]
\theoremstyle{remark}
\newtheorem{remark}{注意}[section]

\begin{document}

% 标题页
\begin{frame}[plain] % 去除导航栏
  \titlepage
\end{frame}

% 目录页
\begin{frame}{目录}
  \tableofcontents[hideallsubsections] % 隐藏子章节
\end{frame}

% 正文部分
\section{补集基础概念}
\begin{frame}{绝对补集}
  \begin{definition}
    设全集为$ U $,集合$ A \subseteq U $,则$ A $的绝对补集定义为:
    $$
    \complement_U A = \{ x \in U \mid x \notin A \}
    $$
  \end{definition}
  
  \begin{example}
    若全集$ U = \{1,2,3,4,5\} $,集合$ A = \{1,3,5\} $,则:
    $$
    \complement_U A = \{2,4\}
    $$
  \end{example}
\end{frame}

\section{符号与表示法}
\begin{frame}{符号系统}
  \begin{columns}[t]
    \column{0.4\textwidth}
      \begin{itemize}
        \item 绝对补集:$ \complement_U A $、$ C_U A $
        \item 相对补集:$ A - B $、$ A \setminus B $
        \item 其他符号:$ A^c $、$ \overline{A} $
      \end{itemize}
    \column{0.5\textwidth}
      \begin{center}
        \includegraphics[width=0.8\textwidth]{venn_diagram.pdf} % 需替换为实际PDF文件
      \end{center}
  \end{columns}
\end{frame}

\section{运算性质}
\begin{frame}{德摩根律}
  \begin{theorem}[德摩根律]
    $$
    \complement_U (A \cup B) = \complement_U A \cap \complement_U B
    $$
    $$
    \complement_U (A \cap B) = \complement_U A \cup \complement_U B
    $$
  \end{theorem}
  
  \begin{proof}[证明思路]
    \begin{itemize}
      \item 通过维恩图验证区域覆盖关系
      \item 利用集合运算定义进行代数推导
    \end{itemize}
  \end{proof}
\end{frame}

\section{典型例题}
\begin{frame}{参数问题}
  \begin{example}
    已知全集$ U = \mathbb{R} $,集合$ A = \{x \mid x^2 - 4x + 3 \leq 0\} $,求$ m $使$ \complement_U A = (-\infty,1) \cup (3,+\infty) $
  \end{example}
  \vspace{0.5cm}
  \begin{solution}
    \begin{align*}
      A &= [1,3] \\
      \complement_U A &= (-\infty,1) \cup (3,+\infty) \\
      \Rightarrow m &= \text{满足条件的参数值}
    \end{align*}
  \end{solution}
\end{frame}

% 结束页
\begin{frame}[standout]
  \centering
  感谢聆听!
  \vfill
  Q\&A
\end{frame}

\end{document}