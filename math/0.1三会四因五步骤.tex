\documentclass[aspectratio=169]{ctexbeamer} %[t]:顶端对齐

\makeatletter
\def\input@path{{../styles}}  % 
\makeatother
\usepackage{ubeamer}
\uBigPaper

\date{\today}
\begin{document}

\begin{frame}[t]{数学课程的总目标}
\begin{spacing}{1.5} %设置行距
{\large
通过义务教育阶段的数学学习,学生逐步:
\begin{enumerate}[label={\arabic*.}]
\item \textbf{会用数学的眼光观察现实世界;}
\item \textbf{会用数学的思维思考现实世界;}
\item \textbf{会用数学的语言表达现实世界。}
\end{enumerate} 
\vspace{1cm}
\alert{简称“三会”。}
}
\end{spacing}
\end{frame}

\begin{frame}[t]{数学考试丢分的四大原因}
\begin{spacing}{1.5} %设置行距
{\large
\begin{enumerate}[label={\arabic*.}]
\item \textbf{知识不透彻;}
\item \textbf{题型不熟练;}
\item \textbf{计算不准确;}
\item \textbf{计算速度慢。} 
\end{enumerate}
\vspace{1cm}
\alert{简称“四因”。}
}
\end{spacing}
\end{frame}

\begin{frame}[t]{学好数学的五个步骤}
\begin{spacing}{1.5} %设置行距
{\large
\begin{enumerate}[label={\arabic*.}]
\item \textbf{发现个案(发现有趣的个案);}
\item \textbf{类似案例(寻找类似的案例);}
\item \textbf{总结规律(找到一般的规律:从特殊到一般);}
\item \textbf{定义证明(给出定义或证明)。} 
\item \textbf{实际应用(应用到实践中去:从一般到特殊)。} 
\end{enumerate}
\vspace{1cm}
\alert{简称“五步骤”,1-3:大胆假设;4:小心求证;5:放心应用。}
}
\end{spacing}
\end{frame}

\end{document}