\documentclass[aspectratio=169]{ctexbeamer} %[t]:顶端对齐

\makeatletter
\def\input@path{{../styles}}  % 
\makeatother
\usepackage{ubeamer}
\uBigPaper

\date{\today}
\begin{document}

%%%%%%%
\begin{frame}[t]{1.12 有理数的混合运算}
\begin{spacing}{1.2}
\normalsize
\alert{有理数的混合运算:}
\begin{enumerate}[label={\arabic*.}]
\item 加法和减法叫做第一级运算,互为逆运算
\item 乘法和除法叫做第二级运算,互为逆运算
\item 乘方、开方和对数叫做第三级运算,开方是乘方求底数的逆运算,对数是乘方求指数的逆运算
\end{enumerate}

\alert{运算的优先级:}
\begin{enumerate}[label={\arabic*.}]
\item 先做乘方, 再做乘除, 最后做加减;
\item 同级运算, 按照从左至右的顺序进行;
\item 如果有括号, 就先算小括号里的, 再算中括号里的, 然后算大括号里的
\end{enumerate}
\end{spacing}
\end{frame}


\end{document}