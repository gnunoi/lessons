\documentclass[aspectratio=169]{ctexbeamer} %[t]:顶端对齐

\makeatletter
\def\input@path{{../styles}}  % 
\makeatother
\usepackage{ubeamer}
\uBigPaper

\date{\today}
\begin{document}

%%%%%%%
\begin{frame}[t]{1.13 近似数}
\begin{spacing}{1.2}
\normalsize
\alert{近似数:}
\begin{enumerate}[label={\arabic*.}]
\item 与实际值非常接近的数, 称为近似数(approximate number)。例如:\\
$\pi = 3.141\phantom{e}592\cdots$
\item 只取整数,精确到个位数:应用四舍五入法,应为$3$
\item 只取1位小数,精确到十分位(或精确到0.1):应为$3.1$
\item 只取2位小数,精确到百分位(或精确到0.01):应为$3.14$
\item 光在真空中的传播速度:$c=299 \phantom{e} 792 \phantom{e} 458m/s$
\item 用科学记数法,只取整数:$c=3×10^8m/s$
\item 用科学记数法,保留1位小数:$c=3.0×10^8m/s$
\item 用科学记数法,保留5位小数:$c=2.99792×10^8m/s$
\item 注意:四舍五入的位置必须为精确位数的向下一位。
\end{enumerate}

\end{spacing}
\end{frame}


\end{document}