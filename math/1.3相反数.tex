\documentclass[aspectratio=169]{ctexbeamer} %[t]:顶端对齐
\usetheme{Madrid} %Madrid,蓝色调为主。
\usecolortheme{beaver} %beaver
\usefonttheme{professionalfonts}

\makeatletter
\def\input@path{{../styles}}  % 
\makeatother
\usepackage{universe}
\uBigPaper

\date{\today}
\begin{document}

% 1.3 相反数
\begin{frame}{1.3 相反数}
\begin{definition}
\textbf{\textcolor{orange}{只有正负号不同的两个数称互为相反数(opposite number)。\\
我们规定: 0 的相反数是0 .}}
\end{definition}
\begin{columns}
\column{0.5\textwidth}
\begin{itemize}
    \item \textbf{数学表达式}:  $a + b = 0$
    \item \textbf{函数定义}:  $f(x) = -x$
    \item \textbf{定义域}: $x \in \mathbb{R}$
    \item \textbf{值域}: $y \in \mathbb{R}$
    \item \textbf{对称性}: 关于原点中心对称
\end{itemize}

\column{0.5\textwidth}

\begin{figure}
\centering
\scalebox{1.8}{\subfile{fig/相反数}}
\end{figure}

\end{columns}
\end{frame}

% 倒数的定义及函数图象
\begin{frame}{倒数的定义及函数图象}
\begin{definition}
\textbf{\textcolor{orange}{乘积为1的两个数互为倒数。\\
注意:0没有倒数。}}
\end{definition}
\begin{columns}
\column{0.5\textwidth}
\begin{itemize}
    \item \textbf{数学表达式}:  $a \cdot b = 1$
    \item \textbf{函数定义}:  $f(x) = \dfrac{1}{x}$
    \item \textbf{定义域}: $x \in \mathbb{R}, x \neq 0$
    \item \textbf{值域}: $y \in \mathbb{R}, y \neq 0$
    \item \textbf{对称性}: 关于原点中心对称
\end{itemize}

\column{0.5\textwidth}
\begin{figure}
\centering
\scalebox{1.8}{\subfile{fig/倒数}}
\end{figure}
\end{columns}
\end{frame}

% 相反数与倒数的比较
\begin{frame}{相反数与倒数的比较}
\begin{columns}
\column{0.5\textwidth}
\begin{itemize}
    \item \textbf{相反数的表达式}:  $a + b = 0$
    \item \textbf{倒数的表达式}:  $a \cdot b = 1$
    \item \textbf{对称性}: 相反数与倒数均关于原点中心对称
\end{itemize}

\column{0.5\textwidth}
\begin{figure}
\centering
\scalebox{1.8}{\subfile{fig/相反数与倒数}}
\end{figure}
\end{columns}
\end{frame}

\end{document}