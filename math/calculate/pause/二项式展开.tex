\documentclass[aspectratio=169]{ctexbeamer} %[t]:顶端对齐

\makeatletter
\def\input@path{{../../../styles}}  % 
\makeatother
\usepackage{ubeamer}
\uBigPaper

\date{\today}
\begin{document}

%%%%%%%
\begin{frame}[t]{二项式速算法}
\begin{spacing}{1.2}
\normalsize
$(a+b)(a+c)=a^2 + a(b+c) +bc$
\begin{enumerate}[label={\arabic*.}]
\item $26 \times 26  = (25+1)(25+1) = 25^2 + 2 \times 25 + 1^2 = 625 + 50 + 1 = 676$
\item $27 \times 27  = (25+2)(25+2) = 25^2 + 2 \times 25 \times 2 + 2^2 = 625 + 100 + 4 = 729$
\item $36 \times 36  = (35+1)(35+1) = 35^2 + 2 \times 35 + 1^2 = 1225 + 70 + 1 = 1296$
\item $37 \times 37  = (35+2)(35+2) = 35^2 + 2 \times 35 \times 2 + 2^2 = 1225 + 140 + 4 = 1369$
\item $37 \times 38  = (40-3)(40-2) = 40^2 - 40 \times (3 + 2) + 3 \times 2 = 1600 - 200 + 6 = 1406$
\item $27 \times 38  = (30-3)(40-2) = 30 \times 40 - 2 \times 30 - 3 \times 40 + 3^2 = 1200 - 180 + 6 = 1026$
\item $27 \times 48  = (30-3)(50-2) = 30 \times 50 - 30 \times 2 - 3 \times 50 + 3 \times 2 = 1500 - 60 - 150 + 6 = 1296$

\end{enumerate}
\alert{总结利用二项式速算法的数字特点,每人出2道类似的题目} \\
\end{spacing}
\end{frame}


\end{document}