\documentclass[aspectratio=169]{ctexbeamer} %[t]:顶端对齐

\makeatletter
\def\input@path{{../../../styles}}  % 
\makeatother
\usepackage{ubeamer}
\uBigPaper

\date{\today}
\begin{document}

%%%%%%%
\begin{frame}[t]{和十速算法}
\begin{spacing}{1.2}
\normalsize
计算:$\overline{ab} \times \overline{ad}$ \\
当个位数之和等于10,即:$b + d = 10$时,可以使用《和十速算法》,即:\\
设:$e = a+1$,$a(a+1)=\overline{AB}$,$bd = \overline {CD}$则:\\
$\overline{ab} \times \overline{ad} = (10a+b)(10a+d) = 100a^2 + 10a(b+d) + bd = 100a(a+1) + bd = \overline{ABCD}$ \\
例题:
\begin{enumerate}[label={\arabic*.}]
\item $21 \times 29  = 100 \times 2 \times 3 + 1 \times 9 = 600 + 9 = 609$
\item $32 \times 38  = 100 \times 3 \times 4 + 2 \times 8 = 1200 + 16 = 1216$
\item $43 \times 47  = 100 \times 4 \times 5 + 3 \times 7 = 2000 + 21 = 2021$
\item $45 \times 45  = 100 \times 4 \times 5 + 5 \times 5 = 2000 + 25 = 2025$
\item $74 \times 76  = 100 \times 7 \times 8 + 4 \times 6 = 5600 + 24 = 5624$
\end{enumerate}
\alert{思考:和十速算法速利用了乘法的什么规律?数字有什么特点?}\\
\alert{每人出2道类似的题目}  \\

\end{spacing}
\end{frame}

%%%%%%%
\begin{frame}[t]{和十速算法}
\begin{spacing}{1.2}
\normalsize
\begin{enumerate}[label={\arabic*.}]
\item $21 \times 29$ \\

$ \because 2 \times 3 = 6, \quad 1 \times 9 = 9, \quad \therefore 21 \times 29 = \overline{06} \quad \overline{09} = 609$
\item $32 \times 38$ \\

$ \because 3 \times 4 = 12, \quad 2 \times 8 = 16, \quad \therefore 32 \times 38 = \overline{12} \quad \overline{16} = 1216$
\item $43 \times 47 = $ \\

$ \because 4 \times 5 = 20, \quad 3 \times 7 = 21, \quad \therefore 43 \times 47 = \overline{20} \quad \overline{21} = 2021$
\item $45 \times 45$ \\

$ \because 4 \times 5 = 20, \quad 5 \times 5 = 25, \quad \therefore 45 \times 45 = \overline{20} \quad \overline{25} = 2025$
\item $74 \times 76$ \\

$ \because 7 \times 8 = 56, \quad 4 \times 6 = 24, \quad \therefore 74 \times 76 = \overline{56} \quad \overline{24} = 5624$
\end{enumerate}


\end{spacing}
\end{frame}

\end{document}