\documentclass[aspectratio=169]{ctexbeamer} %[t]:顶端对齐

\makeatletter
\def\input@path{{../../styles}}  % 
\makeatother
\usepackage{ubeamer}
\uBigPaper

\date{\today}
\begin{document}

%%%%%%%
\begin{frame}[t]{逢五凑十法}
\begin{spacing}{1.2}
\normalsize
计算:$\overline{ab} \times \overline{cd}$ \\
当$b = 5$,$d$为偶数时,通常可以使用《逢五凑十法》,即:\\
对$\overline{cd}$先除以2,再乘以2,即:$\overline{cd} = \overline{ef} \times 2$,则:\\
$\overline{ab} \times \overline{cd} = \overline{ab} \times 2 \times \overline{ef} $ 
例题:
\begin{enumerate}[label={\arabic*.}]
\item $15 \times 18 = 15 \times 2 \times 9=30 \times 9 = 270$
\item $25 \times 34 = 25 \times 2 \times 17=50 \times 17 = 850$
\item $35 \times 16 = 35 \times 2 \times 8=70 \times 8 = 560$
\item $75 \times 14 = 75 \times 2 \times 7=150 \times 7 = 1050$
\end{enumerate}
\alert{思考:逢五凑十法本质就是乘五除二,利用了乘法的什么规律?} \\
\alert{总结逢五凑十法的数字特点,每人出2道类似的题目} \\
\end{spacing}
\end{frame}

%%%%%%%
\begin{frame}[t]{习题}
\begin{spacing}{1.2}
\normalsize
计算下列各式的值:
\begin{enumerate}[label={\arabic*.}]
\item $23 \times 44 \pause = (25 - 2) \times 44 = 25 \times 44 - 2 \times 44 = 1100 - 88 = 1012$
\item $22 \times 36 \pause = 22 \times (35+1) = 22 \times 35 + 22 \times 1 = 770 + 22 = 792$
\item $22 \times 37 \pause = 22 \times (35+2) = 22 \times 35 + 22 \times 2 = 770 + 44 = 814$
\item $24 \times 28 \pause = (25-1) \times 28 = 25 \times 4 \times 7 - 28 = 700 - 28 = 672$
\end{enumerate}

\end{spacing}
\end{frame}

%%%%%%%
\begin{frame}[t]{大数凑十法}
\begin{spacing}{1.2}
\normalsize
计算:$\overline{ab} \times \overline{cd}$ \\
当尾数(或个位数)$b \ge 6$时,通常可以使用《大数凑十法》,即:\\
设:$e = a+1, f = 10-b$,则:\\
$\overline{ab} \times \overline{cd} = \overline{e0} \times \overline{cd} - \overline{f} \times \overline{cd}$ 
例题:
\begin{enumerate}[label={\arabic*.}]
\item $32 \times 39 = 32 \times (40-1)=32 \times 40 - 32=1280-32=1248$
\item $52 \times 39 = 52 \times (40-1)=52 \times 40 - 52=2080-52=2028$
\item $48 \times 43 = (50-2) \times 43=50 \times 43 - 2 \times 43=2150-86=2064$
\item $37 \times 37 = (40-3)(40-3)=40^2 - 2\times40\times3 +3^2=1600-240+9=1369$

\end{enumerate}
\alert{思考:大数凑十法速利用了乘法的什么规律?数字有什么特点?}\\
\alert{每人出2道类似的题目}  \\
\alert{$32 \times 39 $还可以用什么方法速算?} \\
\alert{$48 \times 43 $还可以用什么方法速算?}
\end{spacing}
\end{frame}

%%%%%%%
\begin{frame}[t]{习题}
\begin{spacing}{1.2}
\normalsize
计算下列各式的值:
\begin{enumerate}[label={\arabic*.}]
\item $15 \times 49 \pause = 15 \times (50-1) = 15 \times 50 - 15 = 750 - 15 = 735$
\item $23 \times 29 \pause = 23 \times (30-1) = 23 \times 30 - 23 = 690 - 23 = 667$
\item $24 \times 28 \pause = 24 \times (30-2) = 24 \times 30 - 24 \times 2 = 720 - 48 = 672$
\item $32 \times 57 \pause = 32 \times (60-3) = 32 \times 60 - 32 \times 3 = 1920 - 96 = 1824$
\end{enumerate}

\end{spacing}
\end{frame}

%%%%%%%
\begin{frame}[t]{双向凑十法}
\begin{spacing}{1.2}
\normalsize
计算:$\overline{a9} \times \overline{c9}$ \\
当两个尾数都等于9时,通常可以使用《双向凑十法》,即:\\
设:$e = a+1, f = d+1$,则:\\
$\overline{a9} \times \overline{c9} = (\overline{e0}-1) \times (\overline{f0} - 1)  = \overline{e0} \times \overline{f0} - \overline{e0} - \overline{f0} + 1$ 
例题:
\begin{enumerate}[label={\arabic*.}]
\item $29 \times 39 = (30-1) \times (40-1) = 30 \times 40 - 30 - 40 + 1=1200-70+1=1131$
\item $19 \times 59 = (20 -1) \times (60-1) = 20 \times 60 - 20 - 60 + 1=1200-80+1=1121$
\item $29 \times 69 = (30 -1) \times (70-1) = 30 \times 70 - 30 - 70 + 1=2100-100+1=2001$
\end{enumerate}
\alert{思考:双向凑十法速利用了乘法的什么规律?}\\
\alert{每人出2道类似的题目} 
\end{spacing}
\end{frame}

\end{document}