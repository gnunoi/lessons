\documentclass[aspectratio=169]{ctexbeamer} %[t]:顶端对齐

\makeatletter
\def\input@path{{../../styles}}  % 
\makeatother
\usepackage{ubeamer}
\uBigPaper

\date{\today}
\begin{document}

%%%%%%%
\begin{frame}[t]{平方差法}
\begin{spacing}{1.2}
\normalsize
平方差公式:$a^2 - b^2 = (a+b)(a-b)$ \\
或:$(a+b)(a-b)=a^2 - b^2$ \\
或:$(a-b)(a+b)=a^2 - b^2$ \\
或:$a^2 = (a-b)(a+b) + b^2$ \\
例题:
\begin{enumerate}[label={\arabic*.}]
\item $28 \times 32 = (30-2)(30+2)=30^2-2^2=900-4=896$
\item $39 \times 41 = (40-1)(40+1)=40^2-1^2=1600-1=1599$
\item $45 \times 55 = (50-5)(50+5)=50^2-5^2=2500-25=2475$
\item $62 \times 78 = (70-8)(70+8)=70^2-8^2=4900-64=4836$
\item $63 \times 87 = (75-12)(75+12)=75^2-12^2=5625-144=5481$
\item $45 \times 45 = 45^2 - 5^2 + 5^2 = (45+5)(45-5) + 25 =50 \times 40 + 25=2000 + 25 = 2025$
\end{enumerate}
\alert{总结利用平方差发的数字特点,每人出2道类似的题目} \\
\pause
平均数为整十或整五,可以方便使用平方差法。
\end{spacing}
\end{frame}

%%%%%%%
\begin{frame}[t]{习题}
\begin{spacing}{1.2}
\normalsize
\begin{enumerate}[label={\arabic*.}]
\item $15 \times 25 = \pause (20-5)(20+5) = 20^2-5^2 = 400-25 = 375$
\item $25 \times 35 = \pause (30-5)(30+5) = 30^2-5^2 = 900-25 = 875$
\item $24 \times 36 = \pause (30-6)(30+6) = 30^2-6^2 = 900-36 = 864$
\item $35 \times 45 = \pause (40-5)(40+5) = 40^2-5^2 = 1600-25 = 1575$
\item $15 \times 35 = \pause (25-10)(25+10) = 25^2-10^2 = 625-100 = 525$
\item $24 \times 46 = \pause (35-11)(35+11) = 35^2-11^2 = 1225-121 = 1104$
\item $75 \times 75 = \pause 70 \times 80 + 5 \times 5 = 5600 + 25 = 5625$
\item $85 \times 85 = \pause 80 \times 90 + 5 \times 5 = 7200 + 25 = 7225$
\item $95 \times 95 = \pause 90 \times 100 + 5 \times 5 = 9000 + 25 = 9025$
\end{enumerate}
\alert{思考:还有没有其它速算方法可以计算上述代数式?}
\end{spacing}
\end{frame}

\end{document}