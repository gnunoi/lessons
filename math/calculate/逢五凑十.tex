\documentclass[aspectratio=169]{ctexbeamer} %[t]:顶端对齐

\makeatletter
\def\input@path{{../../styles}}  % 
\makeatother
\usepackage{ubeamer}
\uBigPaper

\date{\today}
\begin{document}

%%%%%%%
\begin{frame}[t]{逢五凑十法}
\begin{spacing}{1.2}
\normalsize
计算:$\overline{ab} \times \overline{cd}$ \\
当$b = 5$,$d$为偶数时,通常可以使用《逢五凑十法》,即:\\
对$\overline{cd}$先除以2,再乘以2,即:$\overline{cd} = \overline{ef} \times 2$,则:\\
$\overline{ab} \times \overline{cd} = \overline{ab} \times \overline{cd} \div 2 \times 2 = (\overline{ab} \times 2) \times \overline{ef} $ \\
例题:
\begin{enumerate}[label={\arabic*.}]
\item $15 \times 18 \pause = 15 \times 2 \times 9=30 \times 9 = 270$
\item $25 \times 34 \pause = 25 \times 2 \times 17=50 \times 17 = 850$
\item $35 \times 16 \pause = 35 \times 2 \times 8=70 \times 8 = 560$
\item $75 \times 14 \pause = 75 \times 2 \times 7=150 \times 7 = 1050$
\end{enumerate}
\alert{思考:逢五凑十法本质就是乘五等于乘十除二,利用了乘法的什么规律?} \\
\alert{总结逢五凑十法的数字特点,每人出2道类似的题目} \\
\end{spacing}
\end{frame}

%%%%%%%
\begin{frame}[t]{习题}
\begin{spacing}{1.2}
\normalsize
计算下列各式的值:
\begin{enumerate}[label={\arabic*.}]
\item $25 \times 48 \pause = 25 \times 4 \times 12 = 100 \times 12 = 1200$
\item $35 \times 28 \pause = 35 \times 2 \times 14 = 70 \times 14 = 980$
\item $45 \times 16 \pause = 45 \times 2 \times 8 = 90 \times 8 = 720$
\item $55 \times 18 \pause = 55 \times 2 \times 9 = 110 \times 9 = 990$
\item $25 \times 36 \pause = 25 \times 4 \times 9 = 100 \times 9 = 900$
\item $23 \times 44 \pause = (25 - 2) \times 44 = 25 \times 44 - 2 \times 44 = 1100 - 88 = 1012$
\item $22 \times 36 \pause = 22 \times (35+1) = 22 \times 35 + 22 \times 1 = 770 + 22 = 792$
\item $22 \times 37 \pause = 22 \times (35+2) = 22 \times 35 + 22 \times 2 = 770 + 44 = 814$
\item $24 \times 28 \pause = (25-1) \times 28 = 25 \times 4 \times 7 - 28 = 700 - 28 = 672$
\end{enumerate}

\end{spacing}
\end{frame}

\end{document}