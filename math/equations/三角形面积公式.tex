\documentclass[aspectratio=169]{ctexbeamer} %[t]:顶端对齐

\makeatletter
\def\input@path{{../../styles}}  % 
\makeatother
\usepackage{ubeamer}
\uBigPaper
\usetikzlibrary{calc, positioning, shapes.misc}

\begin{document}

%%%%%%%
\begin{frame}[t]{平方差公式的证明}
三角形面积公式是:$S = \frac{1}{2}ah$,如何证明呢?
\vspace{1cm}

证明:\\
$\because S_\text{平行四边形} = a \cdot h$ \\
$\therefore S_\text{三角形} = \dfrac{1}{2}S_\text{平行四边形} = \dfrac{1}{2}a \cdot h$

\begin{tikzpicture}[
      overlay, % 浮于正文上方
      remember picture,  % 允许引用页面坐标系
      shift={(current page.north east)},  % 将坐标系原点移到右上角
      xshift=-15cm,  % 向左移动
      yshift=-8cm,    % 向上移动
      scale=1.5,
      transform shape
    ]
    % 定义三角形的三个顶点
    \coordinate (A) at (0,0); % 底边左端点
    \coordinate (B) at (5,0); % 底边右端点
    \coordinate (C) at (2.5,4); % 顶点
    % 绘制三角形
    \draw (A) -- (B) -- (C) -- cycle;
    % 标注底边为a
    \draw [thick] (A) -- (B) node[midway, below] {$a$};
    % 计算顶点到底边的垂足
    \coordinate (D) at ($(A)!(C)!(B)$); % 垂足点
    % 用虚线绘制高
    \draw[thick, red, dashed] (C) -- (D) node[midway, right] {$h$};
    % 画直角符号
    \draw ($(D) + (0, 0.5)$) -- ($(D) + (0.5, 0.5)$) -- ($(D) + (0.5, 0)$);
    % 再画一个倒三角形,构成平行四边形
    % 使用向量运算计算E点坐标:E = B + C - A
    \coordinate (E) at ($(B) + (C) - (A)$);
    \draw (C) -- (E) -- (B);
    

\end{tikzpicture}

\end{frame}
\end{document}
