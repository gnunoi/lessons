\documentclass[aspectratio=169]{ctexbeamer} %[t]:顶端对齐

\makeatletter
\def\input@path{{../../styles}}  % 
\makeatother
\usepackage{ubeamer}
\uBigPaper

\date{\today}
\begin{document}

%%%%%%%
\begin{frame}[t]{因数分解法}
\begin{spacing}{1.2}
\normalsize
例题:
\begin{enumerate}[label={\arabic*.}]
\item $15 \times 28 \pause = 15 \times 4 \times 7 = 60 \times 7 = 420$
\item $25 \times 36 \pause = 25 \times 4 \times 9 = 100 \times 9 = 900$
\item $33 \times 12 \pause = 3 \times 11 \times 12 = 3 \times 121 = 363$
\item $74 \times 27 \pause = 2 \times 37 \times 27 = 2 \times 999 = 1998$
\item $91 \times 22 \pause = 91 \times 11 \times 2 = 1001 \times 2 = 2002$
\end{enumerate}
\alert{总结利用因数分解法的数字特点,每人出2道类似的题目} \\
\end{spacing}
\end{frame}

%%%%%%%
\begin{frame}[t]{习题}
\begin{spacing}{1.2}
\normalsize
\begin{enumerate}[label={\arabic*.}]
\item $125 \times 72 = \pause 125 \times 8 \times 9 = 1000 \times 9 = 9000$
\item $14 \times 28 = \pause 7 \times 2 \times 7 \times 4 = 49 \times 8 = 400 - 8 = 392$

\end{enumerate}
\alert{思考:逢五凑十法是不是因数分解法的特例?}
\end{spacing}
\end{frame}

\end{document}