\documentclass{beamer}

\makeatletter
\def\input@path{{../styles}}  % 
\makeatother
\usepackage{ubeamer}
\uBigPaper
\uSetMiddleFont

\begin{document}
\begin{frame}[t]{海伦公式的证明}
\begin{columns}
\column{0.5\textwidth}
$a,b,c$为三角形三边长,半周长 $p = \dfrac{a+b+c}{2}$。
\begin{figure}[htbp]
\centering
\begin{tikzpicture}[scale=0.6] % 使用transform shape可以是含文字一起缩小
\tiny
    % 坐标系设定
    \coordinate (A) at (3,4);
    \coordinate (B) at (0,0);
    \coordinate (C) at (5,0);
    \coordinate (D) at ($(B)!(A)!(C)$); % (直线起点)!(投影点)!(直线终点)
    
    % 绘制三角形
    \draw[thick] (B) -- node[below]{$a$} (C)
                 -- node[right]{$b$} (A)
                 -- node[left]{$c$} cycle;
    \node[above] at (A) {$A$};
    \node[left] at (B) {$B$};
    \node[right] at (C) {$C$};
    % 绘制高度线
    \draw[dashed, red] (A) -- node[right]{$h$} (D) ;
    \node[above right] at (D) {$D$};
    
    % 标注线段
    \draw[<->,>=stealth] (0,-0.6) -- node[below]{$x$} (3,-0.6);
    \draw[<->,>=stealth] (3,-0.6) -- node[below]{$a-x$} (5,-0.6);
\end{tikzpicture}
\end{figure}

证明:设三角形ABC中:高AD长度为$h$,线段BD长度为$x$,线段DC长度为$a-x$ ,根据勾股定理得到:
$$
\begin{cases}
c^2 = h^2 + x^2 & \text{$\triangle ABD$} \\
b^2 = h^2 + (a-x)^2 & \text{$\triangle ACD$}
\end{cases}
$$

将两式相减消去$h^2$:
$$c^2 - b^2 = x^2 - (a-x)^2 = 2ax - a^2$$
解得:$$x = \frac{a^2 + c^2 - b^2}{2a}$$
代入第一个方程求高度,得到:
$$h^2 = c^2 - \left( \frac{a^2 + c^2 - b^2}{2a} \right)^2$$
\column{0.5\textwidth}

三角形面积表达式:
$$S = \frac{1}{2}ah \Rightarrow S^2 = \frac{a^2}{4}h^2$$
将$h^2$表达式代入:
\begin{align*}
S^2 &= \frac{1}{4}a^2 \left[ c^2 - \frac{(a^2 + c^2 - b^2)^2}{4a^2} \right ] \\
&= \frac{4a^2c^2 - (a^2+c^2-b^2)^2}{16}\\
&=\frac{(2ac)^2 - (a^2+c^2-b^2)^2}{16} \\
&=\frac{(2ac+a^2+c^2-b^2)(2ac-a^2-c^2+b^2)}{16}\\
&=\frac{(a+b+c)(a+c-b)(a+b-c)(b+c-a)}{16} \\
&=\frac{(a+b+c)}{2} \cdot \frac{(a+c-b)}{2} \cdot \frac{(a+b-c)}{2} \cdot \frac{(b+c-a)}{2} \\
&=p \cdot (p-a) \cdot (p-b) \cdot (p-c)
\end{align*}
$\therefore \quad S=\sqrt{p\cdot(p-a)\cdot(p-b)\cdot(p-c)}$
\end{columns}
\end{frame}
\end{document}