\documentclass[aspectratio=169]{ctexbeamer} %[t]:顶端对齐

\makeatletter
\def\input@path{{../../../styles}} 
\makeatother
\usepackage{ubeamer}

\begin{document}

\begin{frame}
\title{不重根二阶齐次递推公式的通项公式推导}
\titlepage
\end{frame}

\begin{frame}{二阶齐次递推公式的通项公式推导}
\frametitle{问题说明}
给定二阶齐次递推公式:
\[
a_{n+1} = 5a_n - 6a_{n-1}
\]
初始条件为:
\[
a_1 = 4, \quad a_2 = 9
\]
求该数列的通项公式。

\end{frame}

\begin{frame}{推导过程}
\frametitle{特征方程}
假设等比数列形式$a_n = r^n$,代入递推公式得到:
\[
r^{n+1} = 5r^n - 6r^{n-1}
\]
整理后得到特征方程:
\[
r^2 - 6r + 9 = 0 \quad \Rightarrow (r - 3)^2 = 0
\]
因此,特征方程有两个根分别为$r_1 = 2, r_2 = 3$。
\end{frame}

\begin{frame}
当特征方程有两个不同的实根 \(r_1\) 和 \(r_2\) 时,通解为:
\begin{equation*}
a_n = A r_1^n + B r_2^n
\end{equation*}
其中,\(A\) 和 \(B\) 由初始条件确定。
\end{frame}

\begin{frame}
\frametitle{代入初始条件求通项}
已知 \(a_1 = 5\),\(a_2 = 13\):

代入 \(n = 1\):
\begin{equation*}
A \times 2 + B \times 3 = 5
\end{equation*}

代入 \(n = 2\):
\begin{equation*}
A \times 4 + B \times 9 = 13
\end{equation*}

解得:
\begin{align*}
A &= 1 \\
B &= 1
\end{align*}
\end{frame}

\begin{frame}
\frametitle{通项公式}
所以,该数列的通项公式为:
\begin{equation*}
a_n = 2^n + 3^n
\end{equation*}

验证:
\begin{itemize}
\item \(a_1 = 2^1 + 3^1 = 2 + 3 = 5\)
\item \(a_2 = 2^2 + 3^2 = 4 + 9 = 13\)
\end{itemize}
\end{frame}

\end{document}