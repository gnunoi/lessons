\documentclass[aspectratio=169]{ctexbeamer} %[t]:顶端对齐

\makeatletter
\def\input@path{{../../../styles}} 
\makeatother
\usepackage{ubeamer}

\begin{document}

% 封面
\begin{frame}{数列的知识点讲解}
\title{等差数列与等比数列}
\titlepage
\end{frame}

\begin{frame}
\tableofcontents
\end{frame}

\section{数列的概念回顾}
\begin{frame}{数列的概念回顾}
\begin{itemize}
\item \textbf{定义}:按照一定顺序排列的一列数称为数列。数列中的每一个数叫做这个数列的项,排在第一位的数是第一项(或首项),排在第$n$位的数是第$n$项(或通项)。
\item \textbf{表示方法}:一般用$a_1, a_2, a_3, \dots, a_n$表示数列的各项,其中$a_n$表示数列的通项。数列可以看作是定义域为正整数集或其有限子集$\{1, 2, 3, \dots, n\}$的函数$a_n = f(n)$,当自变量依次取$1, 2, 3, \dots, n$时对应的函数值依次排列而成。
\end{itemize}
\end{frame}

\section{等差数列}
\subsection{定义}
\begin{frame}{等差数列的定义}
\begin{itemize}
\item \textbf{定义}:一般地,如果一个数列从第二项起,每一项与它的前一项的差等于同一个常数,那么这个数列就叫做等差数列,这个常数叫做等差数列的公差,公差通常用字母$d$表示。
\item \textbf{符号表示}:对于等差数列$\{a_n\}$,有$a_n - a_{n-1} = d\quad(n \geq 2, n \in \mathbb{N}^*, d\text{为常数})$
\end{itemize}
\end{frame}

\subsection{通项公式推导与证明}
\begin{frame}{等差数列的通项公式推导与证明}
\begin{itemize}
\item \textbf{推导过程}:以首项为$a_1$,公差为$d$的等差数列$\{a_n\}$为例。
\begin{align*}
a_2 - a_1 &= d \Rightarrow a_2 = a_1 + d \\
a_3 - a_2 &= d \Rightarrow a_3 = a_2 + d = (a_1 + d) + d = a_1 + 2d \\
a_4 - a_3 &= d \Rightarrow a_4 = a_3 + d = a_1 + 2d + d = a_1 + 3d \\
&\vdots \\
a_n &= a_1 + (n - 1)d
\end{align*}
\item \textbf{证明}:采用数学归纳法。
\begin{itemize}
\item 当$n = 1$时,$a_1 = a_1 + (1 - 1)d = a_1$,等式成立。
\item 假设当$n = k$($k \geq 1$)时,等式成立,即$a_k = a_1 + (k - 1)d$。
\item 那么,当$n = k + 1$时,$a_{k+1} = a_k + d = a_1 + (k - 1)d + d = a_1 + k d$,等式也成立。
\item 根据数学归纳法原理,等差数列的通项公式$a_n = a_1 + (n - 1)d$对于所有正整数$n$都成立。
\end{itemize}
\end{itemize}
\end{frame}

\subsection{前n项和公式推导与证明}
\begin{frame}{等差数列前n项和公式推导与证明}
\begin{itemize}
\item \textbf{推导过程}:设等差数列$\{a_n\}$的前$n$项和为$S_n = a_1 + a_2 + a_3 + \dots + a_n$。
\begin{itemize}
\item 由于$a_1 + a_n = a_2 + a_{n-1} = a_3 + a_{n-2} = \dots = a_n + a_1$,共有$\frac{n}{2}$对这样的和,所以$S_n = \frac{n}{2} \times (a_1 + a_n)$。
\item 又因为$a_n = a_1 + (n - 1)d$,代入上式得$S_n = na_1 + \dfrac{n(n-1)}{2}d$。
\end{itemize}
\item \textbf{证明}:采用倒序相加法。
\begin{itemize}
\item 将$S_n$和倒序后的$S_n$相加:
\begin{align*}
S_n &= a_1 + a_2 + \dots + a_{n-1} + a_n \\
S_n &= a_n + a_{n-1} + \dots + a_2 + a_1 \\
\text{两式相加得}\quad 2 S_n &= (a_1 + a_n) + (a_2 + a_{n-1}) + \dots + (a_{n-1} + a_2) + (a_n + a_1) \\
&\text{共有$n$个$(a_1 + a_n)$项,} \\
&\text{所以}\quad 2 S_n = n (a_1 + a_n) \Rightarrow S_n = \frac{n}{2} \times (a_1 + a_n)
\end{align*}
\item 再结合$a_n = a_1 + (n - 1)d$,得到$S_n = na_1 + \dfrac{n(n-1)}{2} d$。
\end{itemize}
\end{itemize}
\end{frame}

\subsection{基本推论}
\begin{frame}{等差数列的基本推论}
\begin{itemize}
\item \textbf{推论1(通项公式的变形)}:由$a_n = a_1 + (n - 1)d$,可得$a_n = a_m + (n - m)d\quad(m, n \in \mathbb{N}^* \text{且} m \leq n)$,这体现了等差数列中任意两项之间的关系,知道其中一项和公差以及项数关系,就可以求出另一项。
\item \textbf{推论2(等差中项公式)}:$a_{n-m} + a_{n+m} = 2a_n$ 或$a_n = \dfrac{1}{2}(a_{n-m} + a_{n+m})$,其中:$m, n \in \mathbb{N}^* \text{且} m < n$。等差中项公式是等差数列的一个重要性质。
\item \textbf{推论3(前n项和的性质)}:$S_n, S_{2n} - S_n, S_{3n} - S_{2n}$也成等差数列,公差为$n^2 d$。这个推论可以用来解决涉及等差数列前若干项和的有关问题,通过将前$n$项和看作新的数列,利用等差数列的性质进行求解。
\end{itemize}
\end{frame}

\section{等比数列}
\subsection{定义}
\begin{frame}{等比数列的定义}
\begin{itemize}
\item \textbf{定义}:一般地,如果一个数列从第二项起,每一项与它的前一项的比等于同一个常数,那么这个数列就叫做等比数列,这个常数叫做等比数列的公比,公比通常用字母$q$表示($q \neq 0$)。
\item \textbf{符号表示}:对于等比数列$\{a_n\}$,有$\dfrac{a_n}{a_{n-1}} = q\quad(n \geq 2, n \in \mathbb{N}^*, q \neq 0)$。
\end{itemize}
\end{frame}

\subsection{通项公式推导与证明}
\begin{frame}{等比数列的通项公式推导与证明}
\begin{itemize}
\item \textbf{推导过程}:以首项为$a_1$,公比为$q$的等比数列$\{a_n\}$为例。
\begin{align*}
a_2 &= a_1 q \\
a_3 &= a_2 q = a_1 q^2 \\
a_4 &= a_3 q = a_1 q^3 \\
&\vdots \\
a_n &= a_1 q^{n-1}
\end{align*}
\item \textbf{证明}:采用数学归纳法。
\begin{itemize}
\item 当$n = 1$时,$a_1 = a_1 q^0 = a_1$,等式成立。
\item 假设当$n = k$($k \geq 1$)时,等式成立,即$a_k = a_1 q^{k-1}$。
\item 那么,当$n = k + 1$时,$a_{k+1} = a_k q = a_1 q^{k-1} \times q = a_1 q^k$,等式也成立。
\item 根据数学归纳法原理,等比数列的通项公式$a_n = a_1 q^{n-1}$对于所有正整数$n$都成立。
\end{itemize}
\end{itemize}
\end{frame}

\subsection{前n项和公式推导与证明}
\begin{frame}{等比数列前n项和公式推导与证明}
\begin{itemize}
\item \textbf{推导过程}:设等比数列$\{a_n\}$的前$n$项和为$S_n = a_1 + a_2 + a_3 + \dots + a_n$。
\begin{itemize}
\item 当$q = 1$时,数列的每一项都相等,所以$S_n = n a_1$。
\item 当$q \neq 1$时,将$S_n$乘以$q$得$q S_n = a_1 q + a_2 q + \dots + a_n q = a_2 + a_3 + \dots + a_n + a_n q$。
\item 用原式$S_n$减去这个式子:
\begin{align*}
S_n - q S_n &= a_1 - a_n q \Rightarrow S_n (1 - q) = a_1 - a_n q \\
\text{又因为}\quad a_n &= a_1 q^{n-1} \Rightarrow S_n = \dfrac{a_1 - a_1 q^{n-1} \times q}{1 - q} = \dfrac{a_1 (1 - q^n)}{1 - q}
\end{align*}
\end{itemize}
\item \textbf{证明}:当$q = 1$时,显然成立;当$q \neq 1$时,上述推导过程已经证明了公式的正确性。
\end{itemize}
\end{frame}

\subsection{基本推论}
\begin{frame}{等比数列的基本推论}
\begin{itemize}
\item \textbf{推论1(通项公式的变形)}:由$a_n = a_1 q^{n-1}$,可得$a_n = a_m q^{n-m}\quad(m, n \in \mathbb{N}^* \text{且} m \leq n)$,它反映了等比数列中任意两项之间的关系,类似于等差数列中的相应推论,可用于已知某些项和公比求其他项的情况。
\item \textbf{推论2(等比中项公式)}:$a_{n-m} \cdot a_{n+m} = a_n^2$。这个推论体现了等比数列中相邻三项(以中间项为对称轴的三项)的中间项与前后两项的积的关系,但要注意等比中项可能有两个值(正负),需要根据数列的具体情况进行取舍。
\item \textbf{推论3(前n项和的性质)}:$S_n, S_{2n} - S_n, S_{3n} - S_{2n}$也成等比数列,公比为$q^n$。这个推论与等差数列前若干项和的推论类似,不过公比变成了原来的$n$次方,可用于解决有关等比数列前若干项和的综合问题。
\end{itemize}
\end{frame}

\end{document}