\documentclass[aspectratio=169]{ctexbeamer} %[t]:顶端对齐

\makeatletter
\def\input@path{{../../../styles}}  % 
\makeatother
\usepackage{ubeamer}
\uBigPaper
%\uSetMiddleFont

\date{\today}
\begin{document}

\begin{frame}[t]{数列的综合应用}
题目1:已知数列$\{  a_n \}$前$n$项和为$S_n$,$4S_{n+1} = 3a_{n+1} + 9a_n$,$a_1 = 3$, \\
(1) 证明数列$\{  a_{n+1} - 3a_n \}$为等比数列;\\
(2) 设$b_n = \dfrac{a_n}{(n+1)(n+2)}$,求数列$\{  b_n \}$的前$n$项和$T_n$。\\
\pause
证明:\\
(1) $4a_{n+1} = 4S_{n+1} - 4S_n = 3a_{n+1} + 9a_n - (3a_{n} + 9a_{n-1}) =  3a_{n+1} + 6a_n - 9a_{n-1}$,\\
\pause
所以$a_{n+1} = 6a_n - 9a_{n-1} \quad (n \geq 2)$ \\
\pause
令$c_n = a_{n+1} - 3a_n $,则$C_{n+1} = a_{n+2} - 3a_{n+1} = 6a_{n+1} - 9a_{n} - 3a_{n+1} = 3(a_{n+1} - 3a_{n}) = 3C_n$,\\
所以$\dfrac{c_{n+1}}{c_n} = 3$,因此数列$\{  a_{n+1} - 3a_n \}$是公比为3的等比数列。\\
\pause
由$4S_{n+1} = 3a_{n+1} + 9a_n$得$4S_2 = 4a_2 + 4a_1 = 3a_2 + 9a_1 \quad \Rightarrow a_2 = 5a_1 = 15$ \\
\pause
$c_1 = a_2 - 3a_1 = 15 - 9 = 6$,因此数列$\{  a_{n+1} - 3a_n \}$是首项为6、公比为3的等比数列。
\end{frame}

%\uSetMiddleFont
\begin{frame}[t]{数列的综合应用}
题目1:已知数列$\{  a_n \}$前$n$项和为$S_n$,$4S_{n+1} = 3a_{n+1} + 9a_n$,$a_1 = 3$, \\
(2) 设$b_n = \dfrac{a_n}{(n+1)(n+2)}$,求数列$\{  b_n \}$的前$n$项和$T_n$。\\
\pause
解:\\ 
已经求得$a_n = (2n + 1) \cdot 3^{n-1}$,其特征方程为$r^2 - 6r + 9 = 0 \quad \Rightarrow (r - 3)^2 = 0$,\\
\pause
求得特征方程具有重根$r = 3$,因此数列$\{ a_n \}$可以写成的$a_n = (C_1 + C_2 n)3^n$形式,将$a_1, a_2$的值带入,得到方程组\\
\pause
\[
\begin{cases}
3(C_1 + C_2) = 3 \\
9(C_1 + 2C_2) = 15
\end{cases}
\Rightarrow 
\begin{cases}
C1 = \frac{1}{3}\\
C2 = \frac{2}{3}
\end{cases}
\Rightarrow a_n = (\dfrac{1}{3} + \dfrac{2}{3}n)3^n = (2n+1)3^{n-1}
\]
\pause
$b_n = \dfrac{a_n}{(n+1)(n+2)} = \left (\dfrac{3}{n+2} - \dfrac{1}{n+1} \right ) 3^{n-1} = \dfrac{1}{n+2}3^n - \dfrac{1}{n+1} 3^{n-1}$ \\
\pause
所以$T_n = b_1 + b_2 + \cdots + b_n = \dfrac{1}{3}3^1 - \dfrac{1}{2}3^0 + \dfrac{1}{4}3^2 - \dfrac{1}{3}3^1 + \cdots + \dfrac{1}{n+2}3^n - \dfrac{1}{n+1}3^{n-1} = \dfrac{3^n}{n+2} - \dfrac{1}{2}$
\end{frame}

\end{document}
