\documentclass[aspectratio=169]{ctexbeamer} %[t]:顶端对齐

\makeatletter
\def\input@path{{../../../../styles}}  % 
\makeatother
\usepackage{ubeamer}
%\uBigPaper

\date{\today}
\title{一阶线性递推数列通项公式的推导}

\begin{document}

% 封面
\begin{frame}
    \frametitle{}
    \titlepage
\end{frame}

% 定义
\section{一阶线性递推关系式定义}
\begin{frame}
    \frametitle{一阶线性递推关系式定义}
    \begin{itemize}
        \item 一阶线性递推关系式形如 \(a_{n+1}=pa_{n}+q\)(\(p≠1\),\(p\) 和 \(q\) 是常数)
        \item “一阶”:相邻两项间有关联,\(a_{n+1}\) 和 \(a_{n}\)
        \item “线性”:\(a_{n}\) 的次数为 1
    \end{itemize}
    \alert{如果$p = 1$呢?} \\
    \pause 数列$\{   a_n \}$就是公差为$q$的等差数列
\end{frame}

% 推导步骤 - 构造归零形式
\section{推导通项公式的步骤}
\begin{frame}
    \frametitle{推导通项公式的步骤 - 构造归零形式}
    构造新数列:\(a_{n+1} -k =p(a_{n}-k)\),确定常数 \(k\)

    展开并代入原递推式,比较常数项得 \(k = \dfrac{q}{1 -p}(p≠1)\)
\end{frame}

% 推导步骤 - 构造等比数列
\begin{frame}
    \frametitle{推导通项公式的步骤 - 构造等比数列}
    得到 \(\{a_{n} - \dfrac{q}{1 -p}\}\) 是公比为 \(p\) 的等比数列

    首项为 \(a_{1} - \dfrac{q}{1 -p}\),通项为 \(\left(a_{1} - \dfrac{q}{1 -p}\right)p^{n -1}\)
\end{frame}

% 推导步骤 - 求出原数列通项
\begin{frame}
    \frametitle{推导通项公式的步骤 - 求出原数列通项}
    \(a_{n} =\left(a_{1} - \dfrac{q}{1 -p}\right)p^{n -1} + \dfrac{q}{1 -p}\)

    以 \(a_{n+1}=2a_{n}+1\) 为例,\(p=2\),\(q=1\),\(k=-1\)

    \(\{a_{n}+1\}\) 是公比为 2 的等比数列,首项为 \(a_{1}+1\)

    通项公式为 \(a_{n}=(a_{1}+1)\cdot 2^{n -1} -1\)
\end{frame}

% 举例验证
\section{举例验证}
\begin{frame}
    \frametitle{举例验证}
    已知 \(a_{1}=1\),代入通项公式得 \(a_{n}=2^{n} -1\)

    验证 \(n=1\),\(a_{1}=2^{1} -1=1\),符合已知条件

    验证 \(n=2\),按递推关系式 \(a_{2}=3\),按通项公式 \(a_{2}=3\),正确
\end{frame}

% 总结
\section{总结}
\begin{frame}
    \frametitle{总结}
    通过构造新数列,将一阶线性递推数列转化为等比数列来求通项公式

    掌握这种方法可以解决类似的一阶线性递推数列问题
\end{frame}

\end{document}