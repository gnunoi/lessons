\documentclass[aspectratio=169]{ctexbeamer} %[t]:顶端对齐

\makeatletter
\def\input@path{{../../../styles}}  % 
\makeatother
\usepackage{ubeamer}
\uBigPaper

\date{\today}
\begin{document}

\begin{frame}[t]{等差数列}
题目3:已知数列$\{a_n\}$为等差数列,且满足$a_{2n} = 2a_n + 1(n \in \mathbb{N^*})$。\\
(1) 若$a_1 = 1$,求$\{a_n\}$的前$n$项和$S_n$;\\
\pause
\vspace{1cm}

解:\\
(1) 当$n = 1$时,由$a_{2n} = 2a_n + 1$得$a_2 = 2a_1 + 1 = 3$,所以等差数列$\{a_n\}$的公差为$a_2 - a_1 = 3 - 1 = 2$,\\
\pause
所以等差数列$\{a_n\}$的通项公式为$a_n = a_1 + 2(n-1) = 2n - 1$,\\
\pause
所以前$n$项和为$S_n = \dfrac{n(a_1 + a_n)}{2} = \dfrac{n(1+2n-1)}{2} = n^2$ \\
\pause

\end{frame}

\begin{frame}[t]{等差数列}
题目3:已知数列$\{a_n\}$为等差数列,且满足$a_{2n} = 2a_n + 1(n \in \mathbb{N^*})$。\\
(2) 若数列$\{b_n\}$满足$\dfrac{5}{b_2} - \dfrac{1}{b_1} = \dfrac{3}{4}$,且数列$\{a_n \cdot b_n\}$的前$n$项和$T_n = (3n - 4)2^{n+1} + 8$,求数列$\{b_n\}$的通项公式。\\
\pause
解:(2) 当$n = 1$时,$a_1 \cdot b_1 = T_1 = (3-4) \cdot 2^2 + 8 = -4 + 8 = 4$,所以$b_1 = \dfrac{4}{a_1}$,\\
\pause
$n = 2$时,$a_2 \cdot b_2 = T_2 - T_1 = (6-4) \cdot 2^3 + 8 - 4 = 20$,所以$b_2 = \dfrac{20}{a_2}$,\\
\pause
由$a_{2n} = 2a_n + 1$得$a_2 = 2a_1 + 1$,所以$b_2 = \dfrac{20}{a_2} = \dfrac{20}{2a_1 + 1}$, \\
\pause
由$\dfrac{5}{b_2} - \dfrac{1}{b_1} = \dfrac{3}{4}$得$\dfrac{(2a_1 + 1)}{4} - \dfrac{a_1}{4} = \dfrac{a_1+1}{4} = \dfrac{3}{4}$, \\
\pause
所以$a_1 = 2$,$a_2 = 2a_1 + 1 = 5$,$a_n = 2 + (n-1)(a_2 - a_1) = 2 +3(n-1) = 3n - 1$,$b_1 = 2$,$b_2 = 4$,\\
\pause
$n \ge 2$时,$a_n \cdot b_n = T_n - T_{n-1} = (3n-4)2^{n+1} + 8 - (3n-7)2^n - 8 = (3n-1)2^n$,\\
\pause
所以数列$\{b_n\}$的通项公式为$b_n = 2^n$
\end{frame}

\end{document}
