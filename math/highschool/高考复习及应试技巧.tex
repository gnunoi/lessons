\documentclass[aspectratio=169]{ctexbeamer} %[t]:顶端对齐

\usepackage{textcomp}
\usepackage{booktabs}
\usepackage{longtable}

\makeatletter
\def\input@path{{../../styles}}  % 
\makeatother
\usepackage{ubeamer}
\uBigPaper

\date{\today}
\begin{document}

% 封面
\begin{frame}{在已知中学习,在未知中思考,在实践中成长}
\title{高考复习及应试技巧}
\titlepage
\end{frame}

\begin{frame}[t]{高考复习及应试技巧}
\begin{enumerate}[label={\arabic*.}]
\item 考点分布
\item 专项练习
\item 卷面书写
\item 应试技巧
\item 学习环境
\end{enumerate}
\end{frame}

\begin{frame}[t]{考点分布}
\begin{longtable}{@{} l r r r r r @{}}
\toprule
知识点 & 2021年 & 2022年 & 2023年 & 2024年 & 2025年 \\
\midrule
集合与逻辑 & 5 & 5 & 5 & 5 & 5 \\
函数与导数 & 22 & 27 & 19 & 19 & 22 \\
三角函数 & 15 & 15 & 18 & 11 & 13 \\
数列 & 12 & 10 & 19 & 5 & 10 \\
不等式 & 5 & 5 & 5 & 5 & 5 \\
解析几何 & 22 & 22 & 22 & 32 & 25 \\
立体几何 & 17 & 17 & 18 & 25 & 20 \\
概率与统计 & 18 & 18 & 18 & 20 & 22 \\
复数 & 5 & 5 & 5 & 5 & 5 \\
平面向量 & - & - & - & 5 & 5 \\
算法与框图 & 5 & 5 & - & - & - \\
排列组合 & - & - & - & 5 & 5 \\
计数原理 & - & - & - & 5 & 5 \\
新定义题 & - & - & - & 17 & 17 \\
\bottomrule
\end{longtable}
\end{frame}

\begin{frame}[t]{专项练习}
\begin{enumerate}[label={\arabic*.}]
\item 只有轻松愉悦的心情,学习效果才会更好,效率才会更高;
\item 选择比努力更重要,适合自己的习题(试卷)才是好的习题(试卷);
\item 与老师沟通确定需要进行专项练习的知识点,按照从易到难的顺序排列;
\item 每个知识点优选出每种专项练习的10-30题,容易、中等各占一半。
\end{enumerate}
\alert{选择知识点的顺序如下(需要及时调整):}
\begin{enumerate}[label={\arabic*.}]
\item 数列
\item 集合与逻辑
\item 排列与组合
\item 不等式
\item 概率与统计
\item $\cdots$
\end{enumerate}
\alert{任何练习的思考时间都不要超过5分钟。5分钟以内不会,立即看题解。} \\
\alert{合上题解做一遍。1天后、1周后、4周后复习一遍。} \\

\end{frame}

\begin{frame}[t]{卷面书写}
\begin{enumerate}[label={\arabic*.}]
\item 上标的书写:上提半格,字号变小,文字的中部与正文的顶部对齐
\item 下标的书写:上沉半格,字号变小,文字的中部与正文的底部对齐
\end{enumerate}
\end{frame}

\begin{frame}[t]{应试技巧}
\begin{enumerate}[label={\arabic*.}]
\item 列出题号:在草稿纸上列出所有题号,如:
$1 \qquad 2 \qquad 3 \qquad 4 \qquad 5 \qquad 6 \qquad 7 \qquad 8 \qquad 9 \qquad 10$
$11 \qquad 12 \qquad 13 \qquad 14 \qquad 15 \qquad 16 \qquad 17 \qquad 18 \qquad 19 \qquad 20$
\pause
\item 快速阅卷:花2至5分钟快速阅卷,判断试题考察的知识点
\pause
\item 标记题号:将所有熟悉的试题类型的编号在草稿纸上的题号上画圈,如:
$\textcircled{1} \qquad 2 \qquad \textcircled{3} \qquad 4 \qquad \textcircled{5} \qquad \textcircled{6} \qquad 7 \qquad \textcircled{8} \qquad 9 \qquad 10$
$11 \qquad \textcircled{12} \qquad 13 \qquad \textcircled{14} \qquad 15 \qquad \textcircled{16} \qquad 17 \qquad \textcircled{18} \qquad 19 \qquad 20$
\pause
\item 完成做题:优先完成题号有标记的题目,做完后涂写答题卡(根据个人习惯确定),并在题号上打上对勾
\pause
\item 循环做题:重复上述第2步至第4步
\end{enumerate}
\end{frame}

\begin{frame}[t]{学习环境}
\begin{enumerate}[label={\arabic*.}]
\item 在学习环境中,尽量多的配置一些绿色植物
\item 学习1小时,休息10至15分钟
\end{enumerate}
\end{frame}
\end{document}
