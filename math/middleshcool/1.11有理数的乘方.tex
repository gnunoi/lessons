\documentclass[aspectratio=169]{ctexbeamer} %[t]:顶端对齐

\makeatletter
\def\input@path{{../../styles}}  % 
\makeatother
\usepackage{ubeamer}
\uBigPaper

\date{\today}
\begin{document}

%%%%%%%
\begin{frame}[t]{1.11 有理数的乘方}
\begin{spacing}{1.2}
\normalsize
有理数的乘方:求几个相同乘数的积的运算,叫做乘方(involution)。其中,
\begin{enumerate}[label={\arabic*.}]
\item 乘方的结果叫做幂(power)
\item $a$叫做底数(base number)
\item $n$叫做指数(exponent)
\item $a^n$读作$a$的$n$次方
\item 也可读作$a$的$n$次幂
\end{enumerate}

\end{spacing}
\end{frame}

%%%%%%%
\begin{frame}[t]{1.11 有理数的乘方}
\begin{columns}
\column{0.9\textwidth}
\begin{spacing}{1.2}
\normalsize
乘方的运算法则:
\begin{enumerate}[label={\arabic*.}]
\item 乘方是第三级运算,优先级高于乘除法运算(见P.59)
\item $a^1 = a$
\item 乘方的乘法运算:$ a^m \cdot a^n = a ^{m+n}$
\item 乘方的除法运算:$ a^m ÷ a^n = a ^{m-n} \quad (a \ne 0)$
\item 乘方的乘方运算:$(a^m)^n = a^{mn}$
\item $ a^0 = a^{m-m} = a^m ÷ a^m = 1 \quad (a \ne 0)$
\item $ a^{-1} = a^0 ÷ a^1 = \dfrac{1}{a} \quad (a \ne 0)$
\item 整数的任何次幂都是整数
\item 负数的偶数次幂是正数(偶负得正),负数的奇数次幂是负数(奇负得负)
\item 乘方运算具有右结合的性质:$a^{m^n} = a^{(m^n)}$ 
\end{enumerate}
\end{spacing}
\end{columns}
\end{frame}


%%%%%%%
\begin{frame}[t]{1.11 有理数的乘方}
\begin{columns}
\column{0.9\textwidth}
\begin{spacing}{1.2}
\normalsize
科学记数法:
\begin{enumerate}[label={\arabic*.}]
\item 一个绝对值大于 10 的数可以记成$a×10^n$的形式, 其中$1 ≤| a | < 10$, $n$是正整数. 像这样的记数法叫做科学记数法. 例如:\\
$8 \phantom{e} 000 \phantom{e} 000 = 8×10 ^6$
\item 一个绝对值小于 1 的数可以记成$a×10^{-n}$的形式, 其中$1≤|a| < 10$, $n$是正整数. 像这样的记数法也叫做科学记数法. 例如:\\
$0.000  \phantom{0} 008 = 8×10 ^{-6}$

\end{enumerate}
\end{spacing}
\end{columns}
\end{frame}

\end{document}