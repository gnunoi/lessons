\documentclass[aspectratio=169]{ctexbeamer} %[t]:顶端对齐

\makeatletter
\def\input@path{{../../styles}}  % 
\makeatother
\usepackage{ubeamer}
\uBigPaper

\date{\today}
\begin{document}

% 1.5 有理数的大小比较
\begin{frame}[t]{1.5有理数的大小比较规则}
\large
\begin{enumerate}[label={\arabic*.}]
  \item 数轴上右边的数比左边的数\alert{大}
  \item 正数 \alert{>} 0
  \item 负数 \alert{<} 0
  \item 正数 \alert{>} 负数
  \item 两个负数比较,绝对值大的反而\alert{小} !
  \par 如果$a > b > 0$,则:$-a < -b < 0$
\end{enumerate}

\end{frame}

\begin{frame}[t]{数轴比较法}
\large
\begin{enumerate}[label={\arabic*.}]
  \item 画数轴并标出所有数
  \begin{figure}
  \begin{tikzpicture}[scale=2]
  \fontsize{16}{20}\selectfont
  \draw [black, thick, ->, >=stealth] (-6,0) -- (6,0); 
  \foreach \x in {-5, ..., 5}
  	\draw (\x cm,5pt) -- (\x cm,0pt) node[anchor=north] {$\x$};
    \fill [blue] (-2,0) circle(2pt) node [above=0.3cm] {$a$};
    \fill [blue] (3,0) circle(2pt) node [above=0.3cm] {$b$};    
  \end{tikzpicture}
  \end{figure}
  \item 从左到右(从小到大)排列
  \item 结果:$a \alert{<} b$
\end{enumerate}

\end{frame} 

\end{document}