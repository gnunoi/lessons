\documentclass[aspectratio=169]{ctexbeamer} %[t]:顶端对齐

\makeatletter
\def\input@path{{../../styles}}  % 
\makeatother
\usepackage{ubeamer}
\uBigPaper

\date{\today}
\begin{document}

% 5.2.2 一元一次方程应用题
\begin{frame}[t]{5.2.2 一元一次方程应用题}
试一试:小亮和老师一起整理了一篇教学材料, 准备录入成电子稿. 按篇幅估计, 老师单独录入需4 h 完成, 小亮单独录入需6 h 完成. 小亮先
录入了1h 后, 老师开始一起录入, 问: 还需要多少小时完成? \\
\pause
\vspace{1em} \\
解法一:\\
根据老师所需时间为4h,小亮所需时间为6h,可以假设总工作量为24。设还需要$x$小时可以完成,则:\\
小亮每小时工作量为:$24 \div 6 = 4$,老师每小时工作量为:$24 \div 4 = 6$。\\
由已知条件可得,\\
$4 + 4x + 6x = 24$,解方程得到:\\
$10x = 24 - 4 = 20 \Rightarrow x = 20 \div 10 = 2(h)$ \\
\vspace{1em} \\
答:还需要2小时能够完成。
\end{frame}

% 5.2.2 一元一次方程应用题
\begin{frame}[t]{5.2.2 一元一次方程应用题}
试一试:小亮和老师一起整理了一篇教学材料, 准备录入成电子稿. 按篇幅估计, 老师单独录入需4 h 完成, 小亮单独录入需6 h 完成. 小亮先
录入了1h 后, 老师开始一起录入, 问: 还需要多少小时完成? \\
\pause
解法二:\\
设总工作量为$m$,则小亮每小时工作量为:$\dfrac{1}{6}m$,老师每小时工作量为:$\dfrac{1}{4}m$\\
设还需要$x$小时可以完成,则:\\
$\dfrac{1}{6}m + \dfrac{1}{6}mx + \dfrac{1}{4}mx = m$,\\
$(\dfrac{1}{6} + \dfrac{1}{6}x + \dfrac{1}{4}x) = 1$, \\
$2 + 3x + 2x = 12 \quad \Rightarrow \quad 5x = 10$, \\
$x = 2(h)$ \\
答:还需要2小时能够完成。
\end{frame}

% 5.2.2 一元一次方程应用题
\begin{frame}[t]{5.2.2 一元一次方程应用题}
试一试:小亮和老师一起整理了一篇教学材料, 准备录入成电子稿. 按篇幅估计, 老师单独录入需4 h 完成, 小亮单独录入需6 h 完成. 小亮先
录入了1h 后, 老师开始一起录入, 问: 还需要多少小时完成? \\
\pause
\vspace{12pt} \\
解法三:\\
设总工作量为1,则小亮每小时工作量为:$\dfrac{1}{6}$,老师每小时工作量为:$\dfrac{1}{4}$\\
设还需要$x$小时可以完成,则:\\
\vspace{12pt} 
$\dfrac{1}{6} + \dfrac{1}{6}x + \dfrac{1}{4}x = 1$, \\
\vspace{12pt} 
$2 + 3x + 2x = 12 \quad \Rightarrow \quad 5x = 10$, \\
\vspace{12pt} 
$x = 2(h)$ \\
\vspace{12pt} 
答:还需要2小时能够完成。
\end{frame}

\end{document}