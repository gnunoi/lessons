\documentclass{ctexbeamer}

\makeatletter
\def\input@path{{../../styles}}  % 
\makeatother
\usepackage{ubeamer}
\uBigPaper

\title{一元一次方程}

% 封面
\begin{document}
\begin{frame}
\titlepage
\end{frame}

% 方程的定义
\begin{frame}[t]{方程的定义}
方程的定义:\\
  \alert{含有未知数的等式}叫做\alert{方程}。\\
\vspace{1em} \\
例如:\\
 \begin{center}
 \(2x + 3 = 7\) \\
 \end{center}
   其中 \(x\) 是未知数,这个等式就构成了一个方程。
\end{frame}

% 一元一次方程的定义
\begin{frame}[t]{一元一次方程的定义}
一元一次方程的定义:\\
  在一个方程中,只含有一个\alert{未知数(元)},并且\alert{未知数的指数都是 1(次)},这样的方程叫做一元一次方程。\\
\\
例如:
\begin{center}
\(3x - 5 = 10\) 是一元一次方程
\end{center}
  因为只含有\alert{一个未知数 \(x\)},且\alert{未知数 \(x\) 的次数为 1}。\\
\\
补充说明:\\
  一般用\alert{$x$、$y$、$z$}表示\alert{未知数},用\alert{$a$、$b$、$c$}表示\alert{常数}。
\end{frame}

% 拓展提问
\begin{frame}[t]{拓展提问}
提问:\\
\begin{enumerate}[label={\arabic*.}]
\item 什么是二元一次方程?试举例说明。
\pause
\item 什么是三元一次方程?试举例说明。
\pause
\item 什么是一元二次方程?试举例说明。
\pause
\item 什么是一元三次方程?试举例说明。
\pause
\item 什么是二元二次方程?试举例说明。
\end{enumerate}
\end{frame}

% 等式的基本性质
\begin{frame}[t]{等式的基本性质}
见:《七年级下册》5.2 解一元一次方程 P6。\\
\begin{itemize}
\item \textbf{性质 1:} 等式两边同时加上(或减去)\alert{同一个数或同一个整式},所得结果仍然是等式。
\begin{align*}
\text{如果有} \quad a &= b, \quad \text{则:}\\
a + c &= b + c \\
a - c &= b - c
\end{align*}

\item \textbf{性质 2:} 等式两边同时乘以或除以\alert{同一个数(除数不为 0)},所得结果仍然是等式。
\begin{align*}
\text{如果有} \quad a &= b, \quad \text{则:}\\
ac &= bc \\
\dfrac{a}{c} &= \dfrac{b}{c}  \quad (c \neq 0) \\
\end{align*}
\end{itemize}
思考:为什么性质2只强调同一个数?
\end{frame}

% 方程的变形规则
\begin{frame}[t]{方程的变形规则}
见:《七年级下册》5.2 解一元一次方程 P7。\\
由等式的基本性质,可以得到\alert{方程的变形规则}。
\begin{enumerate}[label={\arabic*.}]
\item 方程两边都加上(或都减去)\alert{同一个数或同一个整式},方程的解不变。
\item 方程两边都乘以(都或除以)\alert{同一个不为 0的数},方程的解不变。
\end{enumerate}
\vspace{1em}
方程变形规则的应用:
\begin{enumerate}[label={\arabic*.}]
\item 利用变形规则1, 将方程中的某些项改变符号后, 从方程的一边移到另一边。像这样的变形叫做\alert{移项(transposition)}。\\
如:$x + 3 = 5$,方程两边都减去$3$,得$x = 5 - 3 \Rightarrow x = 2$
\vspace{1em}

\item 利用变形规则2, 将方程的两边都除以未知数的系数. 像这样的变形通常称作\alert{“将未知数的系数化为1” },简称\alert{“简化系数”}。 \\
如:$3x = 15$,方程的两边都除以$3$,得$x = 15 \div 3 \Rightarrow x = 5$
\end{enumerate}
思考:为什么变形规则2只强调\alert{同一个不为0的数}?
\end{frame}

% 方程的解与解方程
\begin{frame}[t]{方程的解}
\alert{能使方程左、右两边的值相等的未知数的值}, 叫做\alert{方程的解(solution) }.\\
\\
例如$x = 2$ 是方程$x + 3 = 5$的解, 它能使得方程的左、右两边的值相等(都等于5) . \\
当方程中只有一个未知数时, 方程的解也叫做方程的根(root).\\
这些性质是解方程的基础。通过运用这些性质,我们可以对一元一次方程进行变形,从而求出未知数的值,即\alert{解方程}。\\
\\
例如,对于方程
\[2x + 4 = 10\]
我们可以先\alert{移项},即:利用性质 1,两边同时减去 4(也就是将4移项到方程的右边),得到
 \[2x = 6\]
再\alert{简化系数},即:利用性质 2,两边同时除以 2,得到
 \[x = 3\] 
这就是方程的解。
\end{frame}

% 解方程的步骤
\begin{frame}[t]{解方程的步骤}
解方程的步骤一般为:
\begin{enumerate}[label={\arabic*.}]
\item 移项
\item 合并同类项
\item 简化系数
\end{enumerate}
\end{frame}

% 移项及其示例
\begin{frame}[t]{移项的定义与示例}
把方程中的某些项改变符号后,从方程的一边移到另一边,这种变形叫做\alert{移项}。
\begin{enumerate}[label={\arabic*.}]
\item \alert{常数移项}:$x + 3 = 5$ \\
将常数项$3$从方程的左边移项到方程的右边,得:$x = 5 - 3 \Rightarrow x = 2$
\item \alert{未知数移项}:$3x = 2x + 5$ \\
将含有未知数的整式项$2x$从方程的右边移项到方程的左边,得:$3x - 2x = 5 \Rightarrow x = 5$
\item \alert{同时移项}:$3x - 5 = 2x + 10$ 
\begin{itemize}
\item 首先,将含有未知数的整式项移到左边,常数项移到右边。\\
把 \(2x\) 移到左边变为 \(-2x\),把 \(-5\) 移到右边变为 \(+5\)。\\
方程变为: \(3x - 2x = 10 + 5\)。
\item 然后合并同类项,左边变为 \(x\),右边变为 \(15\),得到方程的解:\(x = 15\)。
\end{itemize}
\item \alert{左右交换位置}:$10 = x + 5$ \\
$x + 5 = 10 \Rightarrow x = 10 - 5 \Rightarrow x = 5$
\end{enumerate}

\end{frame}

% 简化系数及其示例
\begin{frame}{简化系数的定义与示例}
将方程两边同时除以未知数的系数,使未知数的系数变为 1,从而得到方程的解。
\begin{enumerate}[label={\arabic*.}]
\item \alert{整数系数的简化} \\
求方程 \(4x = 20\)的解。\\
分析:方程两边同时除以 4,得到方程左边得到$x$,即简化系数为1。\\
解:
\pause
\begin{align*}
x &= \dfrac{20}{4} \\
x &= 5
\end{align*}

\item \alert{分数系数的简化} \\
求方程 \(\dfrac{1}{4}x = 5\)的解。\\
分析:方程两边同时乘以 4,得到方程左边得到$x$,即简化系数为1。\\
解:
\pause
\begin{align*}
x &= 5 \times 4 \\
x &= 20
\end{align*}

\end{enumerate}
\end{frame}

% 行程问题及其示例
\begin{frame}[t]{行程问题及其示例}
涉及路程、速度和时间的关系,通常有相遇问题、追及问题等。\\
甲、乙两人分别从相距 90 千米的 A、B 两地骑行出发,相向而行。甲的速度是12千米/时,乙的速度是18千米/时。甲从A地出发骑行了2.5小时之后,乙从B地骑行出发,乙出发后经过多少小时两人相遇?\\
\\
解:\\
\pause
设经过 \(x\) 小时两人相遇。根据路程 = 速度 × 时间,甲先骑行了$12 \times 2.5 = 30$千米,然后,甲又骑行了 \(12x\) 千米,乙骑行了 \(18x\) 千米。所以:\\
\begin{align*}
 12 \times 2.5 + 12x + 18x = 90 \\
30 + 30x = 90 \\
30x = 90 - 30 = 60 \\
x = 60 \div 30 \\
x = 2
\end{align*}
答:已出发后经过$2$小时,两人相遇。
\end{frame}

% 工程问题及其示例
\begin{frame}[t]{工程问题及其示例}
涉及工作总量、工作效率和工作时间的关系,通常假设工作总量为单位 “1”。\\
小亮和老师一起整理了一篇教学材料, 准备录入成电子稿. 按篇幅估计, 老师单独录入需4 h 完成, 小亮单独录入需6 h 完成. 小亮先
录入了1h 后, 老师开始一起录入, 问: 还需要多少小时完成? \\
\\
解:\\
\pause
设总工作量为1,则小亮每小时工作量为:$\dfrac{1}{6}$,老师每小时工作量为:$\dfrac{1}{4}$\\
设还需要$x$小时可以完成,则:\\
\vspace{12pt} 
$\dfrac{1}{6} + \dfrac{1}{6}x + \dfrac{1}{4}x = 1$, \\
\vspace{12pt} 
$2 + 3x + 2x = 12 \quad \Rightarrow \quad 5x = 10$, \\
\vspace{12pt} 
$x = 2$ \\
\vspace{12pt} 
答:还需要2小时能够完成。
\end{frame}

% 经济问题及其示例
\begin{frame}[t]{经济问题及其示例}
涉及成本、售价、利润、利润率等经济指标之间的关系。\\
学校准备添置一批课桌椅, 原订购60 套, 每套200 元. 店方表示: 如果多购买, 可以优惠. 结果校方购买了72 套, 每套减价6 元, 而商店获得同样多的利润. 求每套课桌椅的成本.\\
\\
解:设每套桌椅的成本为$x$元,则:
\pause
\begin{align*}
60(200 - x) &= 72(200-6-x) \\
5(200 - x) &= 6(200 - 6 - x) \\
1000 - 5x &= 1200 - 36 - 6x \\
x &= 1200 - 1000 - 36 = 200 - 36 = 164
\end{align*}
答:每套桌椅的成本为$164$元。
\end{frame}

% 浓度问题及其示例
\begin{frame}[t]{浓度问题及其示例}
涉及溶液的浓度、溶质质量、溶液质量之间的关系。通常已知不同浓度的溶液混合后的浓度,求某种溶液的质量或浓度等。\\
题型知识说明:
\begin{enumerate}[{label=\arabic*.}]
\item 溶液质量 = 溶质质量 + 溶剂质量
\item 浓度 = 溶质质量 ÷ 溶液质量 × 100\%
\item 溶质质量 = 溶液质量 × 浓度
\item 溶液质量 = 溶质质量 ÷ 浓度
\item 溶剂质量 = 溶液质量 - 溶质质量 = 溶液质量 × (100\% - 浓度)
\end{enumerate}
\end{frame}

% 浓度问题及其示例
\begin{frame}[t]{浓度问题及其示例}
例题:有一个 20 克的盐水溶液,浓度为 15\%。现在向其中加入一定量的水后,溶液的浓度变为 10\%。问加入了多少克水?\\
\\
解:设加入了$x$克水,则:\\
\pause
\begin{align}
(20 + x) \times 10 \% &= 20 \times 15 \% \\
2 + 0.1x &= 3 \\
0.1x &= 3 - 2 = 1 \\
x &= 10
\end{align}
答:加入了$10$克水。
\end{frame}

% 总结
\begin{frame}[t]{总结}
本章我们学习了:
\begin{enumerate}[label={\arabic*.}]
\item 方程的定义、方程的解
\item 一元一次方程的定义
\item 等式的基本性质
\item 方程的变形规则:移项和简化系数
\item 方程的求解方法
\item 一元一次方程在行程、工程、经济、浓度等问题中的应用。
\end{enumerate}
希望大家能够熟练掌握这些知识,并将\alert{一元一次方程及其求解的知识}应用到解决实际生活中的问题。
\end{frame}

\end{document}