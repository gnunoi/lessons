\documentclass[aspectratio=169]{ctexbeamer} %[t]:顶端对齐

\makeatletter
\def\input@path{{../../styles}}  % 
\makeatother
\usepackage{ubeamer}
\uBigPaper

\date{\today}
\mode<presentation>{
    \usetheme{default}
    \usecolortheme{rose}
    \usefonttheme{default}
    \setbeamertemplate{navigation symbols}{}
    \setbeamertemplate{caption}[numbered]
}

\title{函数的核心概念与常见类型}

\begin{document}

\begin{frame}
    \titlepage
\end{frame}

\begin{frame}{目录}
    \tableofcontents
\end{frame}

\section{函数的定义}

\begin{frame}{函数的定义}
    \textbf{函数的定义:数学必修A第一册P62} \\
    一般地,设$A$、$B$是非空的实数集,如果对于集合$A$中的任意一个数$x$,按照某种确定的对应关系$f$,在集合$B$中都有唯一确定的数$y$和它对应,那么就称$f: A→B$为从集合$A$到集合$B$的一个\alert{函数(function)},记作\alert{$y = f(x), x \in A$}。\\
    其中:其中,$x$叫做\alert{自变量},$x$的取值范围$A$叫做函数的\alert{定义域(domain)};与$x$的值相对应的$y$值叫做\alert{函数值},函数值的集合$\{f(x)| x \in A\}$叫做函数的\alert{值域(range)}. \\
    函数的共性:
    \begin{enumerate}[label={\arabic*.}]
    		\item 包含两个非空的数集,用$A$和$B$来表示。
        \item 有一个对应关系。
        \item 对于定义域(集合$A$)中的任意一个数$x$(自变量),值域(集合$B$)中都有唯一确定的数$y$(因变量)与之对应。
    \end{enumerate}

    \vspace{0.5cm}
    \textbf{示例:}
    \begin{itemize}
        \item 考虑函数 \( y = 2x + 1 \),当 \( x = 1 \) 时,\( y = 3 \);\( x = 2 \) 时,\( y = 5 \)。
    \end{itemize}
\end{frame}

% 函数的定义域
\section{函数的定义域}
\begin{frame}{函数的定义域}
    \textbf{定义域:}
    \begin{itemize}
        \item 是函数自变量的取值范围。
        \item 确定定义域时,需考虑使函数运算有意义的条件。
    \end{itemize}

    \vspace{0.5cm}
    \textbf{常见限制条件:}
    \begin{itemize}
        \item 分式中分母不能为零,如函数 \( y = \dfrac{1}{x - 1} \),定义域为 \( x \neq 1 \)。
        \item 偶次根式被开方数非负,如函数 \( y = \sqrt{x} \),定义域为 \( x \geq 0 \)。
    \end{itemize}

    \vspace{0.5cm}
    \textbf{例题:}
    \begin{itemize}
        \item 求函数 \( y = \dfrac{x + 2}{x^2 - 4} \) 和 \( y = \sqrt[3]{2x - 1} \) 的定义域。
    \end{itemize}
\end{frame}

\section{函数的值域}

\begin{frame}{函数的值域}
    \textbf{值域:}
    \begin{itemize}
        \item 是函数因变量的取值范围,取决于定义域和对应法则。
    \end{itemize}

    \vspace{0.5cm}
    \textbf{求值域的方法:}
    \begin{itemize}
        \item \textbf{直接法:} 如函数 \( y = x^2 \),定义域为实数集,值域为 \( y \geq 0 \)。
        \item \textbf{配方法:} 如函数 \( y = x^2 - 4x + 3 \),配成 \( y = (x - 2)^2 - 1 \),值域为 \( y \geq -1 \)。
        \item \textbf{反函数法:} 通过求反函数的定义域得到原函数的值域。
    \end{itemize}

    \vspace{0.5cm}
    \textbf{实例分析:}
    \begin{itemize}
        \item 求函数 \( y = \dfrac{2x + 1}{x - 3} \) 的值域。
    \end{itemize}
\end{frame}

% 区间的定义与应用
\section{区间的概念}
\begin{frame}{区间的概念}
    \textbf{区间的定义:}
    \begin{itemize}
        \item 是数学中表示实数集合的一种方式,用于描述函数的定义域、值域以及一些性质时的连续范围。
    \end{itemize}

    \vspace{0.5cm}
    \textbf{区间的分类:}
    \begin{itemize}
        \item \textbf{闭区间:} 满足不等式$ a \le x \le b$的实数$x$的集合,叫做\alert{闭区间},表示为 [a, b],即包含两端的端点$a$和$b$。
        \item \textbf{开区间:} 满足不等式$ a < x < b$的实数$x$的集合,叫做\alert{开区间},表示为 (a, b),即不包含两端的端点$a$和$b$。
        \item \textbf{半开半闭区间:} 满足不等式$ a \le x < b$或$ a < x \le b$的实数$x$的集合,叫做\alert{半开半闭区间},表示为 [a, b) 或 (a, b],即一端包含端点,另一端不包含。
    \end{itemize}
\end{frame}

% 区间的数轴表示
\begin{frame}{区间与数轴表示}
    \textbf{区间的图形表示:}
    \begin{itemize}
        \item 在数轴上,闭区间用实心点表示端点,开区间用空心点表示端点。
    \end{itemize}
\begin{center}
\begin{tabular}{| c | c |}
\hline
\textbf{区间} & \textbf{数轴表示} \\
\hline
\([a, b]\) & 
\begin{tikzpicture}[baseline={([yshift=0.5ex]current bounding box.center)}, thick, scale=1]
    \draw[-{Stealth}] (-1, 0) -- (6, 0);
    \draw[blue, thick] (1, 0) -- (5, 0);
    \draw[red, fill] (1, 0) circle (2pt) node[below] {\(a\)};
    \draw[red, fill] (5, 0) circle (2pt) node[below] {\(b\)};
\end{tikzpicture} \\
\hline
\((a, b)\) & 
\begin{tikzpicture}[baseline={([yshift=0.5ex]current bounding box.center)}, thick, scale=1]
    \draw[-{Stealth}] (-1, 0) -- (6, 0);
    \draw[blue, thick] (1, 0) -- (5, 0);
    \draw[red, fill=white] (1, 0) circle (2pt) node[below] {\(a\)};
    \draw[red, fill=white] (5, 0) circle (2pt) node[below] {\(b\)};
\end{tikzpicture} \\
\hline
\([a, b)\) & 
\begin{tikzpicture}[baseline={([yshift=0.5ex]current bounding box.center)}, thick, scale=1]
    \draw[-{Stealth}] (-1, 0) -- (6, 0);
    \draw[blue, thick] (1, 0) -- (5, 0);
    \draw[red, fill] (1, 0) circle (2pt) node[below] {\(a\)};
    \draw[red, fill=white] (5, 0) circle (2pt) node[below] {\(b\)};
\end{tikzpicture} \\
\hline
\((a, b]\) & 
\begin{tikzpicture}[baseline={([yshift=0.5ex]current bounding box.center)}, thick, scale=1]
    \draw[-{Stealth}] (-1, 0) -- (6, 0);
    \draw[blue, thick] (1, 0) -- (5, 0);
    \draw[red, fill=white] (1, 0) circle (2pt) node[below] {\(a\)};
    \draw[red, fill] (5, 0) circle (2pt) node[below] {\(b\)};
\end{tikzpicture} \\
\hline
\end{tabular}
\end{center}

\end{frame}

% 区间的定义与应用
\begin{frame}{区间的应用}
    \vspace{0.5cm}
    \textbf{定义域和值域的区间表示:}
    \begin{itemize}
        \item 例如,函数 \( y = \sqrt{x - 1} \) 的定义域为 [1, +∞),值域为 [0, +∞)。
    \end{itemize}

    \vspace{0.5cm}
    \textbf{函数单调性的区间表示:}
    \begin{itemize}
        \item 例如,函数 \( y = x^2 \) 在区间 (-∞, 0] 上单调递减,在区间 [0, +∞) 上单调递增。
    \end{itemize}

    \vspace{0.5cm}
    \textbf{区间运算:}
    \begin{itemize}
        \item 区间的并:例如,(1, 3) ∪ [5, 7] 表示两个不连续的区间。
        \item 区间的交:例如,(2, 6) ∩ [4, 8) = [4, 6)。
        \item 区间的差:例如,[1, 7] \ [3, 5] = [1, 3) ∪ (5, 7]。
    \end{itemize}

    \vspace{0.5cm}
    \textbf{例题:}
    \begin{itemize}
        \item 用区间表示函数 \( y = \dfrac{1}{x} \) 的定义域和值域。
        \item 求函数 \( y = -x^2 + 4 \) 的单调递增区间和单调递减区间。
    \end{itemize}
\end{frame}

% 函数的单调性
\section{函数的单调性}
\begin{frame}{函数的单调性}
    \textbf{单调性:}
    \begin{itemize}
        \item 描述函数在定义域某些区间上因变量的变化趋势。
        \item 分为单调递增和单调递减两种情况。
    \end{itemize}

    \vspace{0.5cm}
    \textbf{单调性定义:}
    \begin{itemize}
        \item \textbf{单调递增:} 在区间内,自变量增大,因变量也随之增大。
        \item \textbf{单调递减:} 在区间内,自变量增大,因变量反而减小。
    \end{itemize}

    \vspace{0.5cm}
    \textbf{图像展示:}
    \begin{itemize}
        \item \( y = x^3 \)(单调递增)
        \item \( y = -x + 2 \)(单调递减)
    \end{itemize}

    \vspace{0.5cm}
    \textbf{判断方法:}
    \begin{itemize}
        \item 通过观察函数图像或利用导数判断单调性。
    \end{itemize}
\end{frame}

\section{函数的奇偶性}

\begin{frame}{函数的奇偶性}
    \textbf{奇偶性定义:}
    \begin{itemize}
        \item \textbf{奇函数:} 对定义域内任意 \( x \),有 \( f(-x) = -f(x) \)。例如,\( f(x) = x^3 \)。
        \item \textbf{偶函数:} 对定义域内任意 \( x \),有 \( f(-x) = f(x) \)。例如,\( f(x) = x^2 \)。
    \end{itemize}

    \vspace{0.5cm}
    \textbf{图像特点:}
    \begin{itemize}
        \item 奇函数图像关于原点对称。
        \item 偶函数图像关于 \( y \) 轴对称。
    \end{itemize}

    \vspace{0.5cm}
    \textbf{判断实例:}
    \begin{itemize}
        \item 判断以下函数的奇偶性:\( f(x) = x^4 - 2x^2 \)、\( f(x) = x^5 + x \)。
    \end{itemize}
\end{frame}

\section{一次函数}

\begin{frame}{一次函数}
    \textbf{一般形式:}
    \begin{itemize}
        \item \( y = kx + b \)(\( k \)、\( b \) 为常数,且 \( k \neq 0 \))
    \end{itemize}

    \vspace{0.5cm}
    \textbf{参数解读:}
    \begin{itemize}
        \item \( k \) 决定图像倾斜程度和单调性:
        \begin{itemize}
            \item \( k > 0 \):函数单调递增。
            \item \( k < 0 \):函数单调递减。
        \end{itemize}
        \item \( b \) 是 \( y \) 轴截距,决定图像与 \( y \) 轴交点位置。
    \end{itemize}

    \vspace{0.5cm}
    \textbf{图像绘制:}
    \begin{itemize}
        \item \( y = x \)、\( y = -x + 2 \)、\( y = 2x + 1 \) 等。
    \end{itemize}

    \vspace{0.5cm}
    \textbf{实际应用:}
    \begin{itemize}
        \item 出租车收费模型:起步价为 \( b \),每公里计费为 \( k \)。
    \end{itemize}
\end{frame}

% 一次函数的直线斜率与截距
\begin{frame}[t]
一次函数的一般形式为:$y = kx + b, $
\end{frame}

% 反比例函数
\section{反比例函数}

\begin{frame}{反比例函数}
    \textbf{反比例函数的定义:}
    \begin{itemize}
        \item 反比例函数是一种形如 \( y = \dfrac{k}{x} \)(\( k \) 为常数且 \( k \neq 0 \))的函数。
        \item 自变量 \( x \) 的取值范围是 \( x \neq 0 \) 的所有实数,因变量 \( y \) 也随之不能为零。
    \end{itemize}

    \vspace{0.5cm}
    \textbf{反比例函数的图像:}
    \begin{itemize}
        \item 图像是双曲线,该双曲线不与坐标轴相交。
        \item 当 \( k > 0 \) 时,双曲线的两支分别位于第一和第三象限。
        \item 当 \( k < 0 \) 时,双曲线的两支分别位于第二和第四象限。
    \end{itemize}

    \vspace{0.5cm}
    \textbf{反比例函数的性质:}
    \begin{itemize}
        \item 函数在其定义域上是单调的。当 \( k > 0 \) 时,在 \( (-\infty, 0) \) 和 \( (0, +\infty) \) 上都是单调递减的;当 \( k < 0 \) 时,在 \( (-\infty, 0) \) 和 \( (0, +\infty) \) 上都是单调递增的。
        \item 反比例函数是奇函数,其图像关于原点对称。
    \end{itemize}

    \vspace{0.5cm}
    \textbf{实际应用:}
    \begin{itemize}
        \item 例如,速度与时间的关系(当路程一定时),电流与电阻的关系(当电压一定时)等。
    \end{itemize}
\end{frame}

% 二次函数
\section{二次函数}

\begin{frame}{二次函数}
    \textbf{一般形式:}
    \begin{itemize}
        \item \( y = ax^2 + bx + c \)(\( a \)、\( b \)、\( c \) 为常数,且 \( a \neq 0 \))
    \end{itemize}

    \vspace{0.5cm}
    \textbf{参数分析:}
    \begin{itemize}
        \item \( a \) 决定抛物线开口方向和大小:
        \begin{itemize}
            \item \( a > 0 \):开口向上。
            \item \( a < 0 \):开口向下。
            \item \( |a| \) 越大,开口越小。
        \end{itemize}
        \item 顶点坐标为 \( \left( -\dfrac{b}{2a}, \dfrac{4ac - b^2}{4a} \right) \)。
        \item 对称轴为直线 \( x = -\dfrac{b}{2a} \)。
    \end{itemize}

    \vspace{0.5cm}
    \textbf{图像绘制与变换:}
    \begin{itemize}
        \item 标准二次函数 \( y = x^2 \) 的图像变换。
    \end{itemize}
\end{frame}

\section{总结与练习}
\begin{frame}{总结与练习}
    \textbf{总结回顾:}
    \begin{itemize}
        \item 函数的定义、定义域、值域、单调性、奇偶性。
        \item 一次函数和二次函数的性质和图像。
    \end{itemize}

    \vspace{0.5cm}
    \textbf{练习题:}
    \begin{itemize}
        \item \textbf{填空题:} 函数 \( y = 3x + 2 \) 的定义域为 \_\_\_\_\_\_,值域为 \_\_\_\_\_\_。
        \item \textbf{选择题:} 函数 \( y = x^3 \) 是(  )
        \begin{itemize}
            \item A. 奇函数  B. 偶函数  C. 非奇非偶函数
        \end{itemize}
        \item \textbf{解答题:} 求二次函数 \( y = -2x^2 + 4x + 1 \) 的顶点坐标和对称轴。
    \end{itemize}
\end{frame}

\begin{frame}{结束语}
    \begin{center}
        \textbf{函数在数学的应用不胜枚举!} \\
    \end{center}
\end{frame}

\end{document}