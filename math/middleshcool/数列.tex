\documentclass{ctexbeamer}

\makeatletter
\def\input@path{{../../styles}}  % 
\makeatother
\usepackage{ubeamer}
\uBigPaper

\title{等差数列与等比数列}
\date{\today}

\begin{document}

\begin{frame}
\titlepage
\end{frame}

\begin{frame}
\frametitle{目录}
\tableofcontents
\end{frame}

\section{数列的概念}
\begin{frame}
\frametitle{数列的概念}
\begin{itemize}
    \item 定义:按照一定顺序排列的一列数称为\alert{数列}。数列中的每一个数叫作这个数列的\alert{项}。
    \item 分类:\alert{有穷数列}(项数有限)、\alert{无穷数列}(项数无限)。
    \item 表示方法:\alert{列举法}、\alert{递推公式法}、\alert{通项公式法}等。
    \item 列举法:$2, 5, 8, 11, 14, \ldots$,这个数列的首项为 \(2\),公差为 \(3\)。通过观察可以发现,每一项都比前一项大 \(3\)。
    \item 递推公式法:$a_n = a_{n-1} + d \quad \text{对于} \quad n \geq 2$
    \item 通项公式法:$a_n = 2 + 3(n - 1) = 3n - 1$
\end{itemize}
\end{frame}

\section{等差数列}
\subsection{定义}
\begin{frame}
\frametitle{等差数列的定义}
\begin{itemize}
    \item 定义:一般地,如果一个数列从第2项起,每一项与它的前一项的差都等于同一个常数,那么这个数列就叫作\alert{等差数列}。这个常数叫作等差数列的\alert{公差},通常用字母\alert{$d$}表示。
    \item 关键词:逐项差、常数。
\end{itemize}
\end{frame}

% 通项公式
\subsection{通项公式}
\begin{frame}[t]
\frametitle{等差数列的通项公式}
\begin{itemize}
    \item 已知条件:等差数列 $\{a_n\}$ 的首项为 $a_1$,公差为 $d$。
    \item 推导过程:
        \begin{align*}
            a_2 &= a_1 + d \\
            a_3 &= a_2 + d = a_1 + 2d \\
            a_4 &= a_3 + d = a_1 + 3d \\
            &\vdots \\
            a_n &= a_{n-1} + d = a_1 + (n - 1)d
        \end{align*}
    \item 等差数列的递推公式为 $a_n = a_{n-1} + d (n \geq 2) $。
    \item 等差数列的通项公式为 $a_n = a_1 + (n - 1)d$。
\end{itemize}
\end{frame}

\begin{frame}[t]
\frametitle{例题}
在数列中的$(\qquad)$填入合适的数字,并求出数列第100项的值。$6, 16, 26, (\qquad), \cdots$ \\
解:\\
$d_1 = a_2 - a_1 = 16 - 6 = 10$ \\
$d_2 = a_3 - a_2 = 26 - 16 = 10$ \\
所以,数列为等差数列,公差为$10$。\\
在$(\qquad)$中的数字是第$4$项,即:$a_4 = a_3 + d = 26 + 10 = 36$。\\
通项公式为:$a_n = a_1 + (n-1)d = 6 + 10(n-1) = 10n - 4$ \\
因此,第$100$项的值为:$a_{100} = 10n - 4 = 10 \times 100  - 4 = 996$。
\end{frame}

% 前n项和公式推导
\subsection{前n项和公式推导}
\begin{frame}[t]
\frametitle{等差数列的前n项和公式推导}
\begin{itemize}
    \item 已知条件:等差数列 $\{a_n\}$ 的首项为 $a_1$,公差为 $d$,前n项和为 $S_n$。
    \item 推导过程:
        \[
            S_n = a_1 + a_2 + a_3 + \ldots + a_n = \frac{n(a_1 + a_n)}{2}
        \]
        或
        \[
            S_n = n \cdot a_1 + \dfrac{n(n - 1)d}{2}
        \]
    \item 等差数列的前n项和公式为 $S_n = \dfrac{n(a_1 + a_n)}{2} = n \cdot a_1 + \dfrac{n(n - 1)d}{2}$。
\end{itemize}
\end{frame}

\subsection{推论及案例}
\begin{frame}[t]
\frametitle{等差数列的推论及案例}
\begin{itemize}
    \item 推论一:在有穷等差数列中,与首末两项等距离的两项之和相等,且等于首项与末项之和。
        \begin{itemize}
            \item 案例:等差数列2,5,8,11,14,首项为2,末项为14,第二项5与倒数第二项11的和为16,第三项8与倒数第三项8的和也为16,且2 + 14 = 16。
        \end{itemize}
    \item 推论二:等差数列的部分和性质:设 $S_n$ 为等差数列 $\{a_n\}$ 的前n项和,则 $S_n, S_{2n} - S_n, S_{3n} - S_{2n}, \ldots$ 构成等差数列。
        \begin{itemize}
            \item 案例:等差数列3,7,11,15,19,…,其前2项和 $S_2 = 3 + 7 = 10$,前4项和 $S_4 = 3 + 7 + 11 + 15 = 36$,则 $S_4 - S_2 = 26$,前6项和 $S_6 = 3 + 7 + 11 + 15 + 19 + 23 = 84$,$S_6 - S_4 = 48$,显然10,26,48,…构成公差为16的等差数列。
        \end{itemize}
    \item 推论三:等差数列的项的和的性质:若 $\{a_n\}$ 是公差为d的等差数列,则数列 $\{a_n + a_{n+1}\}$ 也是等差数列,其公差为2d。
        \begin{itemize}
            \item 案例:等差数列1,4,7,10,13,…,公差 $d = 3$,则数列 $\{a_n + a_{n+1}\}$ 为5,11,17,23,…,公差为6 = 2d。
        \end{itemize}
\end{itemize}
\end{frame}

\section{等比数列}
\subsection{定义}
\begin{frame}[t]
\frametitle{等比数列的定义}
\begin{itemize}
    \item 定义:一般地,如果一个数列从第2项起,每一项与它的前一项的比都等于同一个常数,那么这个数列就叫作\alert{等比数列}。这个常数叫作等比数列的\alert{公比},通常用字母 $q$ 表示($q \neq 0$)。
    \item 关键词:逐项比、常数。
\end{itemize}
\end{frame}

\subsection{通项公式推导}
\begin{frame}[t]
\frametitle{等比数列的通项公式推导}
\begin{itemize}
    \item 已知条件:等比数列 $\{a_n\}$ 的首项为 $a_1$,公比为 $q$($q \neq 0$)。
    \item 推导过程:
        \begin{align*}
            a_2 &= a_1q \\
            a_3 &= a_2q = a_1q^2 \\
            a_4 &= a_3q = a_1q^3 \\
            &\vdots \\
            a_n &= a_1q^{n-1}
        \end{align*}
    \item 等比数列的递推公式为 $a_n = a_{n-1}q$。
    \item 等比数列的通项公式为 $a_n = a_1q^{n-1}$。
\end{itemize}
\end{frame}

\subsection{前n项和公式推导}
\begin{frame}[t]
\frametitle{等比数列的前n项和公式推导}
\begin{itemize}
    \item 已知条件:等比数列 $\{a_n\}$ 的首项为 $a_1$,公比为 $q$($q \neq 0$),前n项和为 $S_n$。
    \item 推导过程:
        \[
            S_n = a_1 + a_1q + a_1q^2 + \ldots + a_1q^{n-1}
        \]
        当 $q = 1$ 时,$S_n = na_1$;
        当 $q \neq 1$ 时,
        \[
            S_n = \frac{a_1(1 - q^n)}{1 - q}
        \]
    \item 公式呈现:等比数列的前n项和公式为 $S_n = \begin{cases}
        na_1, & q = 1 \\
        \frac{a_1(1 - q^n)}{1 - q}, & q \neq 1
    \end{cases}$。
\end{itemize}
\end{frame}

\subsection{推论及案例}
\begin{frame}[t]
\frametitle{等比数列的推论及案例}
\begin{itemize}
    \item 推论一:在有穷等比数列中,与首末两项等距离的两项之积相等,且等于首项与末项之积。
        \begin{itemize}
            \item 案例:等比数列1,2,4,8,16,首项为1,末项为16,第二项2与倒数第二项8的积为16,第三项4与倒数第三项4的积也为16,且1 × 16 = 16。
        \end{itemize}
    \item 推论二:等比数列的部分积性质:设 $\{a_n\}$ 是等比数列,其前n项积为 $T_n$,则 $T_n, \frac{T_{2n}}{T_n}, \frac{T_{3n}}{T_{2n}}, \ldots$ 构成等比数列。
        \begin{itemize}
            \item 案例:等比数列2,4,8,16,32,…,其前2项积 $T_2 = 2 × 4 = 8$,前4项积 $T_4 = 2 × 4 × 8 × 16 = 1024$,则 $\frac{T_4}{T_2} = 128$,前6项积 $T_6 = 2 × 4 × 8 × 16 × 32 × 64 = 2^{6+5+4+3+2+1} = 2^{21} = 2097152$,$\frac{T_6}{T_4} = 2048$,显然8,128,2048,…构成公比为16的等比数列。
        \end{itemize}
    \item 推论三:等比数列的项的和的性质:若 $\{a_n\}$ 是公比为 $q$ 的等比数列,则数列 $\{a_n + a_{n+1}\}$ 也是等比数列,其公比为 $q$。
        \begin{itemize}
            \item 案例:等比数列3,6,12,24,48,…,公比 $q = 2$,则数列 $\{a_n + a_{n+1}\}$ 为9,18,36,72,…,公比为2 = q。
        \end{itemize}
\end{itemize}
\end{frame}

\section{综合应用}
\begin{frame}[t]
\frametitle{通项式与前n项和的综合应用}
\begin{itemize}
    \item 例题1:已知等差数列 $\{a_n\}$ 中,$a_3 = 7$,$a_6 = 16$,求这个数列的通项公式和前10项的和。
        \begin{itemize}
            \item 解答:
                \begin{align*}
                    a_1 + 2d &= 7 \\
                    a_1 + 5d &= 16 \\
                    \text{解得:} a_1 &= 1, d = 3 \\
                    \text{通项公式:} a_n &= 1 + (n - 1) × 3 = 3n - 2 \\
                    \text{前10项和:} S_{10} &= 10 × \frac{2 × 1 + (10 - 1) × 3}{2} = 10 × \frac{2 + 27}{2} = 10 × 29/2 = 145
                \end{align*}
        \end{itemize}
\end{itemize}
\end{frame}

\section{综合应用}
\begin{frame}[t]
\frametitle{通项式与前n项和的综合应用}
\begin{itemize}
    \item 例题2:已知等比数列 $\{a_n\}$ 中,$a_2 = 6$,$a_5 = 48$,求这个数列的通项公式和前6项的和。
        \begin{itemize}
            \item 解答:
                \begin{align*}
                    a_1q &= 6 \\
                    a_1q^4 &= 48 \\
                    \text{解得:} a_1 &= 3, q = 2 \\
                    \text{通项公式:} a_n &= 3 × 2^{n-1} \\
                    \text{前6项和:} S_6 &= 3 × \frac{1 - 2^6}{1 - 2} = 3 × \frac{1 - 64}{-1} = 3 × (-63)/(-1) = 3 × 63 = 189
                \end{align*}
        \end{itemize}
\end{itemize}
\end{frame}

\section{练习与巩固}
\begin{frame}[t]
\frametitle{练习与巩固}
\begin{itemize}
    \item 练习题1:已知等差数列的首项为5,公差为4,求第8项和前8项的和。
    \item 练习题2:已知等比数列的首项为2,公比为3,求第5项和前5项的和。
    \item 练习题3:已知等差数列的前三项分别为2,5,8,求其通项公式及前10项的和。
    \item 练习题4:已知等比数列的前三项分别为3,6,12,求其通项公式及前6项的和。
\end{itemize}
\end{frame}

\section{总结与回顾}
\begin{frame}[t]
\frametitle{总结与回顾}
\begin{itemize}
    \item 等差数列:
        \begin{itemize}
            \item 定义:从第2项起,每一项与前一项的差等于同一个常数。
            \item 通项公式:$a_n = a_1 + (n - 1)d$
            \item 前n项和公式:$S_n = \frac{n(a_1 + a_n)}{2}$ 或 $S_n = na_1 + \dfrac{n(n - 1)d}{2}$
            \item 推论:项的对称性、部分和性质、项的和的性质
        \end{itemize}
    \item 等比数列:
        \begin{itemize}
            \item 定义:从第2项起,每一项与前一项的比等于同一个常数。
            \item 通项公式:$a_n = a_1q^{n-1}$。
            \item 前n项和公式:$S_n = \begin{cases}
                na_1, & q = 1 \\
                \frac{a_1(1 - q^n)}{1 - q}, & q \neq 1
            \end{cases}$
            \item 推论:项的对称性、部分积性质、项的和的性质。
        \end{itemize}
    \item 通项式与前n项和的关系:通项式是求前n项和的基础,前n项和公式的应用需要结合通项式中的首项、公差或公比等参数。
\end{itemize}
\end{frame}

\end{document}