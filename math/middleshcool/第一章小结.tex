\documentclass[aspectratio=169]{ctexbeamer} %[t]:顶端对齐

\makeatletter
\def\input@path{{../../styles}}  % 
\makeatother
\usepackage{ubeamer}
\uBigPaper

\date{\today}
\begin{document}

%%%%%%%
\begin{frame}[t]{第一章 小结}
\begin{spacing}{1.2}
\normalsize
负数的性质:假设$a > b > 0$,
\begin{enumerate}[label={\arabic*.}]
\item 负数小于零:$-a = 0 - a < 0$
\item 负数小于正数:$-a < b < a$
\item 绝对值大的负数反而小:$-a < -b < 0 < b < a$
\item 负数的绝对值等于相反数$ |-a| = a$
\item 相反数之和为零\\
$a + (-a) = 0$ \\
$(-a) + a = 0$ 
\item 相反数之商为-1(除数不为零) \\
$ a ÷ (-a) = -1$ \\
$ (-a) ÷ a = -1$ 
\end{enumerate}
\end{spacing}
\end{frame}

%%%%%%%
\begin{frame}[t]{第一章 小结}
\begin{spacing}{1.2}
\normalsize
有理数的加法法则:假设$a > b > 0$,
\begin{enumerate}[label={\arabic*.}]
\item 同号相加取其号, 绝对值作加法\\
$(+a) + (+b) = +(a+b) = a + b$ \\
$(-a) + (-b) = -(a+b)$ 
\item 异号相加取绝对值大的号,绝对值大减小 \\
$(-a) + (+b) = -(a-b)$ \\
$(+a) + (-b) = +(a-b) = a - b$ 
\item 相反数之和为零 \\
 $a + (-a) = 0$ 
\item 加零和不变 \\
$a + 0 = a $
\end{enumerate}
\end{spacing}
\end{frame}

%%%%%%%
\begin{frame}[t]{第一章 小结}
\begin{spacing}{1.5} %设置行距
\Large
有理数的加法法则:
\begin{enumerate}[label={\arabic*.}]
\item 加法交换律:换位相加和不变 \\
$a + b = b + a$
\item 加法结合律: 先加后加和不变 \\
$(a + b) + c = a + (b + c)$
\end{enumerate}
\end{spacing}
\end{frame}

%%%%%%%
\begin{frame}[t]{第一章 小结}
\begin{spacing}{1.5}
\Large
有理数的减法法则:\\
\begin{enumerate}[label={\arabic*.}]
\item 减去一个数, 等于加其相反数\\
$ a - b = a + (-b)$ \\
$ a - (-b) = a + (+b) = a + b$ \\
\end{enumerate}
\end{spacing}
\end{frame}


%%%%%%%
\begin{frame}[t]{第一章 小结}
\begin{spacing}{1.2}
\normalsize
增减括号法则:
\begin{enumerate}[label={\arabic*.}]
\item 偶负取正号(正号可省略) \\
$+(+a) = +a = a$ \\
$-(-a) = +a = a$ 
\item 奇负取负号 \\
$-(+a) = -a$ \\
$+(-a) = -a$ \\
$-[-(-a)] = -a$
\end{enumerate}

\end{spacing}
\end{frame}

%%%%%%%
\begin{frame}[t]{第一章 小结}
\begin{spacing}{1.2}
\normalsize
有理数的乘法法则:
\begin{enumerate}[label={\arabic*.}]
\item 偶负得正:\\
\vspace{0.5cm}
$(-a)(-b)= ab$ \\ 
\vspace{0.5cm}
$(-a)(-b)(-c)(-d)  = abcd$
\item 奇负得负(绝对值相乘):\\
\vspace{0.5cm}
$(-a)b = a(-b) = -ab$ \\
\vspace{0.5cm}
$(-a)(-b)(-c)  = -abc$
\item 乘零得零:$ a \cdot 0  =  0a = 0$
\item 交换律:$abc = acb = bca = cab$
\item 结合律:$(ab)c = a(bc)$
\item 分配律:$a(b + c) = ab + ac$
\end{enumerate}

\end{spacing}
\end{frame}


%%%%%%%
\begin{frame}[t]{第一章 小结}
\begin{spacing}{1.2}
\normalsize
有理数的除法法则:假设除数不为零,
\begin{enumerate}[label={\arabic*.}]
\item 偶负得正:$(-a) ÷  (-b)  = a  ÷  b = \dfrac{a}{b}$ 
\item 奇负得负(绝对值相除):$(-a)÷b  = a÷(-b) = -a÷b = -\dfrac{a}{b}$ \\
\vspace{0.5cm}
$(-a)÷(-b)÷(-c)  =-a÷(bc) = -\dfrac{a}{bc}$
\item 零除得零(0不能为除数):$0 ÷  a = 0$
\item 除以一数,等于乘其倒数:$a ÷ b = a  \cdot \dfrac{1}{b} = \dfrac{a}{b}$  \\
\vspace{0.5cm}
$a ÷ \dfrac{1}{b} = a  \cdot b = ab$ \\
\vspace{0.5cm}
$a ÷ \dfrac{n}{m} = a  \cdot \dfrac{m}{n} = \dfrac{am}{n}$ \\
\vspace{0.5cm}
$\dfrac{b}{a} ÷ \dfrac{n}{m} = \dfrac{b}{a}  \cdot \dfrac{m}{n} = \dfrac{bm}{an}$
\end{enumerate}

\end{spacing}
\end{frame}

%%%%%%%
\begin{frame}[t]{第一章 小结}
\begin{spacing}{1.2}
\normalsize
有理数的乘方:求几个相同乘数的积的运算,叫做乘方(involution)。其中,
\begin{enumerate}[label={\arabic*.}]
\item 乘方的结果叫做幂(power)
\item $a$叫做底数(base number)
\item $n$叫做指数(exponent)
\item $a^n$读作$a$的$n$次方
\item 也可读作$a$的$n$次幂
\end{enumerate}

\end{spacing}
\end{frame}

%%%%%%%
\begin{frame}[t]{第一章 小结}
\begin{columns}
\column{0.9\textwidth}
\begin{spacing}{1.2}
\normalsize
乘方的运算法则:
\begin{enumerate}[label={\arabic*.}]
\item 乘方是第三级运算,优先级高于乘除法运算(见P.59)
\item $a^1 = a$
\item 乘方的乘法运算:$ a^m \cdot a^n = a ^{m+n}$
\item 乘方的除法运算:$ a^m ÷ a^n = a ^{m-n} \quad (a \ne 0)$
\item 乘方的乘方运算:$(a^m)^n = a^{mn}$
\item $ a^0 = a^{m-m} = a^m ÷ a^m = 1 \quad (a \ne 0)$
\item $ a^{-1} = a^0 ÷ a^1 = \dfrac{1}{a} \quad (a \ne 0)$
\item 整数的任何次幂都是整数
\item 负数的偶数次幂是正数(偶负得正),负数的奇数次幂是负数(奇负得负)
\item 乘方运算具有右结合的性质:$a^{m^n} = a^{(m^n)}$ 
\end{enumerate}
\end{spacing}
\end{columns}
\end{frame}

%%%%%%%
\begin{frame}[t]{第一章 小结}
\begin{spacing}{1.2}
\normalsize
\alert{有理数的混合运算:}
\begin{enumerate}[label={\arabic*.}]
\item 加法和减法叫做第一级运算,互为逆运算
\item 乘法和除法叫做第二级运算,互为逆运算
\item 乘方、开方和对数叫做第三级运算,开方是乘方求底数的逆运算,对数是乘方求指数的逆运算
\end{enumerate}

\alert{运算的优先级:}
\begin{enumerate}[label={\arabic*.}]
\item 先做乘方, 再做乘除, 最后做加减;
\item 同级运算, 按照从左至右的顺序进行;
\item 如果有括号, 就先算小括号里的, 再算中括号里的, 然后算大括号里的
\end{enumerate}
\end{spacing}
\end{frame}

%%%%%%%
\begin{frame}[t]{第一章 小结}
\begin{spacing}{1.2}
\normalsize
\alert{近似数:}
\begin{enumerate}[label={\arabic*.}]
\item 与实际值非常接近的数, 称为近似数(approximate number)。例如:\\
$\pi = 3.141\phantom{e}592\cdots$
\item 只取整数,精确到个位数:应用四舍五入法,应为$3$
\item 只取1位小数,精确到十分位(或精确到0.1):应为$3.1$
\item 只取2位小数,精确到百分位(或精确到0.01):应为$3.14$
\item 光在真空中的传播速度:$c=299 \phantom{e} 792 \phantom{e} 458m/s$
\item 用科学记数法,只取整数:$c=3×10^8m/s$
\item 用科学记数法,保留1位小数:$c=3.0×10^8m/s$
\item 用科学记数法,保留5位小数:$c=2.99792×10^8m/s$
\item 注意:四舍五入的位置必须为精确位数的向下一位。
\end{enumerate}

\end{spacing}
\end{frame}

%%%%%%%
\begin{frame}[t]{第一章 小结}
\begin{columns}
\column{0.9\textwidth}
\begin{spacing}{1.2}
\normalsize
\alert{自然语言与数学语言的比较:}
\begin{enumerate}[label={(\alph*)}]
\item 自然语言:- 3 与 0.3 的和乘以 2 的倒数(P.78)
\item 数学语言:$(-3 + 0.3) × \dfrac{1}{2} = -2.7 × \dfrac{1}{2} = -\dfrac{27}{20} = -1.35$
\item 自然语言:- 3 与 0.3 的和乘以 2\alert{的积}的倒数
\item 数学语言:$\dfrac{1}{(-3 + 0.3) × {2}} = \dfrac{1}{-5.4} = -\dfrac{10}{50} = -\dfrac{5}{27}$

\end{enumerate}

\end{spacing}
\end{columns}
\end{frame}


\end{document}