\documentclass[aspectratio=169]{ctexbeamer} %[t]:顶端对齐

\makeatletter
\def\input@path{{../../styles}}  % 
\makeatother
\usepackage{ubeamer}
\uBigPaper
\uSetBigFont

\date{\today}
\begin{document}

% 圆的周长
\begin{frame}[t]{圆的周长}
圆周率$\pi$取值3.14。
\begin{columns}
\column{0.95\textwidth}
\begin{enumerate}[label={\arabic*.}]
\item 已知圆的直径为d,\alert{圆的周长 = $\pi d$}
\item 已知圆的直径为r,\alert{圆的周长 = $2\pi r$}
\end{enumerate}
练习:
\begin{enumerate}[label={\arabic*.}]
\item 已知圆的直径为2cm,求圆的周长。\\
\pause
解:圆的周长 = $\pi \times 4 = 3.14 \times 4 = 12.56(cm)$
\item 已知圆的直径为4cm,求半圆的周长。\\
\pause
解:半圆的周长 = $\dfrac{1}{2} \times \pi \times 4 + 4= 6.28 + 2 = 8.28(cm)$
\item 已知圆的直径为4cm,求1/4圆的周长。\\
\pause
解:1/4圆的周长 = $\dfrac{1}{4} \times \pi \times 4 + 4= 3.14 + 4 = 7.14(cm)$
\end{enumerate}
\end{columns}
\end{frame}

\begin{tikzpicture}[
      overlay, % 浮于正文上方
      remember picture,  % 允许引用页面坐标系
      shift={(current page.south east)},  % 将坐标系原点移到页面右下角
      xshift=-10cm,  %(避免贴边)
      yshift=10cm,  %(避免贴边)
      scale=1,
      transform shape
    ]
    \coordinate (O) at (0,0);
    \draw (O) circle(4cm);
    
\end{tikzpicture}

\end{document}