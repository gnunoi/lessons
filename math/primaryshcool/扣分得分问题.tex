\documentclass[aspectratio=169]{ctexbeamer} %[t]:顶端对齐

\makeatletter
\def\input@path{{../../styles}}  % 
\makeatother
\usepackage{ubeamer}
\uBigPaper
\uSetBigFont

\date{\today}
\begin{document}

% 得分扣分问题
\begin{frame}[t]{扣分得分问题}
20道竞赛题,每做对一题得5分,每做错一题扣2分,已知得分为72分,问做对了几道题?\\
\pause
\begin{columns}
\column{0.95\textwidth}
解:\\
因为得分为偶数且做错一题扣2分为偶数,做对一题的得分为奇数,因此做对了的题目数量一定是偶数。\\
又因为得分大于70分,做对了的题目数必然大于14。\\
大于14小于等于20的偶数有16、18和20。\\
显然,不可能是20,除非是满分100分。\\
如果做对了16题,则得分为$5 \times 16 - 2 \times 4 = 80 - 8 = 72$,符合题意。\\
如果做对了18题,则得分为$5 \times 18 - 2 \times 2 = 90 - 4 = 86$,不符合题意。\\
因此,做对了\alert{16}道题,做错了\alert{4}道题。注:做错的题目数量不必回答,仅作参考。

\end{columns}
\end{frame}

% 得分扣分问题
\begin{frame}[t]{扣分得分问题}
20道竞赛题,每做对一题得5分,每做错一题扣2分,已知得分为72分,问做对了几道题?\\
\pause
\begin{columns}
\column{0.95\textwidth}
解:\\
设做对了$x$道题,则做错了$20 - x$道题,\\
得分 = $5x - 2(20 - x) = 5x - 40 + 2x = 7x - 40 = 72$,\\
所以$7x = 72 + 40 = 112 \Rightarrow x = 112 \div 7 = 16$ \\
因此,做对了\alert{16}道题。

\end{columns}
\end{frame}
\end{document}