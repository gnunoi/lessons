\documentclass[aspectratio=169]{ctexbeamer} %[t]:顶端对齐

\makeatletter
\def\input@path{{../../styles}}  % 
\makeatother
\usepackage{ubeamer}
\uBigPaper
\uSetBigFont

\date{\today}
\begin{document}

% 比
\begin{frame}[t]{两数的比}
\begin{columns}
\column{0.9\textwidth}
两个数相除,又叫作这两个数的比。如:$6 \div 4 = 6 : 4 = \dfrac{6}{4}= 1.5$
\begin{enumerate}[label={(\arabic*).}]
\item 6是这个比的前项
\item 4是这个比的后项
\item 1.5是6:4的比值
\item “:”叫作比号
\end{enumerate}
\end{columns}
\end{frame}

% 比例
\begin{frame}[t]{比例}
\begin{columns}
\column{0.9\textwidth}
像$12 : 6= 8 : 4$,$6 : 4 = 3 : 2$这样的式子叫作比例。比与比例的关系:
\begin{enumerate}[label={\arabic*.}]
\item  \alert{用于表示两个数或量之间的比较关系,强调的是两个量的相对大小。}
\item 比例是两个比之间的相等关系关系。如:$12 : 6 = 8 : 4$。其中,远离等号的12和4叫作比例的外项,紧挨等号的6和8叫作比例的内项。
\item \alert{比例是由两个比构成的等式。}
\item 因为比可以写成分数的形式,所以比例可以写成由两个分数构成的等式。如:$\dfrac{12}{6} = \dfrac{8}{4}$
\item  \alert{比例的内项乘积等于外项乘积。如:$\dfrac{12}{6} = \dfrac{8}{4} \Rightarrow 12 \times 4 = 8 \times 6$。这一规律也可以称为交叉相乘。}
\item 比侧重于两个数或量之间的直接比较,而不是它们之间的关系是否符合某种特定的数学规律。
\item  \alert{比例强调的是两个比之间的相等性,通常用于解决涉及比例关系的实际问题,如相似三角形、比例尺等。}
\end{enumerate}
\end{columns}
\end{frame}

% 正比例
\begin{frame}[t]{正比例}
\begin{center}
\begin{tabular}{| l | l | l | l | l | l | l |}
    \hline
    时间/h & 1 & 2 & 3 & 4 & 5 & t \\
    \hline
    路程/km & 90 & 180 & 270 & 360 & 450 & s \\
    \hline
\end{tabular} 
\end{center}
像这样,路程与时间两个量,时间变化,路程也随着时间变化,而且路程与时间的比值(也就是速度)一定,我们就说路程和时间成\alert{正比例}。\\
我们将路程记作$s$,速度记作$v$,时间记作$t$,则$s = v \cdot t = vt$
\begin{columns}
\column{0.9\textwidth}

\begin{enumerate}[label={\arabic*.}]
\alert{\item 当速度一定的情况下,路程与时间成正比例(或正比例关系)。}
\alert{\item 当时间一定的情况下,路程与速度成正比例(或正比例关系)。}
\end{enumerate}

\begin{center}
\begin{tabular}{| l | l | l | l | l | l | l | l | l | l | l | l | l | l | l |}
    \hline
    速度/km/h & 10 & 20 & 30 & 40 & 50 & 60 & 70 & 80 & 90 & 100 & 110 & 120 & v \\
    \hline
    路程/km & 20  & 40 & 60 & 80 & 100 & 120 & 140 & 160 & 180 & 200 & 220 & 240 & s \\
    \hline
\end{tabular} 
\end{center}

\end{columns}
\end{frame}

% 反比例
\begin{frame}[t]{反比例}
\begin{center}
\begin{tabular}{| l | l | l | l | l | l | l |}
    \hline
    时间/h & 1 & 2 & 3 & 4 & 5 & t \\
    \hline
    路程/km & 90 & 180 & 270 & 360 & 450 & s \\
    \hline
\end{tabular} 
\end{center}
像这样,路程与时间两个量,时间变化,路程也随着时间变化,而且路程与时间的比值(也就是速度)一定,我们就说路程和时间成\alert{正比例}。\\
我们将路程记作$s$,速度记作$v$,时间记作$t$,则$s = v \cdot t = vt$
\begin{columns}
\column{0.9\textwidth}

\begin{enumerate}[label={\arabic*.}]
\alert{\item 当速度一定的情况下,路程与时间成正比例(或正比例关系)。}
\alert{\item 当时间一定的情况下,路程与速度成正比例(或正比例关系)。}
\end{enumerate}

\begin{center}
\begin{tabular}{| l | l | l | l | l | l | l | l | l | l | l | l | l | l | l |}
    \hline
    速度/km/h & 10 & 20 & 30 & 40 & 50 & 60 & 70 & 80 & 90 & 100 & 110 & 120 & v \\
    \hline
    路程/km & 20  & 40 & 60 & 80 & 100 & 120 & 140 & 160 & 180 & 200 & 220 & 240 & s \\
    \hline
\end{tabular} 
\end{center}

\end{columns}
\end{frame}

\end{document}