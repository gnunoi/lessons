\documentclass[aspectratio=169]{ctexbeamer} %[t]:顶端对齐

\makeatletter
\def\input@path{{../../styles}}  % 
\makeatother
\usepackage{ubeamer}
\uBigPaper
\uSetBigFont

\date{\today}
\begin{document}

\begin{frame}[t]{两数的比}
\begin{columns}
\column{0.9\textwidth}
%\begin{spacing}{1.2}
两个数相除,又叫作这两个数的比。如:$6 \div 4 = 6 : 4 = \dfrac{6}{4}= 1.5$
\begin{enumerate}[label={(\arabic*).}]
\item 6是这个比的前项
\item 4是这个比的后项
\item 1.5是6:4的比值
\item “:”叫作比号
\end{enumerate}
%\end{spacing}
\end{columns}
\end{frame}

\begin{frame}[t]{比例}
\begin{columns}
\column{0.9\textwidth}
%\begin{spacing}{1.2}
像$12 : 6= 8 : 4$,$6 : 4 = 3 : 2$这样的式子叫作比例。比与比例的关系:
\begin{enumerate}[label={(\arabic*).}]
\item  \alert{用于表示两个数或量之间的比较关系,强调的是两个量的相对大小。}
\item 比例是两个比之间的相等关系关系。如:$12 : 6 = 8 : 4$。其中,远离等号的12和4叫作比例的外项,紧挨等号的6和8叫作比例的内项。
\item \alert{比例是由两个比构成的等式。}
\item 因为比可以写成分数的形式,所以比例可以写成由两个分数构成的等式。如:$\dfrac{12}{6} = \dfrac{8}{4}$
\item  \alert{比例的内项乘积等于外项乘积。如:$\dfrac{12}{6} = \dfrac{8}{4} \Rightarrow 12 \times 4 = 8 \times 6$。这一规律也可以称为交叉相乘。}
\item 比侧重于两个数或量之间的直接比较,而不是它们之间的关系是否符合某种特定的数学规律。
\item  \alert{比例强调的是两个比之间的相等性,通常用于解决涉及比例关系的实际问题,如相似三角形、比例尺等。}
\end{enumerate}
%\end{spacing}
\end{columns}
\end{frame}


\end{document}