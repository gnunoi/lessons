\documentclass[aspectratio=169]{ctexbeamer} %[t]:顶端对齐

\makeatletter
\def\input@path{{../../styles}}  % 
\makeatother
\usepackage{ubeamer}
\uBigPaper
\uSetBigFont

\date{\today}
\begin{document}

% 试卷一
% 第2题
\begin{frame}[t]{试卷一}
第2题:$\dfrac{3}{7}$的分数单位是$(\qquad)$,再加上$(\qquad)$个这样的单位就是最小的质数。
\pause
\begin{columns}
\column{0.95\textwidth}
解:\\
第一问是对分数单位知识点的考察,需要非常清晰什么是分数单位。\alert{五年级上册第65页}明确指出:像$\dfrac{1}{2}, \quad \dfrac{1}{3}, \quad \dfrac{1}{4}, \quad \dfrac{1}{5}, \quad \dfrac{1}{6}, \quad \cdots$这样的分数叫作\alert{分数单位}。\\
因此,$\dfrac{3}{7}$的分数单位是$(\quad \alert{\dfrac{1}{7}} \quad )$。\\
\pause
第二问,要了解质数的概念。\alert{五年级上册第38页}明确指出,一个数只有1和它本身两个因数,这个数叫作\alert{质数}。根据\alert{五年级上册第43页}的表格,我们发现最小的质数是\alert{2}。这个问题就变成了$\dfrac{3}{7}$再加上几个$\dfrac{1}{7}$等于2。\\
$\left(2 - \dfrac{3}{7}\right) \div \dfrac{1}{7} = \dfrac{14 -3}{7} \div \dfrac{1}{7} = \dfrac{11}{7} \div \dfrac{1}{7}= 11$,因此第二问的答案就是\alert{11}。
\end{columns}
\end{frame}


% 第7题
\begin{frame}[t]{试卷一}
第7题:生产一批零件,甲乙合作10天可以完成。若甲单独做,18天可以完成;那么,乙单独做要$(\qquad)$天能够完成。
\pause
\begin{columns}
\column{0.95\textwidth}
\pause
解:\\
设乙单独做需要$x$天能够完成,则: \\
$\dfrac{1}{10} = \dfrac{1}{18} + \dfrac{1}{x}$ \\
所以,$ \dfrac{1}{x} = \dfrac{1}{10} - \dfrac{1}{18} = \dfrac{9-5}{90} = \dfrac{4}{90}$ \\
所以,$x = 90 \div 4 = 22.5$ \\
答案:\alert{22.5}。
\end{columns}
\end{frame}

% 第8题
\begin{frame}[t]{试卷一}
第8题:若$a - b = 1$($a, b$是不为$0$的自然数),则$a, b$的最大公因数是$(\qquad)$,最小公倍数是$(\qquad)$。
\pause
\begin{columns}
\column{0.95\textwidth}
解:\\
$a - b = 1 \Rightarrow a = b + 1$,由此可以得出$a, b$必为一个奇数、一个偶数,最大公因数为1。如果不明白可以试着举几个例子去理解,如:$b = 1, a = 2$或$b = 2, a = 3$,最大公因数均为\alert{1}。\\
\pause
最小公倍数 = 两数的乘积 ÷ 最大公因数,因此$a, b$的最小公倍数为\alert{$a \cdot b$}
\end{columns}
\end{frame}

% 第9题
\begin{frame}[t]{试卷一}
第9题:一个盒子里有5个红球、1个绿球和2个黄球,每次摸出一个球后再放回盒中,这样摸600次,摸到绿球的次数约占总次数的$(\qquad)$,大约一共能摸到$(\qquad)$次黄球。
\pause
\begin{columns}
\column{0.95\textwidth}
\pause
解:\\
一共有$5 + 1 + 2 = 8$个球,其中有1个绿球、2个黄球。\\
所以,摸到绿球的概率 = $1 \div 8 \times 100\% = \alert{12.5\%}$;\\
\pause
摸到黄球的概率 = $2 \div 8 \times 100\% = \alert{25\%}$;\\
摸600次,大约能摸到黄球的次数  = $600 \times 25\% = 600 \div 4 = \alert{150}$ \\
答案:\alert{12.5\%}、\alert{150}。
\end{columns}
\end{frame}

% 第12题
\begin{frame}[t]{试卷一}
第12题:一个长方体的长、宽、高分别为$a$米、$b$米、$c$米,如果把它的高增加增加3米后新长方体的体积比原来增加$(\qquad)$立方米。
\pause
\begin{columns}
\column{0.95\textwidth}
\pause
解:\\
原长方体的体积为:$a \times b \times c$(立方米) \\
新长方体的体积为:$a \times b \times (c+3)$(立方米) \\
增加的体积为:$a \times b \times (c+3) - a \times b \times c = 3ab$(立方米) \\
答案:\alert{3ab}。
\end{columns}
\end{frame}

\end{document}