\documentclass[aspectratio=169]{ctexbeamer} %[t]:顶端对齐

\makeatletter
\def\input@path{{../../styles}}  % 
\makeatother
\usepackage{ubeamer}
\uBigPaper
\uSetBigFont

\date{\today}
\begin{document}

% 试卷一
% 第2题
\begin{frame}[t]{试卷一}
第2题:$\dfrac{3}{7}$的分数单位是$(\qquad)$,再加上$(\qquad)$个这样的单位就是最小的质数。
\pause
\begin{columns}
\column{0.95\textwidth}
解:\\
第一问是对分数单位知识点的考察,需要非常清晰什么是分数单位。\alert{五年级上册第65页}明确指出:像$\dfrac{1}{2}, \quad \dfrac{1}{3}, \quad \dfrac{1}{4}, \quad \dfrac{1}{5}, \quad \dfrac{1}{6}, \quad \cdots$这样的分数叫作\alert{分数单位}。\\
因此,$\dfrac{3}{7}$的分数单位是$(\quad \alert{\dfrac{1}{7}} \quad )$。\\
\pause
第二问,要了解质数的概念。\alert{五年级上册第38页}明确指出,一个数只有1和它本身两个因数,这个数叫作\alert{质数}。根据\alert{五年级上册第43页}的表格,我们发现最小的质数是\alert{2}。这个问题就变成了$\dfrac{3}{7}$再加上几个$\dfrac{1}{7}$等于2。\\
$\left(2 - \dfrac{3}{7}\right) \div \dfrac{1}{7} = \dfrac{14 -3}{7} \div \dfrac{1}{7} = \dfrac{11}{7} \div \dfrac{1}{7}= 11$,因此第二问的答案就是\alert{11}。
\end{columns}
\end{frame}

% 第6题
\begin{frame}[t]{试卷一}
第6题:用4个同样的正方体拼成一个长方体(如图),表面积减少32平方厘米,每个小正方体的体积是$(\qquad)$立方厘米。
\vspace{1em}
\pause
\begin{columns}
\column{0.5\textwidth}
解:\\
由图可知,前后重叠的面有$4$个(见蓝色边框所示),左右重叠的面有$4$个(见红色边框所示),共有$8$个面重合,面积减少$32$平方厘米,则每个小正方形的面积为$32 \div 8 = 4$平方厘米,\\
所以,边长为2厘米,因此每个小正方体的体积是$2 \times 2 \times 2 = 8$立方厘米。\\
答案:\alert{8}
\column{0.4\textwidth}
\end{columns}

\begin{tikzpicture}[
      overlay, % 浮于正文上方
      remember picture,  % 允许引用页面坐标系
      shift={(current page.north east)},  % 将坐标系原点移到页面右下角
      xshift=-8cm,  % 向左移动(避免贴边)
      yshift=-8cm,   % 向上移动(避免贴边)
      scale=3,
%      transform shape
    ]
\coordinate (A1) at (0,0,0);
\coordinate (A2) at (1,0,0);
\coordinate (A3) at (1,1,0);
\coordinate (A4) at (0,1,0);
\coordinate (B1) at (0,0,1);
\coordinate (B2) at (1,0,1);
\coordinate (B3) at (1,1,1);
\coordinate (B4) at (0,1,1);

% 定义右边正方体的顶点
\coordinate (C1) at (1,0,0);
\coordinate (C2) at (2,0,0);
\coordinate (C3) at (2,1,0);
\coordinate (C4) at (1,1,0);
\coordinate (D1) at (1,0,1);
\coordinate (D2) at (2,0,1);
\coordinate (D3) at (2,1,1);
\coordinate (D4) at (1,1,1);

% 定义前面正方体的顶点
\coordinate (E1) at (0,0,1);
\coordinate (E2) at (1,0,1);
\coordinate (E3) at (1,1,1);
\coordinate (E4) at (0,1,1);
\coordinate (F1) at (0,0,2);
\coordinate (F2) at (1,0,2);
\coordinate (F3) at (1,1,2);
\coordinate (F4) at (0,1,2);

% 定义右边前面正方体的顶点
\coordinate (G1) at (1,0,1);
\coordinate (G2) at (2,0,1);
\coordinate (G3) at (2,1,1);
\coordinate (G4) at (1,1,1);
\coordinate (H1) at (1,0,2);
\coordinate (H2) at (2,0,2);
\coordinate (H3) at (2,1,2);
\coordinate (H4) at (1,1,2);

% 绘制左边正方体
\draw [dashed] (A1) -- (A2);
\draw [dashed, red] (A2) -- (A3);
\draw (A3) -- (A4);
\draw [dashed] (A4) -- (A1);
\draw [dashed, blue] (B1) -- (B2);
\draw [dashed] (B2) -- (B3);
\draw [blue] (B3) -- (B4);
\draw [dashed,blue] (B4) -- (B1);
\draw [dashed] (A1) -- (B1);
\draw [dashed, red] (A2) -- (B2);
\draw [red] (A3) -- (B3);
\draw (A4) -- (B4);

% 绘制右边正方体
\draw [dashed] (C1) -- (C2);
\draw (C2)-- (C3);
\draw (C3) -- (C4);
\draw [dashed, blue] (D1) -- (D2);
\draw [blue] (D2) -- (D3);
\draw [blue] (D3) -- (D4);
\draw [dashed] (D4) -- (D1);
\draw (C2) -- (D2);
\draw (C3) -- (D3);


% 绘制前面正方体
\draw (F1) -- (F2);
\draw [red] (F2) -- (F3);
\draw (F3) -- (F4);
\draw (F4) -- (F1);
\draw [dashed] (E1) -- (F1);
\draw [dashed, red] (E2) -- (F2);
\draw [red] (E3) -- (F3);
\draw (E4) -- (F4);

% 绘制右边前面正方体
\draw (H1) -- (H2);
\draw (H2) -- (H3);
\draw (H3) -- (H4);
\draw (G2) -- (H2);
\draw (G3) -- (H3);

% 标记顶点
%\foreach \point in {A1,A2,A3,A4,B1,B2,B3,B4,C1,C2,C3,C4,D1,D2,D3,D4,E1,E2,E3,E4,F1,F2,F3,F4,G1,G2,G3,G4,H1,H2,H3,H4}
%    \fill (\point) circle (1pt);
\end{tikzpicture}

\end{frame}

% 第7题
\begin{frame}[t]{试卷一}
第7题:生产一批零件,甲乙合作10天可以完成。若甲单独做,18天可以完成;那么,乙单独做要$(\qquad)$天能够完成。
\pause
\begin{columns}
\column{0.95\textwidth}
解:\\
设乙单独做需要$x$天能够完成,则: \\
$\dfrac{1}{10} = \dfrac{1}{18} + \dfrac{1}{x}$ \\
所以,$ \dfrac{1}{x} = \dfrac{1}{10} - \dfrac{1}{18} = \dfrac{9-5}{90} = \dfrac{4}{90}$ \\
所以,$x = 90 \div 4 = 22.5$ \\
答案:\alert{22.5}。
\end{columns}
\end{frame}

% 第8题
\begin{frame}[t]{试卷一}
第8题:若$a - b = 1$($a, b$是不为$0$的自然数),则$a, b$的最大公因数是$(\qquad)$,最小公倍数是$(\qquad)$。
\pause
\begin{columns}
\column{0.95\textwidth}
解:\\
$a - b = 1 \Rightarrow a = b + 1$,由此可以得出$a, b$必为一个奇数、一个偶数,最大公因数为1。如果不明白可以试着举几个例子去理解,如:$b = 1, a = 2$或$b = 2, a = 3$,最大公因数均为\alert{1}。\\
\pause
最小公倍数 = 两数的乘积 ÷ 最大公因数,因此$a, b$的最小公倍数为\alert{$a \cdot b$}
\end{columns}
\end{frame}

% 第9题
\begin{frame}[t]{试卷一}
第9题:一个盒子里有5个红球、1个绿球和2个黄球,每次摸出一个球后再放回盒中,这样摸600次,摸到绿球的次数约占总次数的$(\qquad)$,大约一共能摸到$(\qquad)$次黄球。
\pause
\begin{columns}
\column{0.95\textwidth}
解:\\
一共有$5 + 1 + 2 = 8$个球,其中有1个绿球、2个黄球。\\
所以,摸到绿球的概率 = $1 \div 8 \times 100\% = \alert{12.5\%}$;\\
\pause
摸到黄球的概率 = $2 \div 8 \times 100\% = \alert{25\%}$;\\
摸600次,大约能摸到黄球的次数  = $600 \times 25\% = 600 \div 4 = \alert{150}$ \\
答案:\alert{12.5\%}、\alert{150}。
\end{columns}
\end{frame}

% 第10题
\begin{frame}[t]{试卷一}
第10题:如右图,桌上有一张梯形的纸片, 折叠后得到的图形所覆盖桌面的面积是原来梯形的$\dfrac{3}{5}$。已知阴影面积为5平方厘米,原梯形面积是平方厘米。
\vspace{1em}
\pause
\begin{columns}
\column{0.95\textwidth}
解:\\
由折叠后得到的图形所覆盖桌面的面积是原来梯形的$\dfrac{3}{5}$可得重叠部分的面积为梯形纸片面积的$\dfrac{2}{5}$,所以阴影部分面积为原梯形面积的$1 - \dfrac{2}{5} \times 2 = \dfrac{1}{5}$。\\
因为阴影面积为5平方厘米,所以原梯形面积 = $5 \div \dfrac{1}{5} = 25$平方厘米。\\
答案:\alert{25}
\end{columns}

\end{frame}

% 第12题
\begin{frame}[t]{试卷一}
第12题:一个长方体的长、宽、高分别为$a$米、$b$米、$c$米,如果把它的高增加增加3米后新长方体的体积比原来增加$(\qquad)$立方米。
\pause
\vspace{1em}
\begin{columns}
\column{0.95\textwidth}
解法一:\\
原长方体的体积为:$a \times b \times c$(立方米) \\
新长方体的体积为:$a \times b \times (c+3)$(立方米) \\
增加的体积为:$a \times b \times (c+3) - a \times b \times c = 3ab$(立方米) \\
\pause
\vspace{1em}
解法二:\\
新增的部分也是一个长方体,长、宽、高分别为$a$米、$b$米、$3$米,
因此新增部分的体积为$V = a \cdot b \times 3 = 3ab$。\\
答案:\alert{3ab}。\\
\end{columns}
\end{frame}

\end{document}