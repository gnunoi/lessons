\documentclass[aspectratio=169]{ctexbeamer} %[t]:顶端对齐

\makeatletter
\def\input@path{{../../styles}}  % 
\makeatother
\usepackage{ubeamer}
\uBigPaper
\uSetBigFont

\date{\today}
\begin{document}

% 试卷二
% 第6题
\begin{frame}[t]{填空题}
第6题:一条绳子长50米,先用去$\dfrac{3}{5}$,又用去$5\dfrac{1}{2}$米,还剩$(\qquad)$米。
\pause
\vspace{1em}
\begin{columns}
\column{0.95\textwidth}
解:\\
$50 - 50 \times \dfrac{3}{5} - 5\dfrac{1}{2} = 50 \times \dfrac{2}{5} -  5\dfrac{1}{2} = 20 - 5\dfrac{1}{2} = 14\dfrac{1}{2}$(米) \\
\vspace{1em}
答案:\alert{$14.5$}或\alert{$14\dfrac{1}{2}$}
\end{columns}
\end{frame}

% 第8题
\begin{frame}[t]{填空题}
第8题:要配制含盐量为25\%的盐水。现有40克盐,需要加入$(\qquad)$克水才能满足要求。
\pause
\vspace{1em}
\begin{columns}
\column{0.95\textwidth}
解法一:\\
$40 \div 25\% - 40 = 160 - 40 = 120$ \\
\vspace{1em}

解法二:\\
设需要加入$x$克水,则:\\
$25 : 100 = 40 : 40 + x \Rightarrow 25 \times (40 + x) = 100 \times 40 \Rightarrow 40 + x = 160 \Rightarrow x = 120 $ \\
\vspace{1em}

答案:\alert{$120$}
\end{columns}
\end{frame}

% 第9题
\begin{frame}[t]{填空题}
第9题:按规律填空:60,44,36,32,$(\qquad)$,$(\qquad)$。
\pause
\vspace{1em}
\begin{columns}
\column{0.95\textwidth}
解:\\
找规律:\\
$60 - 44 = 16$ \\
$44 - 36 = 8$ \\
$36 - 32 = 4$ \\
我们发现:$8 : 16 = 1 : 2$,$4 : 8 = 1 : 2$,即相邻两数的差是上一个差的$\dfrac{1}{2}$,\\
所以32与之后的一个数的差为2,下一个差为1,则:\\
$32 - 2 = 30, 30 - 1 = 29 $ \\
\vspace{1em}
答案:\alert{( 30 ), ( 29 )}
\end{columns}
\end{frame}

% 第11题
\begin{frame}[t]{填空题}
第11题:一个长方体、一个圆柱和一个圆锥,它们的底面积和体积分别相等。如果长方体的高是6厘米,圆柱的高是$(\qquad)$厘米,圆锥的高是$(\qquad)$厘米。
\pause
\vspace{1em}
\begin{columns}
\column{0.95\textwidth}
说明:答案必须是\alert{最简整数比},必须记住\alert{约分}。\\
\vspace{1em}
解:\\
长方体的体积公式:$V_1 = S_1 \cdot h_1$ \\
圆柱体的体积公式:$V_2 = S_2 \cdot h_2$ \\
圆锥体的体积公式:$V_3 = \dfrac{1}{3} S_3 \cdot h_3$ \\
由题意可知:$V_1 = V_2 = V_3, S_1 = S_2 = S_3$,则:\\
$h_2 = h_1 = 6$(厘米) \\
$h_3 = 3h_1 = 3 \times 6 = 18$(厘米) \\
\vspace{1em}
答案:\alert{圆柱的高是( 6 )厘米,圆锥的高是( 18 )厘米。}
\end{columns}
\end{frame}

% 第12题
\begin{frame}[t]{填空题}
第12题:停车场上,轿车和两轮摩托车共25辆,共有70个车轮,则轿车有$(\qquad)$辆,两轮摩托车有$(\qquad)$辆。
\pause
\vspace{1em}
\begin{columns}
\column{0.95\textwidth}
解:\\
假设全部为两轮摩托车,则25辆两轮摩托车有$25 \times 2 = 50$个车轮,剩余车轮数量为$70 - 50 = 20$,\\
因为一辆轿车比一辆两轮摩托车多2个车轮,所以剩余20个车轮是$20 \div 2 = 10$辆轿车所有。\\
两轮摩托车数量$ = 25 - 10  = 15$ \\
\vspace{1em}
答案:\alert{轿车有( 10 )辆,两轮摩托车有( 15 )辆。}
\end{columns}
\end{frame}

% 第13题
\begin{frame}[t]{填空题}
第13题:一辆汽车从甲地开往乙地用15小时,返回时这两汽车每小时行驶全程的$\dfrac{1}{12}$,这辆汽车往返时间比是$(\qquad)$,往返速度比是$(\qquad)$。
\pause
\vspace{1em}
\begin{columns}
\column{0.95\textwidth}
说明:答案必须是\alert{最简整数比},必须记住\alert{约分}。\\
\vspace{1em}
解:\\
去用时15小时,回用时$1 \div \dfrac{1}{12} = 12$小时,则:\\
往返时间比为$15 : 12 = 5 : 4$ \\
因为速度比与时间比是反比例关系,所以往返速度比为$12 : 15 = 4 : 5$ \\
\vspace{1em}
答案:\alert{往返时间比为(5 : 4),往返速度比为(4 : 5)。}
\end{columns}
\end{frame}

% 判断题第1题
\begin{frame}[t]{判断题}
第1题:一个非零自然数不是奇数就是偶数,不是质数就是合数。$(\qquad)$
\pause
\vspace{1em}
\begin{columns}
\column{0.95\textwidth}
\vspace{1em}
解:\\
一个非零自然数不是奇数就是偶数,正确。\\
一个非零自然数不是质数就是合数,错误。因为1既不是质数也不是合数。 \\
\vspace{1em}
答案:\alert{( $ \times $ )}
\end{columns}
\end{frame}

% 判断题第6题
\begin{frame}[t]{判断题}
第6题:把一个长方形木框拉成一个平行四边形木框,木框的面积不变。$(\qquad)$
\pause
\vspace{1em}
\begin{columns}
\column{0.95\textwidth}
\vspace{1em}
解:\\
使用边界思维:如果四条边完全重合,则面积为\alert{0}。所以面积是不断变小的。\\
\vspace{1em}
或者换个解法:\\
平行四边形的面积公式:\alert{$S = a \cdot h$ }\\
长方形的面积公式:\alert{$S = a \cdot b$} \\
在平行四边形与长方形对应的边长相等的前提下,\alert{$h \le b$},\\
所以平行四边形的面积\alert{$\le$}长方形的面积,即面积逐渐变小。\\
\vspace{1em}
答案:\alert{( $ \times $ )}
\end{columns}
\end{frame}

% 精挑细选第1题
\begin{frame}[t]{精挑细选}
第1题:小花上午8时30分从家出发去姥姥家,下午2时到达姥姥家,她一共用了$(\qquad)$。\\
A. 6时 $\qquad$ B. 5时30分 $\qquad$ C. 60时30分 \\
\pause
\vspace{1em}
\begin{columns}
\column{0.95\textwidth}
\vspace{1em}
解:\\
下午2时,即$2 + 12 = 14$时。\\
用时 = $14:00 - 8:30 = 5:30$ \\
\vspace{1em}
答案:\alert{( B )}
\end{columns}
\end{frame}

% 精挑细选第3题
\begin{frame}[t]{精挑细选}
第3题:$4x + 8$写成$4(x+8)$,结果比原来$(\qquad)$。\\
A. 多4 $\qquad$ B. 少24 $\qquad$ C. 多24 \\
\pause
\vspace{1em}
\begin{columns}
\column{0.95\textwidth}
\vspace{1em}
解:\\
$4(x+8) - (4x+8) = 4x + 32 - 4x - 8 = 32 - 8 = 24$。\\
\vspace{1em}
答案:\alert{( B )}
\end{columns}
\end{frame}


% 精挑细选第8题
\begin{frame}[t]{精挑细选}
第8题:$a$是整数,$\dfrac{a}{7}$是假分数,$\dfrac{10}{a}$也是假分数,那么$a$的取值有$(\qquad)$种可能。\\
A. 3 $\qquad$ B. 4 $\qquad$ C. 2 \\
\pause
\vspace{1em}
\begin{columns}
\column{0.95\textwidth}
\vspace{1em}
解:\\
$\dfrac{a}{7} \ge 1 \Rightarrow a \ge 7$,且:\\
$\dfrac{10}{a} \ge 1 \Rightarrow a \le 7$,\\
因为$a$是正数,则$a$的取值范围是:$7, 8, 9, 10$,有4种可能。\\
\vspace{1em}
答案:\alert{( B )}
\end{columns}
\end{frame}

\end{document}