\documentclass[aspectratio=169]{ctexbeamer} %[t]:顶端对齐

\makeatletter
\def\input@path{{../../styles}}  % 
\makeatother
\usepackage{ubeamer}
\uBigPaper
\uSetBigFont

\date{\today}
\begin{document}

% 试卷三
% 第2题
\begin{frame}[t]{填空题}
第2题:一个数由3个亿、5个千万、4个十万组成,这个数写作$(\qquad)$,改成以“万”为单位的数是$(\qquad)$,省略“亿”后面的尾数约是$(\qquad)$。
\pause
\vspace{1em}
\begin{columns}
\column{0.95\textwidth}
解:\\
$3 \times 10^8 + 5 \times 10^7 + 4 \times 10^5 = 3 \, 5040\, 0000 $ (小学写法)\\
$3 \times 10^8 + 5 \times 10^7 + 4 \times 10^5 = 350 \, 400 \, 000 $ (中学写法)\\
\vspace{1em}
答案:这个数写作\alert{$3 \, 5040\, 0000$}或\alert{$350 \, 400 \, 000$},\\
改成以“万”为单位的数是\alert{$(3 \, 5040)$万}或\alert{$(35 \, 040)$万}, \\
省略“亿”后面的尾数约是\alert{$(\, 4\, )$亿}。
\end{columns}
\end{frame}

% 第5题
\begin{frame}[t]{填空题}
第5题:一个数的30\%是60,这个数是$(\qquad)$。
\pause
\vspace{1em}
\begin{columns}
\column{0.95\textwidth}
小学解法:\\
$60 \div 30\% = 60 \div 30 \times 100 = 200$ \\
\vspace{1em}
中学解法: \\
设这个数是$x$,则:
$x \times 30\% = 60 \Rightarrow x = 60 \div 30\% = 200$\\
\vspace{1em}
答案:这个数是\alert{$( \, 200 \, )$}
\end{columns}
\end{frame}

% 第7题
\begin{frame}[t]{填空题}
第7题:用边长为2分米的方砖铺地,需要1500块方砖,改用4分米的方砖铺这块地,需要$(\qquad)$块方砖。
\pause
\vspace{1em}
\begin{columns}
\column{0.95\textwidth}
解:\\
$1500 \times  2 \times 2 \div (4 \times 4) = 15 \times 400 \div 16 = 15 \times 25 = 375$ \\
\vspace{1em}
答案:需要\alert{$( \, 375 \, )$}块方砖。
\end{columns}
\end{frame}

% 第11题
\begin{frame}[t]{填空题}
第11题:一根绳子,第一次剪去全长的$\dfrac{3}{8}$,第二次剪去15米,还剩全长的25\%,这根绳子长$(\qquad)$米,还剩下$(\qquad)$米。
\pause
\vspace{1em}
\begin{columns}
\column{0.95\textwidth}
小学解法:\\
整条绳子长 = $15 \div (\dfrac{5}{8} - 25\%) = 15 \div (\dfrac{5}{8} - \dfrac{1}{4}) = 15 \div \dfrac{3}{8} = 15 \div 3 \times 8 = 40$(米)\\
还剩的绳长 = $40 \times 25\% = 10$(米)\\
\vspace{1em}
中学解法:\\
设整条绳子长$x$米,则有:$x - \dfrac{3}{8}x - 15 = 25\% x $ \\
所以, $\dfrac{5}{8}x - \dfrac{2}{8}x = 15 \Rightarrow x = 15 \div \dfrac{3}{8} = 40$ \\
还剩的绳长 = $40 \times 25\% = 10$(米)\\
\vspace{1em}
答案:这根绳子长\alert{$(\, 40 \, )$}米,还剩下\alert{$(\, 10 \, )$}米。
\end{columns}
\end{frame}

% 第12题
\begin{frame}[t]{填空题}
第12题:在5,6,7,8,9这5个数中,两数相加和为偶数的可能性是$\dfrac{(\qquad)}{(\qquad)}$。
\pause
\vspace{1em}
\begin{columns}
\column{0.95\textwidth}
解:\\
5个数选2个数相加共有$C_5^2 = 5 \times 4 \div 2 = 10$种选择。\\
两数和为偶数,则两数同为奇数或同为偶数。\\
3个奇数选2个数相加共有$C_3^2 = 3 \times 2 \div 2 = 3$种选择,\\
2个奇数选2个数相加共有$C_2^2 = 2 \times 1 \div 2 = 1$种选择,\\
因此,和为偶数共有$3 + 1 = 4$种选择。\\
所以,两数相加和为偶数的可能性是$ 4 \div 10 = \dfrac{2}{5}$ \\
\vspace{1em}
答案:两数相加和为偶数的可能性是\alert{$\dfrac{(\, 2 \, )}{(\, 5 \, )}$}。
\end{columns}
\end{frame}

% 第18题
\begin{frame}[t]{选择题}
第18题:长方形的周长一定,则长方形的长和宽$(\qquad)$。\\
A. 成正比例 \hspace{2em} B. 成反比例 \hspace{2em} C. 不成比例 \\
\pause
\vspace{1em}
\begin{columns}
\column{0.95\textwidth}
解:\\
设长方形的长和宽分别为$a$和$b$,则有$2(a + b) = \text{定值}$\\
$a + b = \text{定值} \div 2 = \text{定值}$ \\
正比例的定义:$a \div b = \text{定值}$ 或:$a : b = \text{定值}$ 或:$\dfrac{a}{b} = \text{定值}$\\
反比例的定义:$a \cdot b = \text{定值}$ \\
显然,已知条件既不满足正比例的条件,也不满足反比例的条件,因此长方形的长和宽不成比例关系。\\
\vspace{1em}
答案:\alert{$(\, C \, )$}。
\end{columns}
\end{frame}

% 第20题
\begin{frame}[t]{选择题}
鸡兔同笼,有20个头,56只脚,兔有$(\qquad)$只。\\
A. 8 \hspace{1em} B. 10 \hspace{1em} C. 12 \\
\pause
\vspace{1em}
\begin{columns}
\column{0.95\textwidth}
小学解法:\\
假设笼中全部是鸡,20只鸡有脚40只,剩余脚$56 - 40 = 16$只,\\
一只兔子比一只鸡多2只脚,故剩余16只脚为$16 \div 2 = 8$只兔子多出,\\
因此兔有\alert{$8$}只。\\
\vspace{1em}
中学解法:\\
假设笼中有$x$只兔,\\
则有$x \times 4 + (20 - x) \times 2 = 56 \Rightarrow 4x + 40 - 2x = 56 \Rightarrow 2x = 56 - 40 = 16 \Rightarrow x = 16 \div 2 = 8$,\\
答案:\alert{$(\, A \, )$}。
\end{columns}
\end{frame}

\end{document}