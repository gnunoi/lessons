\documentclass[aspectratio=169]{ctexbeamer} %[t]:顶端对齐

\makeatletter
\def\input@path{{../../styles}}  % 
\makeatother
\usepackage{ubeamer}
\uBigPaper
\uSetBigFont

\date{\today}
\begin{document}

% 质数与合数
\begin{frame}[t]{质数与合数}
\alert{质数}是指在\alert{大于1的自然数}中,除了\alert{1和本身}之外,\alert{不能}被其他自然数整除的数。例如,2、3、5、7、11 等都是质数。\\
\alert{合数}是指在\alert{大于1的自然数}中,除了\alert{1和本身}之外,\alert{还能}被其他自然数整除的数。例如,4、6、8、9、10、12 等都是合数。
\begin{columns}
\column{0.95\textwidth}
\begin{enumerate}[label={\arabic*.}]
\item 自然数0既\alert{不是质数},也\alert{不是合数}。 
\item 自然数1既\alert{不是质数},也\alert{不是合数}。 
\item 自然数2是最小的\alert{质数},也是唯一的\alert{偶质数}。 
\item \alert{除了2以外},所有其它的质数都是\alert{奇数}。 
\item 自然数3是最小的\alert{奇质数}。 
\item 自然数4是最小的\alert{合数},也是最小的\alert{偶合数}。 
\item 自然数9是最小的\alert{奇合数}。
\end{enumerate}
\end{columns}
\end{frame}

% 寻找1-100中的质数
\begin{frame}[t]{寻找1-100中的质数}
\begin{columns}
\column{0.4\textwidth}
见《五年级上册》第43页
\begin{enumerate}[label={\arabic*.}]
\item 划掉1;
\item 划掉除2外所有2的倍数;
\item 划掉除3外所有3的倍数;
\item 划掉除5外所有5的倍数;
\item 划掉除7外所有7的倍数;
\end{enumerate}
\column{0.6\textwidth}
\end{columns}

\begin{tikzpicture}[
      overlay, % 浮于正文上方
      remember picture,  % 允许引用页面坐标系
      shift={(current page.south east)},  % 将坐标系原点移到页面右下角
      xshift=-22cm,  %(避免贴边)
      yshift=18cm,  %(避免贴边)
      scale=1,
      transform shape
    ]
    \tikzset{cell/.style={minimum width=2em, minimum height=1.5em, draw=blue}}
    \foreach \row in {1,...,10} {
        \foreach \col in {1,...,10} {
            \node[cell] at (\col*2,-\row*1.6) {\pgfmathparse{(\row-1)*10+\col}\pgfmathprintnumber{\pgfmathresult}};
        }
    }
\end{tikzpicture}

\end{frame}

% 质因数分解
\begin{frame}[t]{质因数分解}
\alert{质数}是指在\alert{大于1的自然数}中,除了\alert{1和本身}之外,\alert{不能}被其他自然数整除的数。例如,2、3、5、7、11 等都是质数。
\begin{columns}
\column{0.95\textwidth}
\begin{enumerate}[label={\arabic*.}]
\item \alert{试除法:}用较小的质数依次去试除被分解的数,若这个质数能够整除被分解的数,则将其记录下来,然后用被分解数除以这个质数得到的商继续用同样的方法分解,直到商为 1 为止。
\item \alert{质因数分解树法:}将要分解的数写在树的顶端,然后依次将它分解成两个因数,直到分解到质数为止。分解时,每个分支的两个端点都是上一个数的因数,其中一个要是质数。
\item \alert{短除法:}将被分解的数写在短除符号右边,用一个能整除该数的质数作为除数写在短除符号左边,将商写在下方。然后继续对商进行同样的操作,直到商为 1 为止,左边的质数连乘起来就是该数的质因数分解。

\end{enumerate}
\end{columns}
\end{frame}

% 质因数分解练习
\begin{frame}[t]{质因数分解练习}
对以下数字进行质因数分解:
\begin{columns}
\column{0.95\textwidth}
\begin{enumerate}[label={\arabic*.}]
\item $96  = 8 \times 12 = 2 \times 2 \times 2 \times 2 \times 2 \times 3$
\item $72  = 8 \times 9 = 2 \times 2 \times 2 \times 3 \times 3$
\item $100  = 2 \times 2 \times 5 \times 5$
\item $75  = 3 \times 5 \times 5$
\item $50  = 2 \times 5 \times 5$
\item $36  = 4 \times 9 = 2 \times 2 \times 3 \times 3$
\item $32  = 4 \times 8 = 2 \times 2 \times 2 \times 2 \times 2$
\end{enumerate}
\end{columns}
\end{frame}

% 求最大公因数
\begin{frame}[t]{求最大公因数}
\begin{columns}
\column{0.95\textwidth}
\begin{enumerate}[label={\arabic*.}]
\item \alert{质因数分解法:}将每个数分解质因数,然后找出它们的公共质因数,将这些公共质因数相乘,所得的积就是它们的最大公因数。
\item \alert{辗转相除法(欧几里得算法):}对于两个正整数 a 和 b(a > b),用 a 除以 b 得到余数 r,若余数 r 为 0,则 b 是 a 和 b 的最大公因数;若余数 r 不为 0,则用 b 除以 r,再得到余数 r1,如此反复,直到余数为 0,此时的除数即为最大公因数。
\item \alert{更相减损术(不建议使用,效率太低):}对于两个正整数 a 和 b(a > b),用较大的数 a 减去较小的数 b,得到差 c,然后用 b 和 c 再次比较,继续相减,直到两个数相等,这个相等的数就是它们的最大公因数。
\end{enumerate}
\end{columns}
\end{frame}

% 求最大公因数练习
\begin{frame}[t]{求最大公因数练习}
\begin{columns}
\column{0.95\textwidth}
\begin{enumerate}[label={\arabic*.}]
\item 求100与75的最大公因数: $100 = 2 \times 2 \times 5 \times 5$, $75 = 3 \times 5 \times 5$,所以100与75的最大公因数为$5 \times 5 = 25$。
\item 求96与72的最大公因数: $96 = 2 \times 2 \times 2 \times 2 \times 2 \times 3$, $72 = 2 \times 2 \times 2 \times 3 \times 3$,所以96与72的最大公因数为$2 \times 2 \times 2 \times 3  = 24$。
\item 求72与48的最大公因数: $72 = 24 \times 3$, $48 = 24 \times 2$,所以72与48的最大公因数为$24$。
\item 求75与50的最大公因数: $75 = 25 \times 3$, $50 = 25 \times 2$,所以75与50的最大公因数为$25$。
\end{enumerate}
\end{columns}
\end{frame}

% 最大公因数习题
\begin{frame}[t]{最大公因数习题}
\begin{columns}
\column{0.95\textwidth}
\begin{enumerate}[label={\arabic*.}]
\item 箱子里面有75个白色乒乓球和50个黄色乒乓球,将箱子里面的乒乓球分成堆,要求每堆的白色乒乓球一样多,每堆的黄色乒乓球也一样多,最多可以分成$(\quad)$堆,每堆有白色乒乓球$(\quad)$个、黄色乒乓球$(\quad)$个。 \alert{答案:75与50的最大公约数为25,则最多可以分成25对,每堆白色乒乓球$75 \div 25 = 3$个,黄色乒乓球$75 \div 25 = 2$个。}
\item 用康乃馨和粉百合两种花做花束,康乃馨有96朵,粉百合有72朵,要求每束花里面的康乃馨数量相同,每束花里面的粉百合数量也相同,最多可以做$(\quad)$束花,每束花有康乃馨$(\quad)$朵、粉百合$(\quad)$朵。 \alert{答案:96与72的最大公约数为24,则最多可以做24束花,每束花有康乃馨$96 \div 24 = 4$朵、粉百合$72 \div 24 = 3$朵。}
\end{enumerate}
\end{columns}
\end{frame}

% 求最小公倍数
\begin{frame}[t]{求最小公倍数}
\begin{columns}
\column{0.95\textwidth}
\begin{enumerate}[label={\arabic*.}]
\item 最小公倍数 = 两数乘积 ÷ 最大公因数,如:
\item 求100与75的最小公倍数。 最小公倍数$ = 100 \times 75 \div 25 = 100 \times 3 = 300$
\item 求96与72的最小公倍数。 最小公倍数$ = 96 \times 72 \div 24 = 96 \times 3 = 288$
\end{enumerate}
\end{columns}
\end{frame}

\end{document}