\documentclass[aspectratio=169]{ctexbeamer} %[t]:顶端对齐

\makeatletter
\def\input@path{{../../styles}}  % 
\makeatother
\usepackage{ubeamer}
\uBigPaper
\uSetBigFont

\date{\today}
\begin{document}

% 试卷一
% 第2题
\begin{frame}[t]{鸡兔同笼问题}
笼中30头88脚,鸡兔各几何?\\
\pause
\begin{columns}
\column{0.95\textwidth}
小学解法:\\
假设笼中全部是鸡,30鸡有脚60只,剩余脚28只,\\
一只兔子比一只鸡多2只脚,故剩余28只脚为14只兔子多出,\\
因此鸡的数量为\alert{$30 - 14 = 16$}只,兔子\alert{$14$}只。\\
\vspace{1em}
中学解法:\\
假设笼中有$x$只鸡,则有$30-x$只兔,\\
则有$x \times 2 + (30 - x) \times 4 = 88 \Rightarrow 2x + 120 - 4x = 88 \Rightarrow 2x = 120 - 88 = 32 \Rightarrow x = 16$,\\
兔子数量 = $30 - 16 = 14$
因此鸡的数量为\alert{$16$}只,兔子\alert{$14$}只。
\end{columns}
\end{frame}

\end{document}