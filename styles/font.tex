% beamer定义的默认值
% 命令	字号大小(pt)	行距(pt)
% \tiny	5pt	6pt
% \scriptsize	7pt	9pt
% \footnotesize	8pt	10pt
% \small	 9pt	11pt
% \normalsize	10pt	12pt
% \large	12pt	14pt
% \Large	14pt	17pt
% \LARGE	17pt	20pt
% \huge	20pt	24pt
% \Huge	24pt	28pt

% 重新定义不同类型的文字大小
\renewcommand{\tiny}{\fontsize{10pt}{12pt}\selectfont} % 5,6
\renewcommand{\scriptsize}{\fontsize{14pt}{18pt}\selectfont} % 7,9
\renewcommand{\footnotesize}{\fontsize{16pt}{20pt}\selectfont} % 8,10
\renewcommand{\small}{\fontsize{18pt}{22pt}\selectfont} % 9,11
\renewcommand{\normalsize}{\fontsize{20pt}{24pt}\selectfont} % 10,12
\renewcommand{\large}{\fontsize{24pt}{28pt}\selectfont} % 12,14
\renewcommand{\Large}{\fontsize{28pt}{34pt}\selectfont} % 14,17
\renewcommand{\LARGE}{\fontsize{34pt}{40t}\selectfont} % 17,20
\renewcommand{\huge}{\fontsize{40pt}{48pt}\selectfont} % 20,24
\renewcommand{\Huge}{\fontsize{48pt}{56pt}\selectfont} % 24,28

% 设置不同文本类型的字体大小
\setbeamerfont{title}{size=\Huge}                % 主标题字号
\setbeamerfont{frametitle}{size=\huge}           % 框架标题字号
\setbeamerfont{theorem title}{size=\LARGE}       % 定理标题字号
\setbeamerfont{lemma title}{size=\LARGE}         % 引理标题字号
\setbeamerfont{corollary title}{size=\Large}    % 推论标题字号
\setbeamerfont{block title}{size=\Large}         % 代码块标题字号
\setbeamerfont{subtitle}{size=\large}            % 副标题字号
\setbeamerfont{headline}{size=\large}            % 页眉字号
\setbeamerfont{block body}{size=\normalsize}     % 代码块正文字号
\setbeamerfont{theorem body}{size=\normalsize}   % 定理正文字号
\setbeamerfont{lemma body}{size=\normalsize}     % 引理正文字号
\setbeamerfont{corollary body}{size=\normalsize} % 推论正文字号
\setbeamerfont{normal text}{size=\small}         % 普通文本字号
\setbeamerfont{author}{size=\small}              % 作者字号
\setbeamerfont{table title}{size=\small}         % 表格标题字号
\setbeamerfont{item}{size=\footnotesize}         % 项目列表项字号
\setbeamerfont{date}{size=\footnotesize}         % 日期字号
\setbeamerfont{caption}{size=\scriptsize}        % 图像/表格标题字号
\setbeamerfont{footline}{size=\scriptsize}       % 页脚字号
\setbeamerfont{footnote}{size=\scriptsize}       % 脚注字号

