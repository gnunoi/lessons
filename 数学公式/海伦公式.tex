\documentclass{article}
\usepackage{ctex}
\usepackage{amsmath}

\begin{document}

\title{海伦公式的详细推导}
\author{}
\date{}
\maketitle

设三角形的三边长分别为 \( a \), \( b \), \( c \),其中 \( s \) 是半周长,计算公式为:
\[
s = \frac{a + b + c}{2}
\]
目标是求三角形的面积 \( A \),并推导出海伦公式:
\[
A = \sqrt{s(s-a)(s-b)(s-c)}
\]

\section*{步骤 1:利用三角形的面积公式}
三角形的面积 \( A \) 可以通过两边及夹角来计算,公式为:
\[
A = \frac{1}{2}ab \sin C
\]
其中,\( a \) 和 \( b \) 是两边的长度,\(\angle C\) 是这两边之间的夹角。

\section*{步骤 2:应用余弦定理}
根据余弦定理,我们有:
\[
c^2 = a^2 + b^2 - 2ab \cos C
\]
从上式解得:
\[
\cos C = \frac{a^2 + b^2 - c^2}{2ab}
\]

\section*{步骤 3:使用面积公式}
将余弦值代入面积公式 \( A = \frac{1}{2}ab \sin C \):
\[
A = \frac{1}{2}ab \cdot \sqrt{1 - \left( \frac{a^2 + b^2 - c^2}{2ab} \right)^2}
\]
这就是面积与三边的关系。

\section*{步骤 4:转化为半周长形式}
通过一些代数变换,最终得出海伦公式:
\[
A = \sqrt{s(s-a)(s-b)(s-c)}
\]

\end{document}
