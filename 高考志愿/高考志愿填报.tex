\documentclass[aspectratio=169]{ctexbeamer} %[t]:顶端对齐

\makeatletter
\def\input@path{{../styles}}  % 
\makeatother
\usepackage{ubeamer}
\uBigPaper

\date{\today}
\begin{document}

% 学科门类
\begin{frame}{学科门类}
\centering{\fontsize{60}{72}\selectfont 12个学科门类}
\end{frame}
% 专业大类
\begin{frame}{专业大类}
\centering{\fontsize{60}{72}\selectfont 93个专业大类}
\end{frame}
% 专业
\begin{frame}{专业}
\centering{\fontsize{60}{72}\selectfont 845种专业}
\end{frame}

% 12个学科门类
\begin{frame}[t]{12个学科门类}
\begin{columns}
\column{0.95\textwidth}
\begin{enumerate}[label={\arabic*.}]
\item \textbf{哲学 }:包括1个专业大类,4种专业
\item \textbf{经济学} :包括4个专业大类,26种专业
\item \textbf{法学}:包括6个专业大类,54种专业
\item \textbf{教育学}:包括2个专业大类,35种专业
\item \textbf{文学}: 包括3个专业大类,127种专业
\item \textbf{历史学}:包括1个专业大类,9种专业
\item \textbf{理学}:包括12个专业大类,51种专业
\item \textbf{工学}:包括32个专业大类,287种专业
\item \textbf{农学}:包括7个专业大类,49种专业
\item \textbf{医学}:包括11个专业大类,66种专业
\item \textbf{管理学}:包括9个专业大类,71种专业
\item \textbf{艺术学}:包括5个专业大类,65种专业
\end{enumerate} 
\end{columns}
\end{frame}

% 专业代码说明
\begin{frame}[t]{专业代码说明:基本专业、特设专业、国家控制布点专业}
专业目录包含\alert{基本专业}和\alert{特设专业}。\\

\vspace{1cm}
基本专业一般是指学科基础\alert{比较成熟}、\alert{社会需求相对稳定}、\alert{布点数量相对较多}、\alert{继承性较好}的专业。\\

\vspace{1cm}
特设专业是满足经济社会发展\alert{特殊需求}所设置的专业,在专业代码后加\alert{“T”}表示。\\

\vspace{1cm}
专业目录中涉及\alert{国家安全}、\alert{特殊行业}等专业\alert{由国家控制布点},称为\alert{国家控制布点专业},在专业代码后加\alert{“K”}表示。

\end{frame}

% 专业代码说明
\begin{frame}[t]{专业代码说明(一):特设专业}
T代表特设专业\\

\vspace{1cm}
含义 :特设专业是针对不同高校办学特色,或适应近年来人才培养特殊需求设置的专业,通常是新兴的、具有广阔发展潜力的专业。教育主管部门会对特设专业进行动态管理,每年都需向社会公布。如果特设专业\alert{发展成熟},会\alert{成为基本专业};若\alert{办不下去},则将\alert{退出特设专业名单}。\\

\vspace{1cm}
特点 :特设专业具有特色鲜明、对口新兴行业、有一定发展潜力等特点,同时,其招生规模往往相对较小,\alert{就业需求相对较少},未来发展\alert{不确定性}更强一些。
\end{frame}

% 专业代码说明
\begin{frame}[t]{专业代码说明(二):国家控制布点专业}
K代表国家控制布点专业\\

\vspace{1cm}
含义 :\alert{国家控制布点专业}是指在\alert{特定领域}具有重要\alert{战略意义}或\alert{国家安全}考虑,受到严格开设限制的专业,高校开设这类专业需要经过审批。\\

\vspace{1cm}
特点 :这类专业总体可分为两类,一类是非常\alert{ “热门”} 的专业,如金融学、法学、信息安全、临床医学、工商管理、会计学、旅游管理等;另一类是\alert{“冷门”}但专业性强、就业面窄的专业,如侦查学、运动训练、航海技术、消防工程、法医学等。
\end{frame}

% 专业代码说明
\begin{frame}[t]{专业代码说明(三):特设控制布点专业}
TK 代表\alert{特设控制布点专业}\\

\vspace{1cm}
含义 :兼有\alert{特设专业}与\alert{控制布点专业}的属性和特性,既是适应特定需求而特别设置,又因专业性强、就业面窄或防止过多高校开设导致招生规模过大等原因,需要国家控制布点。\\

\vspace{1cm}
特点 :这类专业一般\alert{有明确的}、\alert{固定规模需求},同时对学生也有一些\alert{特殊要求},考上了基本不用担心就业,如国际组织与全球治理等专业。
\end{frame}

% 01 学科门类:哲学
\begin{frame}[t]{01 哲学}
哲学门类包括:1个专业大类,4种专业
\begin{spacing}{1.5} %设置行距
\begin{columns}
\column{0.95\textwidth}
{\large
\begin{enumerate}[label={\arabic*.}]
\item 010101 哲学
\item 010102 逻辑学
\item 010103K 宗教学
\item 010104T 伦理学
\end{enumerate} 
}
\end{columns}
\end{spacing}
\end{frame}

% 02 学科门类:经济学
\begin{frame}[t]{02 经济学}
经济学门类包括:4个专业大类,26种专业
\begin{spacing}{1.5} %设置行距
\begin{columns}
\column{0.95\textwidth}
{\large
\begin{enumerate}[label={\arabic*.}]
\item 0201 经济学类(9种专业)
\item 0202 财政学类(3种专业)
\item 0203 金融学类(11种专业)
\item 0204 经济与贸易类(3种专业)
\end{enumerate} 
}
\end{columns}
\end{spacing}
\end{frame}

% 03 学科门类:法学
\begin{frame}[t]{03 法学}
法学门类包括:6个专业大类,54种专业
\begin{spacing}{1.5} %设置行距
\begin{columns}
\column{0.95\textwidth}
{\large
\begin{enumerate}[label={\arabic*.}]
\item 0301 法学类(12种专业)
\item 0302 政治学类(6种专业)
\item 0303 社会学类(7种专业)
\item 0304 民族学类(1种专业)
\item 0305 马克思主义类(5种专业)
\item 0306 公安学类(23种专业)
\end{enumerate} 
}
\end{columns}
\end{spacing}
\end{frame}

% 04 学科门类:教育学
\begin{frame}[t]{04 教育学}
教育学门类包括:2个专业大类,35种专业
\begin{spacing}{1.5} %设置行距
\begin{columns}
\column{0.95\textwidth}
{\large
\begin{enumerate}[label={\arabic*.}]
\item 0401 教育学类(18种专业)
\item 0402 体育学类(17种专业)
\end{enumerate} 
}
\end{columns}
\end{spacing}
\end{frame}

% 05 学科门类:文学
\begin{frame}[t]{05 文学}
文学门类包括:3个专业大类,127种专业
\begin{spacing}{1.5} %设置行距
\begin{columns}
\column{0.95\textwidth}
{\large
\begin{enumerate}[label={\arabic*.}]
\item 0501 中国语言文学类(13种专业)
\item 0502 外国语言文学类(104种专业)
\item 0503 新闻传播学类(10种专业)
\end{enumerate} 
}
\end{columns}
\end{spacing}
\end{frame}

% 06 学科门类:历史学
\begin{frame}[t]{06 历史学}
历史学门类包括:1个专业大类,9种专业
\begin{spacing}{1.5} %设置行距
\begin{columns}
\column{0.95\textwidth}
{\large
\begin{enumerate}[label={\arabic*.}]
\item 0601 历史学类(9种专业)
\end{enumerate} 
}
\end{columns}
\end{spacing}
\end{frame}

% 07 学科门类:理学
\begin{frame}[t]{07 理学}
理学门类包括:12个专业大类,51种专业 \\

\begin{spacing}{1.2} %设置行距
\begin{columns}
\column{0.48\textwidth}
{\large
\begin{enumerate}[label={\arabic*.}]
\item 0701 数学类(4种专业)
\item 0702 物理学类(6种专业)
\item 0703 化学类(7种专业)
\item 0704 天文学类(1种专业)
\item 0705 地理科学类(4种专业)
\item 0706 大气科学类(4种专业)
\end{enumerate} 
}
\column{0.48\textwidth}
{\large
\begin{enumerate}[label={\arabic*.},start=7]
\item 0707 海洋科学类(5种专业)
\item 0708 地球物理学类(4种专业)
\item 0709 地质学类(4种专业)
\item 0710 生物科学类(6种专业)
\item 0711 心理学类(2种专业)
\item 0712 统计学类(4种专业)
\end{enumerate} 
}
\end{columns}
\end{spacing}
\end{frame}

% 08 学科门类:工学
\begin{frame}[t]{08 工学(一)}
工学门类包括:32个专业大类,287种专业 \\

\begin{spacing}{1.0} %设置行距
\begin{columns}
\column{0.45\textwidth}
{\large
\begin{enumerate}[label={\arabic*.}]
\item 0801 力学类(2种专业)
\item 0802 机械类(20种专业)
\item 0803 仪器类(3种专业)
\item 0804 材料类(23种专业)
\item 0805 能源动力类(7种专业)
\item 0806 电气类(10种专业)
\item 0807 电子信息类(22种专业)
\item 0808 自动化类(8种专业)
\end{enumerate} 
}
\column{0.45\textwidth}
{\large
\begin{enumerate}[label={\arabic*.},start=9]
\item 0809 计算机类(19种专业)
\item 0810 土木类(13种专业)
\item 0811 水利类(6种专业)
\item 0812 测绘类(6种专业)
\item 0813 化工与制药类(9种专业)
\item 0814 地质类(7种专业)
\item 0815 矿业类(8种专业)
\item 0816 纺织类(5种专业)
\end{enumerate} 
}
\end{columns}
\end{spacing}
\end{frame}

% 08 学科门类:工学
\begin{frame}[t]{08 工学(二)}
工学门类包括:32个专业大类,种专业 \\

\begin{spacing}{1.0} %设置行距
\begin{columns}
\column{0.45\textwidth}
{\large
\begin{enumerate}[label={\arabic*.}, start=17]
\item 0817 轻工类(7种专业)
\item 0818 交通运输类(12种专业)
\item 0819 海洋工程类(6种专业)
\item 0820 航空航天类(12种专业)
\item 0821 兵器类(8种专业)
\item 0822 核工程类(4种专业)
\item 0823 农业工程类(7种专业)
\item 0824 林业工程类(5种专业)
\end{enumerate} 
}
\column{0.45\textwidth}
{\large
\begin{enumerate}[label={\arabic*.},start=25]
\item 环境科学与工程类(7种专业)
\item 生物医学工程类(5种专业)
\item 食品科学与工程类(13种专业)
\item 建筑类(7种专业)
\item 安全科学与工程类(5种专业)
\item 生物工程类(3种专业)
\item 公安技术类(12种专业)
\item 交叉工程类(6种专业)
\end{enumerate} 
}
\end{columns}
\end{spacing}
\end{frame}

% 09 学科门类:农学
\begin{frame}[t]{09 农学}
农学门类包括:7个专业大类,49种专业
\begin{spacing}{1.2} %设置行距
\begin{columns}
\column{0.95\textwidth}
{\large
\begin{enumerate}[label={\arabic*.}]
\item 0901 植物生产类(17种专业)
\item 0902 自然保护与环境生态类(8种专业)
\item 0903 动物生产类(7种专业)
\item 0904 动物医学类(6种专业)
\item 0905 林学类(5种专业)
\item 0906 水产类(4种专业)
\item 0907 草学类(2种专业)
\end{enumerate} 
}
\end{columns}
\end{spacing}
\end{frame}

% 10 学科门类:医学
\begin{frame}[t]{10 医学}
医学门类包括:11个专业大类,66种专业 \\

\begin{spacing}{1.2} %设置行距
\begin{columns}
\column{0.48\textwidth}
{\large
\begin{enumerate}[label={\arabic*.}]
\item 1001 基础医学类(3种专业)
\item 1002 临床医学类(7种专业)
\item 1003 口腔医学类(1种专业)
\item 1004 公共卫生与预防医学类(6种专业)
\item 1005 中医学类(13种专业)
\item 1006 中西医结合类(1种专业)
\end{enumerate} 
}
\column{0.48\textwidth}
{\large
\begin{enumerate}[label={\arabic*.},start=7]
\item 1007 药学类(9种专业)
\item 1008 中药学类(6种专业)
\item 1009 法医学类(1种专业)
\item 1010 医学技术类(17种专业)
\item 1011 护理学类(2种专业)
\end{enumerate} 
}
\end{columns}
\end{spacing}
\end{frame}

% 12 学科门类:管理学
\begin{frame}[t]{12 管理学}
历史学门类包括:9个专业大类,71种专业
\begin{spacing}{1.0} %设置行距
\begin{columns}
\column{0.95\textwidth}
{\large
\begin{enumerate}[label={\arabic*.}]
\item 1201 管理科学与工程类(11种专业)
\item 1202 工商管理类(18种专业)
\item 1203 农业经济管理类(3种专业)
\item 1204 公共管理类(21种专业)
\item 1205 图书情报与档案管理类(3种专业)
\item 1206 物流管理与工程类(4种专业)
\item 1207 工业工程类(3种专业)
\item 1208 电子商务类(3种专业)
\item 1209 旅游管理类(5种专业)
\end{enumerate} 
}
\end{columns}
\end{spacing}
\end{frame}

% 13 学科门类:艺术学
\begin{frame}[t]{13 艺术学}
历史学门类包括:5个专业大类,65种专业
\begin{spacing}{1.0} %设置行距
\begin{columns}
\column{0.95\textwidth}
{\large
\begin{enumerate}[label={\arabic*.}]
\item 1301 艺术学理论类(3种专业)
\item 1302 音乐与舞蹈学类(15种专业)
\item 1303 戏剧与影视学类(18种专业)
\item 1304 美术学类(14种专业)
\item 1305 设计学类(15种专业)
\end{enumerate} 
}
\end{columns}
\end{spacing}
\end{frame}


\end{document}